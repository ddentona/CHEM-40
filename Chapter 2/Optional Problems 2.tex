\documentclass[11pt]{article}
\usepackage{amsmath}
\usepackage{siunitx}

\begin{document}
Ch. 2 Problems: 5, 8, 9, 39, 41, 55, 63, 67, 71, 75, 83, 85, 91, 93, 99, 109, 111, 115, 119, 121, 123, 125

7th Edition
\pagebreak
\section{Problem 121}
A car has a mileage rating of 38 mi per gallon of gasoline. How many miles can the car travel on 76.5 L of gasoline?

\section{Problem 123}
Consider these observations on two blocks of different unknown metals:

\begin{center}
    \begin{tabular*}{0.25\textwidth}{c c}
        Block name  & Volume \\
        Block A     & 125 \unit{\centi\meter^3}\\
        Block B     & 145 \unit{\centi\meter}
    \end{tabular*}
\end{center}          
        
If block A has a greater mass than block B, what can be said of the relative densities of the two metals? (Assume that both blocks are solid.)

\subsection{Solution}
Block A has a greater mass and a smaller volume. As such, since $\rho = \frac{V}{m}$, it has a greater density.

\section{Problem 125}
You measure the masses and volumes of two cylinders. The mass of cylinder 1 is 1.35 times the mass of cylinder 2. The volume of cylinder 1 is 0.792 the volume of cylinder 1. If the density of cylinder 1 is 3.85 \unit{\gram/\centi\meter^3}, what is the density of cylinder 2?

\subsection{Solution}
\begin{gather}
    m_1 = 1.35 m_2\\
    V_1 = 0.792 V_2\\
    \rho_1 = \frac{m_1}{V_1} = \frac{1.35 m_2}{0.792 V_2} = \frac{1.35}{0.792}\rho_2\\
    \rho_2 = \frac{0.792}{1.35}\rho_1 = \frac{0.792}{1.35}*3.85 \unit{\gram/\centi\meter^3} = \boxed{2.26 \unit{\gram/\centi\meter^3}}
\end{gather}

\end{document}