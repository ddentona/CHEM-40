\documentclass[10pt]{article}
\usepackage[export]{adjustbox}
\usepackage{amsmath}
\usepackage{array}
\usepackage[makeroom]{cancel}
\usepackage{chemfig}
\usepackage{enumitem}
\usepackage{float}
\usepackage{graphicx}
%Load mhchem using some package options
\usepackage[version=4]{mhchem}
\usepackage{multicol}
\usepackage{siunitx}
\usepackage{wrapfig}

\title{
    Problem Set \#12
    \\  \small
    CHEM101A: General College Chemistry
    }
\author{Donald Aingworth}
\date{November 7, 2025}

\newcommand{\E}[1]{\times 10^{#1}}
\newcommand{\hc}{1.9864748\E{-25}\,\unit{\joule\,\meter}}
\newcommand{\U}[1]{\underline{#1}}

\begin{document}
    \DeclareSIUnit{\atm}{atm}
    \DeclareSIUnit{\molarity}{M}
    \DeclareSIUnit{\M}{M}
    \DeclareSIUnit{\torr}{torr}

    \maketitle

    \setcounter{section}{10}
    \pagebreak
    \section{Topic F Problem 11}
        There are two ions that have the empirical formula \ce{CNO-}. 
        The cyanate ion has the three atoms in the order N-C-O, while the fulminate ion has the three atoms in the order C-N-O.
        \begin{enumerate}[label=\alph*)]
            \item   Draw all of the resonance structures that satisfy the octet rule for each ion. Include all non-zero formal charges in your structures.
            \item   Based on formal charges, which of these resonance structures would you expect to make a significant contribution to the actual structures of the cyanate and fulminate ions?
            \item   Based on your resonance structures for the cyanate ion, which atom (or atoms) carries the negative charge?
            \item   Repeat part c for the fulminate ion.
        \end{enumerate}
        
        \subsection{Solution}

    \pagebreak
    \section{Topic F Problem 12}
        Draw all of the reasonable resonance structures that satisfy the octet rule for each of the following substances, and tell which structures will be major contributors to the actual structure of the molecule or ion.
        \begin{enumerate}[label=\alph*)]
            \item   \ce{CO3^2-}
            \item   \ce{HCO3-} (the H is attached to one of the O atoms in \ce{CO3^2-})
            \item   \ce{H2CO3} (each H is attached to one of the O atoms in \ce{CO3^2-})
        \end{enumerate}
        
        \subsection{Solution}


    % \pagebreak
    \section{Topic F Problem 13}
        Each of the three substances in the previous problem contains three carbon-oxygen bonds.
        For each substance, which carbon-oxygen bonds are the same length?
        
        \subsection{Solution}
            \begin{enumerate}[label=\alph*/]
                \item   All three are the same length.
                \item   The ones not with oxygens not bonded to hydrogens.
                \item   The ones with oxygens bonded to hydrogens.
            \end{enumerate}


    \pagebreak
    \section{Topic F Problem 14}
        Rank the bonds labeled A through E in order of carbon-carbon bond distance, starting with the shortest bond distance. 
        Be sure to consider resonance!
        \begin{center}
            \includegraphics[width=\textwidth]{img-F14.png}
        \end{center}
        
        \subsection{Solution}


    \pagebreak
    \section{Topic F Problem 15}
        What are the approximate values (in degrees) for the bond angles labeled A through E in the molecule below?
        \begin{center}
            \includegraphics[width=0.7\textwidth]{img-F15.png}
        \end{center}
        
        \subsection{Solution}
            \begin{enumerate}[label=\Alph*:]
                \item   120\unit{\degree}
                \item   109.5\unit{\degree}
                \item   90\unit{\degree}
                \item   120\unit{\degree}
                \item   180\unit{\degree}
            \end{enumerate}


    \pagebreak
    \section{Topic F Problem 16}
        For each of the following pairs of molecules, one molecule is polar while the other is not.
        Tell which molecule is polar, and justify your answer using Lewis structures and VSEPR.
        \begin{multicols}{2}
            \begin{enumerate}[label=\alph*)]
                \item   \ce{CO2} and \ce{SO2}
                \item   \ce{NCl3} and \ce{BCl3}
                \item   \ce{SiF4} and \ce{SeF4}
                \item   \ce{IF5} and \ce{PF5}
            \end{enumerate}
        \end{multicols}
        
        \subsection{Solution}


    \pagebreak
    \section{Topic F Problem 17}
        What is the hybridization on each of the following atoms?
        \begin{enumerate}[label=\alph*)]
            \item   The nitrogen atom in \ce{NH3}
            \item   The phosphorus atom in \ce{PF5}
            \item   The carbon atom in \ce{CO2}
            \item   The carbon atom in \ce{CH2O}
            \item   The iodine atom in \ce{IF5}
            \item   The nitrogen atom in \ce{NO2-}
        \end{enumerate}
        
        \subsection{Solution}


    % \pagebreak
    \section{Topic F Problem 18}
        List all of the valence orbitals in each of the atoms listed in the previous problem, and tell how many of each there are. 
        Do not list orbitals on the outer atoms.
        
        \subsection{Solution}


    \pagebreak
    \section{Topic F Problem 19}
        \begin{wrapfigure}{r}{0.15\textwidth}
            \vspace{-30pt}
            \includegraphics[width=0.15\textwidth]{img-F19.png}
            % \label{fig:wrapfig}
        \end{wrapfigure}
        According to the valence-bond model, what atomic orbitals overlap to form each of the following chemical bonds?
        \begin{enumerate}[label=\alph*)]
            \item   The \ce{H-Br} bond in \ce{HBr}
            \item   The \ce{I-I} bond in \ce{I2}
            \item   A \ce{C-F} bond in \ce{CF4}
            \item   The \ce{C-H} bond in \ce{HCN}
            \item   A \ce{C-Cl} bond in \ce{COCl2}
            \item   The \ce{C-C} bond in acetic acid (the molecule on the right)
            \item   The \ce{C-O} single bond in acetic acid
        \end{enumerate}
        
        \subsection{Solution}

    \pagebreak
    \section{Topic F Problem 20}
        According to the valence-bond model, what atomic orbitals hold the nonbonding electrons in a molecule of water?

        \subsection{Solution}

    \pagebreak
    \section{Topic F Problem 21}
        Questions a through d (on the next page) refer to the molecule below.
        \begin{center}
            \includegraphics[width=0.3\textwidth]{img-F21.png}
        \end{center}

        \begin{enumerate}[label=\alph*)]
            \item   How many sigma bonds and how many pi bonds are there in this molecule?
            \item   What atomic orbitals form the carbon-nitrogen double bond? Tell whether each orbital is involved in a sigma bond or a pi bond.
            \item   What atomic orbitals form the carbon-nitrogen triple bond? Tell whether each orbital is involved in a sigma bond or a pi bond.
            \item   What atomic orbitals contain the nonbonding electrons?
        \end{enumerate}
        
        \subsection{Solution}

    \pagebreak
    \tableofcontents
\end{document}