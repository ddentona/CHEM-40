\documentclass[10pt]{article}
\usepackage{amsmath}
\usepackage{enumitem}
%Load mhchem using some package options
\usepackage[version=4]{mhchem}
\usepackage{multicol}
\usepackage{siunitx}

\title{
    Problem Set \#2
    \\  \small
    CHEM101A: General College Chemistry
    }
\author{Donald Aingworth IV}
\date{August 29, 2025}

\begin{document}
    \maketitle

    \pagebreak
    \section{Topic A Problem 12}
        What mass of \ce{Fe2O3} would react with 20.00 g of \ce{Zn}? 
        The chemical equation for this reaction is: 
        \begin{equation}
            \ce{3Zn + Fe2O3 -> 2Fe + 3ZnO}
        \end{equation}

        \subsection{Solution}
            The simple stoichiometry is the way to go here.
            \begin{equation}
                20.00 \unit{\gram} \times \frac{1\,\unit{\mole}\,\ce{Zn}}{65.38 \unit{\gram}} \times \frac{1\,\ce{Fe2O3}}{3\,\ce{Zn}} \times \frac{159.7\,\unit{\gram}\,\ce{Fe2O3}}{1\,\unit{\mole}\,\ce{Fe2O3}}  =   \boxed{16.28\,\unit{\gram}\,\ce{Fe2O3}}
            \end{equation}
            
    \pagebreak
    \section{Topic A Problem 13}
        x moles of \ce{C4H10} reacts with oxygen according to the following equation: 
        \begin{equation}
            \ce{2C4H10 + 13O2 -> 8CO2 + 10H2O}
        \end{equation}
        a) How many moles of water are formed?\\
        b) How many moles of oxygen are consumed?

        \subsection{Solution (a)}
            The ratio of \ce{C4H10} used to \ce{H2O} created in this reaction is 1:5.
            With x moles of \ce{C4H10}, that would gives us \boxed{5x\,\unit{\mole}\,\ce{H2O}}.

        \subsection{Solution (b)}
            The ratio of \ce{C4H10} used to \ce{O2} consumed in this reaction is 2:13.
            With x moles of \ce{C4H10}, that would gives us \boxed{\frac{13}{2}x\,\unit{\mole}\,\ce{O2}}.

    \pagebreak
    \section{Topic A Problem 14}
        10.00 g of \ce{N2} is mixed with 33.61 g of \ce{F2}, and the elements react according to the following equation: 
        \begin{equation}
            \ce{N2 + 3F2 -> 2NF3}
        \end{equation}
        a) Which element is the limiting reactant?\\
        b) What is the theoretical yield of \ce{NF3}?\\
        c) If the reaction goes to completion, how many grams of the excess reactant will remain?\\
        d) Set up an ICE table for this reaction.

        \subsection{Solution (a)}
            First, we calculate the theoretical yields for each for the reactants.
            \begin{align}
                m_{\ce{N2}} &=  10.00\,\unit{\gram} \times \frac{1\,\unit{\mole}\,\ce{N2}}{28.02\,\unit{\gram}\,\ce{N2}} 
                                                    \times \frac{2\,\ce{NF3}}{1\,\ce{N2}}
                                                    \times \frac{71.01\,\unit{\gram}\,\ce{NF3}}{1\,\unit{\mole}\,\ce{NF3}}
                    =   50.69\,\unit{\gram}\,\ce{NF3}
                \\
                m_{\ce{F2}} &=  33.61\,\unit{\gram} \times \frac{1\,\unit{\mole}\,\ce{F2}}{38.00\,\unit{\gram}\,\ce{F2}} 
                                                    \times \frac{2\,\ce{NF3}}{3\,\ce{F2}}
                                                    \times \frac{71.01\,\unit{\gram}\,\ce{NF3}}{1\,\unit{\mole}\,\ce{NF3}}
                    =   41.87\,\unit{\gram}\,\ce{NF3}
            \end{align}

            With a lower final mass, \boxed{\ce{F2}} is the limiting reactant.

        \subsection{Solution (b)}
            The theoretical yield was found in part (a).
            \boxed{41.87\,\unit{\gram}\,\ce{NF3}}
        
        \subsection{Solution (c)}
            Use a similar strategy to part (a).
            \begin{equation}
                33.61\,\unit{\gram} \times \frac{1\,\unit{\mole}\,\ce{F2}}{38.00\,\unit{\gram}\,\ce{F2}} 
                                    \times \frac{1\,\ce{N2}}{3\,\ce{F2}}
                                    \times \frac{28.02\,\unit{\gram}\,\ce{N2}}{1\,\unit{\mole}\,\ce{N2}}
                    =   8.261\,\unit{\gram}\,\ce{NF3}
            \end{equation}

            Subtract this from the available mass of \ce{N2} to get the final \ce{N2}.
            \begin{equation}
                10.00\,\unit{\gram}\,\ce{N2} - 8.261\,\unit{\gram}\,\ce{N2}    =   \boxed{1.74\,\unit{\gram}\,\ce{N2}}
            \end{equation}
        
        \subsection{Solution (d)}
            I used tabular for this table.
            Please excuse any poor or improper formatting.

            \begin{center}
                \begin{tabular}{| c | c | c | c |}
                    \hline
                    \unit{\mole} & \ce{N2} &\ce{+ 3F2} &\ce{-> 2NF3} \\
                    \hline 
                    I   & 0.3569    &0.8844     &0\\
                    \hline
                    C   & -0.2948   &-0.8844    &0.5896\\
                    \hline
                    E   & 0.0621    &0          &0.5896\\ \hline
                \end{tabular}
            \end{center}

    \pagebreak
    \section{Topic A Problem 15}
        a) If 58.26 g of iodine reacts with excess aluminum, what is the theoretical yield of aluminum iodide? 
        The reaction is \ce{2 Al + 3 I2 -> 2 AlI3}.\\
        b) If 56.11 g of aluminum iodide is actually formed in the reaction in part a, what is the percent yield of aluminum iodide?

        \subsection{Solution (a)}
        

    \pagebreak
    \section{Topic A Problem 16}
        A chemist mixes 16.00 g of \ce{HCl} with 10.00 g of \ce{Mg} and obtains an 81.3\% yield of \ce{MgCl2}.
        What mass of \ce{MgCl2} did the chemist obtain? 
        The chemical reaction is:
        \begin{equation}
            \ce{Mg + 2 HCl -> MgCl2 + H2}
        \end{equation}

        \subsection{Solution}

    % \pagebreak
    % \section{Topic A Problem 17}
    %     How many milliliters of liquid \ce{Br2} (density = 3.1 g/mL) will react with 6.143 g of \ce{Cr}, if the product of this reaction is \ce{CrBr3}?

    %     \subsection{Solution}

    % \pagebreak
    % \section{Topic A Problem 18}
    %     Ethane (\ce{C2H6}) reacts with oxygen according to the following chemical equation:
    %     \begin{equation}
    %         \ce{2 C2H6 + 7 O2 -> 4 CO2 + 6 H2O}
    %     \end{equation}
    %     a) If you mix 5 moles of \ce{C2H6} with 13 moles of \ce{O2}, how many moles of each substance will you end up with, assuming the reaction goes to completion? 
    %     Include an ICE table in your answer.\\
    %     b) If you mix 81.43 g of \ce{C2H6} with 194.60 g of \ce{O2}, how many grams of each substance will you end up with, assuming the reaction goes to completion? 
    %     Include an ICE table in your answer. 
    %     (Note: your ICE table should be in terms of moles.)\\
    %     c) A chemist mixes 3.414 moles of \ce{O2} with an unknown number of moles of \ce{C2H6}. 
    %     The chemist obtains 1.657 moles of \ce{O2}. 
    %     How many moles of \ce{C2H6} must have been present originally, assuming the reaction went to completion? 
    %     Include an ICE table in your answer.

    %     \subsection{Solution}

    % \pagebreak
    % \section{Topic A Problem 19}
    %     Ammonia reacts with oxygen according to the following chemical equation:
    %     \begin{equation}
    %         \ce{4 NH3 + 3 O2 -> 2 N2 + 6 H2O}
    %     \end{equation}
    %     Suppose you mix x moles of \ce{NH3} with y moles of \ce{O2}.\\
    %     a) If \ce{NH3} is the limiting reactant, how many moles of each substance will you end up with, assuming the reaction goes to completion? 
    %     Include an ICE table in your answer.\\
    %     b) If \ce{O2} is the limiting reactant, how many moles of each substance will you end up with, assuming the reaction goes to completion? 
    %     Include an ICE table in your answer.\\
    %     c) If you end up with 0.4y moles of \ce{O2}, what must the relationship be between x and y, assuming the reaction goes to completion?

    %     \subsection{Solution}

    % \pagebreak
    % \section{Topic A Problem 20}
    %     You have x grams of \ce{Na2Cr2O7}. 
    %     How many grams of CrCl3 will be formed if the \ce{Na2Cr2O7} undergoes the reaction below? 
    %     Express your answer in terms of x.
    %     \begin{equation}
    %         \ce{Na2Cr2O7 + 3 Zn + 14 HCl -> 2 CrCl3 + 3 ZnCl2 + 2 NaCl + 7 H2O}
    %     \end{equation}

    %     \subsection{Solution}

    % \pagebreak
    % \section{Topic A Problem 21}
    %     A metal sample weighing 1.410 g contains a mixture of copper and aluminum. 
    %     When excess \ce{HCl} is added to this sample, the aluminum reacts as follows:
    %     \begin{equation}
    %         \ce{2 Al + 6 HCl -> 2 AlCl3 + 3 H2}
    %     \end{equation}
    %     849 mL of \ce{H2} (density 0.08264 g/L) is produced. 
    %     Calculate the mass percentage of each element in the original sample. 
    %     Note that copper does not react with \ce{HCl}.

    %     \subsection{Solution}

    % \pagebreak
    % \section{Topic A Problem 22}
    %     A chemist has a mixture of \ce{AgNO3} and \ce{KNO3} that weighs a total of 4.177 g. 
    %     The chemist dissolves the mixture in water and then adds a solution of \ce{NaOH}. 
    %     The AgNO3 reacts with the \ce{NaOH} as follows:
    %     \begin{equation}
    %         \ce{2 AgNO3(aq) + 2 NaOH(aq) -> Ag2O(s) + 2 NaNO3(aq) + H2O(l)}
    %     \end{equation}
    %     The chemist finds that 1.080 grams of \ce{Ag2O} were formed. 
    %     Calculate the mass percentages of \ce{AgNO3} and \ce{KNO3} in the original mixture. 
    %     (Note that \ce{KNO3} does not react with \ce{NaOH}.)

    %     \subsection{Solution}

    % \pagebreak
    % \section{Topic A Problem 23}
    %     A 25.000 g sample of sulfur is burned. 
    %     Some of the sulfur reacts to form \ce{SO2}:
    %     \begin{equation}
    %         \ce{S + O2 -> SO2}
    %     \end{equation}
    %     The rest of the sulfur reacts to form \ce{SO3}:
    %     \begin{equation}
    %         \ce{2 S + 3 O2 -> 2 SO3}
    %     \end{equation}
    %     The total mass of products (\ce{SO2} and \ce{SO3}) is 58.723 g. 
    %     Calculate the masses of \ce{SO2} and \ce{SO3} in this mixture.

    %     \subsection{Solution}

    % \pagebreak
    % \section{Topic B Problem 1}
    %     Answer each of the following questions about making solutions.\\
    %     a) If you dissolve 4.18 g of solid \ce{Mg(NO3)2} in enough water to make 150 mL of solution, what will be the molarity of the resulting solution?\\
    %     b) If you need to make 100 mL of 1.08 M \ce{CaCl2}, what mass of solid CaCl2 will you need?\\
    %     c) You have 25.0 g of solid \ce{KCl}, and you use all of it to make a 0.500 M \ce{KCl} solution. 
    %     What volume of solution did you make?

    %     \subsection{Solution}

    % \pagebreak
    % \section{Topic B Problem 2}
    %     Answer the following questions about dilutions.\\
    %     a) If you add 100 mL of water to 10 mL of 0.605 M \ce{HCl}, what will be the molarity of the resulting solution?\\
    %     b) You have 200 mL of 1.50 M \ce{HNO3}. 
    %     If you wish to dilute this solution to a final concentration of 0.300 M, what volume of water should you add?\\
    %     c) You need to make 1.50 liters of 0.400 M \ce{NaOH} by diluting a 2.00 M \ce{NaOH} solution. 
    %     What volume of the 2.00 M \ce{NaOH} should you use, and what volume of water should you add to it?

    %     \subsection{Solution}

    % \pagebreak
    % \section{Topic B Problem 3}
    %     All of the compounds below dissolve in water. 
    %     Which of them are strong electrolytes, which are weak electrolytes, and which are nonelectrolytes?
    %     \begin{multicols}{4}
    %         \begin{enumerate}[label=\alph*)]
    %             \item   \ce{NaCl}
    %             \item   \ce{Mg(NO3)2}
    %             \item   \ce{HClO2}
    %             \item   \ce{MgCrO4}
    %             \item   \ce{H3PO4}
    %             \item   \ce{AgF}
    %             \item   \ce{C2H5OH}
    %             \item   \ce{HC3H5O3}
    %             \item   \ce{CH3CN}
    %             \item   \ce{H2SO4}
    %             \item   \ce{NH4Br}
    %             \item   \ce{(CH3)2CO}
    %         \end{enumerate}
    %     \end{multicols}
    %     \subsection{Solution}

    % \pagebreak
    % \section{Topic B Problem 4}
    %     What ions (if any) are present in each of the following solutions, and what is the molar concentration of each ion?
        
    %     \begin{multicols}{2}
    %         \begin{enumerate}[label=\alph*)]
    %             \item   0.1 M \ce{NaBr}
    %             \item   0.04 M \ce{KNO3}
    %             \item   0.2 M \ce{FeCl3}
    %             \item   1.5 M \ce{(NH4)2SO4}
    %         \end{enumerate}
    %     \end{multicols}

    %     \subsection{Solution}

    % \pagebreak
    % \section{Topic B Problem 5}
    %     How many moles of each ion are present in 175 mL of 0.147 M \ce{Fe2(SO4)3}?

    %     \subsection{Solution}

    % \pagebreak
    % \section{Topic B Problem 6}
    %     Which of the following are acceptable ways to make one liter of 1 M \ce{NaCl}?\\
    %     a) Put 1 liter of water into a container, then add 1 mole of NaCl and stir until the \ce{NaCl} dissolves.\\
    %     b) Put 1 mole of \ce{NaCl} into a container, then add 1 liter of water and stir until the \ce{NaCl} dissolves.
    %     c) Put 1 mole of \ce{NaCl} into a container, then add water with stirring until the total volume reaches 1 liter.

    %     \subsection{Solution}

    % \pagebreak
    % \section{Topic B Problem 7}
    %     Janet dissolves 6.50 g of solid potassium phosphate in enough water to make 100.0 mL of solution. 
    %     Farid then adds enough water to the solution to reduce the concentration of potassium ions to 0.250 M. 
    %     How much water did Farid add?

    %     \subsection{Solution}

    % \pagebreak
    % \section{Topic B Problem 8}
    %     Gerardo dissolves 8.213 g of solid \ce{Mg(NO3)2} in enough water to make 200.0 mL of solution.
    %     Marciela then adds enough solid \ce{Al(NO3)3} to increase the concentration of nitrate ions to 0.900 M. 
    %     Assuming that the solution volume does not change significantly, what mass of \ce{Al(NO3)3} did Marciela add?

    %     \subsection{Solution}

    % \pagebreak
    % \section{Topic B Problem 9}
    %     Chantelle dissolves 2.35 g of \ce{NaCl}, 3.12 g of \ce{CaCl2}, and 1.88 g of \ce{FeCl3} in enough water to make 175 mL of solution. 
    %     What is the molarity of chloride ions in this solution?

    %     \subsection{Solution}

    % \pagebreak
    % \section{Topic B Problem 10}
    %     Wenzhou prepares 200 mL of a solution of \ce{SnCl4} in which the concentration of chloride ions is 0.240 M.\\
    %     a) What is the molarity of the \ce{SnCl4} solution (i.e. what should the bottle be labeled)?\\
    %     b) What mass of \ce{SnCl4} did Wenzhou use?

    %     \subsection{Solution}

    % \pagebreak
    % \section{Topic B Problem 11}
    %     A beaker holds x liters of 0.2 M \ce{AlBr3}. 
    %     Give answers to each part below in terms of x.\\
    %     a) How many moles of aluminum ions are in this solution?\\
    %     b) How many moles of bromide ions are in this solution?\\
    %     c) How much water must you add if want to dilute the original solution to a concentration of 0.02 M?

    %     \subsection{Solution}

    % \pagebreak
    % \section{Topic B Problem 12}
    %     Using the solubility rules, determine which of the following compounds are insoluble in water. 
    %     There is a solubility rules handout available in Canvas.
    %     \begin{multicols}{3}
    %         \begin{enumerate}[label=\alph*)]
    %             \item   \ce{K2Cr2O7} 
    %             \item   \ce{Mn(NO3)2}
    %             \item   \ce{FeS}
    %             \item   \ce{ZnBr2}
    %             \item   \ce{MgSO4}
    %             \item   \ce{NaHCO3}
    %             \item   \ce{Ba3(PO4)2}
    %         \end{enumerate}
    %     \end{multicols}

    %     \subsection{Solution}

    \pagebreak
    \tableofcontents

\end{document}