\documentclass[10pt]{article}
\usepackage{amsmath}
\usepackage[makeroom]{cancel}
\usepackage{enumitem}
%Load mhchem using some package options
\usepackage[version=4]{mhchem}
\usepackage{multicol}
\usepackage{siunitx}

\title{
    Problem Set \#2
    \\  \small
    CHEM101A: General College Chemistry
    }
\author{Donald Aingworth IV}
\date{August 29, 2025}

\begin{document}
    \DeclareSIUnit{\molarity}{M}
    \DeclareSIUnit{\M}{M}

    \maketitle

    \pagebreak
    \section{Topic A Problem 12}
        What mass of \ce{Fe2O3} would react with 20.00 g of \ce{Zn}? 
        The chemical equation for this reaction is: 
        \begin{equation}
            \ce{3Zn + Fe2O3 -> 2Fe + 3ZnO}
        \end{equation}

        \subsection{Solution}
            The simple stoichiometry is the way to go here.
            \begin{equation}
                20.00 \unit{\gram} \times \frac{1\,\unit{\mole}\,\ce{Zn}}{65.38 \unit{\gram}} \times \frac{1\,\ce{Fe2O3}}{3\,\ce{Zn}} \times \frac{159.7\,\unit{\gram}\,\ce{Fe2O3}}{1\,\unit{\mole}\,\ce{Fe2O3}}  =   \boxed{16.28\,\unit{\gram}\,\ce{Fe2O3}}
            \end{equation}
            
    \pagebreak
    \section{Topic A Problem 13}
        x moles of \ce{C4H10} reacts with oxygen according to the following equation: 
        \begin{equation}
            \ce{2C4H10 + 13O2 -> 8CO2 + 10H2O}
        \end{equation}
        a) How many moles of water are formed?\\
        b) How many moles of oxygen are consumed?

        \subsection{Solution (a)}
            The ratio of \ce{C4H10} used to \ce{H2O} created in this reaction is 1:5.
            With x moles of \ce{C4H10}, that would gives us \boxed{5x\,\unit{\mole}\,\ce{H2O}}.

        \subsection{Solution (b)}
            The ratio of \ce{C4H10} used to \ce{O2} consumed in this reaction is 2:13.
            With x moles of \ce{C4H10}, that would gives us \boxed{\frac{13}{2}x\,\unit{\mole}\,\ce{O2}}.

    \pagebreak
    \section{Topic A Problem 14}
        10.00 g of \ce{N2} is mixed with 33.61 g of \ce{F2}, and the elements react according to the following equation: 
        \begin{equation}
            \ce{N2 + 3F2 -> 2NF3}
        \end{equation}
        a) Which element is the limiting reactant?\\
        b) What is the theoretical yield of \ce{NF3}?\\
        c) If the reaction goes to completion, how many grams of the excess reactant will remain?\\
        d) Set up an ICE table for this reaction.

        \subsection{Solution (a)}
            First, we calculate the theoretical yields for each for the reactants.
            \begin{align}
                m_{\ce{N2}} &=  10.00\,\unit{\gram} \times \frac{1\,\unit{\mole}\,\ce{N2}}{28.02\,\unit{\gram}\,\ce{N2}} 
                                                    \times \frac{2\,\ce{NF3}}{1\,\ce{N2}}
                                                    \times \frac{71.01\,\unit{\gram}\,\ce{NF3}}{1\,\unit{\mole}\,\ce{NF3}}
                    =   50.69\,\unit{\gram}\,\ce{NF3}
                \\
                m_{\ce{F2}} &=  33.61\,\unit{\gram} \times \frac{1\,\unit{\mole}\,\ce{F2}}{38.00\,\unit{\gram}\,\ce{F2}} 
                                                    \times \frac{2\,\ce{NF3}}{3\,\ce{F2}}
                                                    \times \frac{71.01\,\unit{\gram}\,\ce{NF3}}{1\,\unit{\mole}\,\ce{NF3}}
                    =   41.87\,\unit{\gram}\,\ce{NF3}
            \end{align}

            With a lower final mass, \boxed{\ce{F2}} is the limiting reactant.

        \subsection{Solution (b)}
            The theoretical yield was found in part (a).
            \boxed{41.87\,\unit{\gram}\,\ce{NF3}}
        
        \subsection{Solution (c)}
            Use a similar strategy to part (a).
            \begin{equation}
                33.61\,\unit{\gram} \times \frac{1\,\unit{\mole}\,\ce{F2}}{38.00\,\unit{\gram}\,\ce{F2}} 
                                    \times \frac{1\,\ce{N2}}{3\,\ce{F2}}
                                    \times \frac{28.02\,\unit{\gram}\,\ce{N2}}{1\,\unit{\mole}\,\ce{N2}}
                    =   8.261\,\unit{\gram}\,\ce{NF3}
            \end{equation}

            Subtract this from the available mass of \ce{N2} to get the final \ce{N2}.
            \begin{equation}
                10.00\,\unit{\gram}\,\ce{N2} - 8.261\,\unit{\gram}\,\ce{N2}    =   \boxed{1.74\,\unit{\gram}\,\ce{N2}}
            \end{equation}
        
        \subsection{Solution (d)}
            I used tabular for this table.
            Please excuse any poor or improper formatting.

            \begin{center}
                \begin{tabular}{| c | c | c | c |}
                    \hline
                    \unit{\mole} & \ce{N2} &\ce{+ 3F2} &\ce{-> 2NF3} \\
                    \hline 
                    I   & 0.3569    &0.8844     &0\\
                    \hline
                    C   & -0.2948   &-0.8844    &0.5896\\
                    \hline
                    E   & 0.0621    &0          &0.5896\\ \hline
                \end{tabular}
            \end{center}

            For those interested in how I went about getting these values, I can explain.
            I started with the initial mass of \ce{F2}, which has been previously established to be the limiting reactant, and converted that to moles.
            I did (roughly) the same thing for the known quantity of \ce{N2} initially.
            We also start with no \ce{NF3}.
            Assuming the percentage yield to be 100\%, every mole of \ce{F2} would be used, so the Change row for \ce{F2} would be the negative of the initial quantity of \ce{F2}.
            Multiply that by the ratio of \ce{N2} to \ce{F2} ($\frac{1}{3}$) to get the Change row of \ce{N2}.
            The same can be done for \ce{NF2}, just taking the negative thereof and with a ratio of $\frac{2}{3}$ instead of $\frac{1}{3}$.
            With all of this, we only have to add the initial and the change together (respecting the positive or negative signs) to get the values for the End row.

    \pagebreak
    \section{Topic A Problem 15}
        a) If 58.26 g of iodine reacts with excess aluminum, what is the theoretical yield of aluminum iodide? 
        The reaction is \ce{2 Al + 3 I2 -> 2 AlI3}.\\
        b) If 56.11 g of aluminum iodide is actually formed in the reaction in part a, what is the percent yield of aluminum iodide?

        \subsection{Solution (a)}
            Watch me use the power of Stiochiometry Magic.
            \begin{equation}
                58.26\,\unit{\gram} \times  \frac{1\,\unit{\mole}\,\ce{I2}}{253.8\,\unit{\gram}\,\unit{I2}}
                                    \times  \frac{2\,\ce{AlI3}}{3\,\ce{I2}}
                                    \times  \frac{407.68\,\unit{\gram}\,\ce{AlI3}}{1\,\unit{\mole}\,\ce{AlI3}}
                    =   \boxed{62.39\,\unit{\gram}\,\ce{AlI3}}
            \end{equation}
        
        \subsection{Solution (b)}
            Here we use the formula for the pecent yield.
            \begin{equation}
                \text{PY}   =   \frac{\rm AY}{\rm TY}   \times  100\%
                    =   \frac{56.11\,\unit{\gram}}{62.39\,\unit{\gram}} \times  100\%
                    =   0.8994 \times 100\%
                    =   \boxed{89.94\%}
            \end{equation}
        

    \pagebreak
    \section{Topic A Problem 16}
        A chemist mixes 16.00 g of \ce{HCl} with 10.00 g of \ce{Mg} and obtains an 81.3\% yield of \ce{MgCl2}.
        What mass of \ce{MgCl2} did the chemist obtain? 
        The chemical reaction is:
        \begin{equation}
            \ce{Mg + 2 HCl -> MgCl2 + H2}
        \end{equation}

        \subsection{Solution}
            First calculate the theoretical yield of \ce{MgCl2} in the cases of \ce{HCl} and \ce{Mg} being the limiting reactants.
            \begin{align}
                MM(\ce{MgCl2})  &=  24.31\,\unit{\gram/\mole} + 2 * 35.45\,\unit{\gram/\mole}
                    =   95.21\,\unit{\gram/\mole}\\
                MM(\ce{HCl})    &=  1.008\,\unit{\gram/\mole} + 35.45\,\unit{\gram/\mole}
                    =   36.458\,\unit{\gram/\mole}\\
                m_{\ce{Mg}}     &=  10.00\,\unit{\gram} \times  \frac{1\,\unit{\mole}\,\ce{Mg}}{24.31\,\unit{\gram}\,\ce{Mg}}
                                                        \times  \frac{1\,\ce{MgCl2}}{1\,\ce{Mg}}
                                                        \times  \frac{95.21\,\unit{\gram}\,\ce{MgCl2}}{1\,\unit{\mole}\,\ce{MgCl2}}\\
                                &=  39.16\,\unit{\gram}\,\ce{MgCl2}
                \\
                m_{\ce{HCl}}    &=  16.00\,\unit{\gram} \times  \frac{1\,\unit{\mole}\,\ce{HCl}}{36.458\,\unit{\gram}\,\ce{HCl}}
                                                        \times  \frac{1\,\ce{MgCl2}}{2\,\ce{HCl}}
                                                        \times  \frac{95.21\,\unit{\gram}\,\ce{MgCl2}}{1\,\unit{\mole}\,\ce{MgCl2}}\\
                                &=  20.89\,\unit{\gram}\,\ce{MgCl2}
            \end{align}

            The latter is lower, so the \ce{HCl} would be the limiting reactant and $20.89\,\unit{\gram}$ $\,\ce{MgCl2}$ would be the theoretical yield.
            Multiplying this by the (decimal version of) the percetage yield to get the actual yield.
            \begin{gather}
                20.89\,\unit{\gram}\,\ce{MgCl2} * 0.813 = \boxed{17.0\,\unit{\gram}\,\ce{MgCl2}}
            \end{gather}

    \pagebreak
    \section{Topic A Problem 17}
        How many milliliters of liquid \ce{Br2} (density = 3.1 g/mL) will react with 6.143 g of \ce{Cr}, if the product of this reaction is \ce{CrBr3}?

        \subsection{Solution}
            First write a chemical equation for this and balance it.
            \begin{center}
                \ce{3Br2 + 2Cr -> 2CrBr3}
            \end{center}

            The rest of the path is paved with the magic of Stoichiometry.
            \begin{equation}
                6.143\,\unit{\gram}\,\ce{Cr}    \times  \frac{1\,\unit{\mole}\,\ce{Cr}}{52.00\,\unit{\gram}\,\ce{Cr}}
                                                \times  \frac{3\,\ce{Br2}}{2\,\ce{Cr}}
                                                \times  \frac{159.8\,\unit{\gram}\,\ce{Br2}}{1\,\unit{\mole}\,\ce{Br2}}
                                                \times  \frac{1\,\unit{\milli\liter}}{3.1\,\unit{\gram}}
                    =   \boxed{9.1\,\unit{\milli\liter}\,\ce{Br2}}
            \end{equation}

    \pagebreak
    \section{Topic A Problem 18}
        Ethane (\ce{C2H6}) reacts with oxygen according to the following chemical equation:
        \begin{equation}
            \ce{2 C2H6 + 7 O2 -> 4 CO2 + 6 H2O}
        \end{equation}
        a) If you mix 5 moles of \ce{C2H6} with 13 moles of \ce{O2}, how many moles of each substance will you end up with, assuming the reaction goes to completion? 
        Include an ICE table in your answer.\\
        b) If you mix 81.43 g of \ce{C2H6} with 194.60 g of \ce{O2}, how many grams of each substance will you end up with, assuming the reaction goes to completion? 
        Include an ICE table in your answer. 
        (Note: your ICE table should be in terms of moles.)\\
        c) A chemist mixes 3.414 moles of \ce{O2} with an unknown number of moles of \ce{C2H6}. 
        The chemist obtains 1.657 moles of \ce{O2}. 
        How many moles of \ce{C2H6} must have been present originally, assuming the reaction went to completion? 
        Include an ICE table in your answer.

        \subsection{Solution (a)}
            \begin{center}
                \begin{tabular}{| c | c |c| c |c| c |c| c |}
                    \hline
                    \multicolumn{1}{|c|}{\unit{\mole}}  &   \multicolumn{1}{c}{\ce{2 C2H6}} &\multicolumn{1}{c}{+}& \multicolumn{1}{c}{\ce{7 O2}}   &\multicolumn{1}{c}{\ce{->}}&   \multicolumn{1}{c}{\ce{4 CO2}}  &\multicolumn{1}{c}{+}& \multicolumn{1}{c|}{\ce{6 H2O}}
                    \\  \hline
                    I   &   5               &&  13  &&  0               &&  0
                    \\  \hline
                    C   &   $-\frac{26}{7}$ &&  -13 &&  $\frac{52}{7}$  &&  $\frac{78}{7}$
                    \\  \hline
                    E   &   $\frac{9}{7}$   &&  0   &&  $\frac{52}{7}$  &&  $\frac{78}{7}$
                    \\  \hline
                \end{tabular}
            \end{center}

            You will end up with \boxed{1.286\,\unit{\mole}\,\ce{C2H6}}, \boxed{0\,\unit{\mole}\,\ce{O2}}, \boxed{7.429,\unit{\mole}\,\ce{CO2}}, and \boxed{11.143\,\unit{\mole}\,\ce{H2O}}. 

        \subsection{Solution (b)}
            \begin{center}
                \begin{tabular}{| c | c |c| c |c| c |c| c |}
                    \hline
                    \multicolumn{1}{|c|}{\unit{\mole}}  &   \multicolumn{1}{c}{\ce{2 C2H6}} &\multicolumn{1}{c}{+}& \multicolumn{1}{c}{\ce{7 O2}}   &\multicolumn{1}{c}{\ce{->}}&   \multicolumn{1}{c}{\ce{4 CO2}}  &\multicolumn{1}{c}{+}& \multicolumn{1}{c|}{\ce{6 H2O}}
                    \\  \hline
                    I   &   2.708   &&  6.081   &&  0       &&  0
                    \\  \hline
                    C   &   -1.737  &&  -6.081  &&  3.475   &&  5.212
                    \\  \hline
                    E   &   0.971   &&  0       &&  3.475   &&  5.212
                    \\  \hline
                \end{tabular}

                \textbf{Final table}
            \end{center}

            \textbf{Calculations}

            Convert grams to moles for oxygen and ethane.
            \begin{align}
                n(\ce{O2})  &=  \frac{m}{MM}
                    =   \frac{194.60\,\unit{\gram}\,\ce{O2}}{32.00\,\unit{\gram/\mole}}
                    =   6.08125\,\unit{\mole}\,\ce{O2}\\
                n(\ce{C2H6})    &=  \frac{m}{MM}
                    =   \frac{81.43\,\unit{\gram}\,\ce{C2H6}}{30.068\,\unit{\gram/\mole}}
                    =   2.708194759\,\unit{\mole}\,\ce{C2H6}
            \end{align}

            Next, we check which would result in the most product, for the coefficient $c$ of each reactant.
            \begin{align}
                ML  &=  \frac{n}{c}\\
                ML(\ce{O2})     &=  \frac{6.08125\,\unit{\mole}\,\ce{O2}}{7\,\ce{O2}}
                    =   0.86875\,\unit{\mole}\\
                ML(\ce{C2H6})   &=  \frac{2.708194759\,\unit{\mole}\,\ce{C2H6}}{2\,\ce{C2H6}}
                    =   1.357\,\unit{\mole}
            \end{align}

            This makes the \ce{O2} the limiting reactant.
            We can just fill out the ICE table's `C' (change) values from here by using ratios of coefficients of reactants and products.
            I won't go write all my calculations here, but they tend to have a simple formula.
            \begin{equation}
                n_2 =   n_1 * \frac{c_2}{c_1}
            \end{equation}

            This leads into the filling out of the bottom row (end), with a simple formula, E = I + C.
            From here, all we need to do is convert moles to grams.
            \begin{align}
                MM(\ce{C2H6})   &=  2 * 12.01\,\unit{\gram/\mole} + 6 * 1.008\,\unit{\gram/\mole}
                    =   30.068\,\unit{\gram/\mole}\\
                MM(\ce{O2})     &=  2 * 16.00\,\unit{\gram/\mole}
                    =   32.00\,\unit{\gram/\mole}\\
                MM(\ce{CO2})    &=  12.01 + 2 * 16.00\,\unit{\gram/\mole}
                    =   44.01\,\unit{\gram/\mole}\\
                MM(\ce{H2O})    &=  2 * 1.008\,\unit{\gram/\mole} + 16.00\,\unit{\gram/\mole}
                    =   18.016\,\unit{\gram/\mole}\\
                m   &=  MM * n\\
                m(\ce{C2H6})    &=  30.068\,\unit{\gram/\mole}\,\ce{C2H6} * 0.971\,\unit{\mole}
                    =   \boxed{29.20\,\unit{\gram}\,\ce{C2H6}}\\
                m(\ce{O2})      &=  32.00\,\unit{\gram/\mole}\,\ce{O2} * 0\,\unit{\mole}
                    =   \boxed{0\,\unit{\gram}\,\ce{O2}}\\
                m(\ce{CO2})     &=  44.01\,\unit{\gram/\mole}\,\ce{CO2} * 3.475\,\unit{\mole}
                    =   \boxed{152.9\,\unit{\gram}\,\ce{CO2}}\\
                m(\ce{H2O})     &=  18.016\,\unit{\gram/\mole}\,\ce{H2O} * 5.212\,\unit{\mole}
                    =   \boxed{93.90\,\unit{\gram}\,\ce{H2O}}
            \end{align}

        \subsection{Solution (c)}
            We'll just put together an ICE table and fill it out until we have enough information to get the answer.
            We don't even need to fill it out completely.
            \begin{center}
                \begin{tabular}{| c | c |c| c |c| c |c| c |}
                    \hline
                    \multicolumn{1}{|c|}{\unit{\mole}}  &   \multicolumn{1}{c}{\ce{2 C2H6}} &\multicolumn{1}{c}{+}& \multicolumn{1}{c}{\ce{7 O2}}   &\multicolumn{1}{c}{\ce{->}}&   \multicolumn{1}{c}{\ce{4 CO2}}  &\multicolumn{1}{c}{+}& \multicolumn{1}{c|}{\ce{6 H2O}}
                    \\  \hline
                    I   &   0.502   &&  3.414   &&  0       &&  0
                    \\  \hline
                    C   &   -0.502  &&  -1.757  &&          &&  
                    \\  \hline
                    E   &   0       &&  1.657   &&          &&  
                    \\  \hline
                \end{tabular}
            \end{center}

            I have a few steps I made for this.
            \begin{enumerate}
                \item   We can start by filling out our knowns. 
                In the interest of candor, we don't even know if the initial amount of \ce{CO2} and \ce{H2O} is 0, that's just an assumption I made.
                \item   Since the \ce{O2} has some left over at the end, it is definitely not the limiting reactant.
                That makes \ce{C2H6} the limiting reactant, so none of it should be left by the end and we can fill that out.
                \item   The formula of C = E - I can be applied for Oxygen, giving us our change in \ce{O2}.
                \begin{itemize}
                    \item   If you ever took Physics or Calculus (maybe even Algebra), you'll recognize this as equivalent to $\Delta x = x_f - x_i$. 
                \end{itemize}
                \item   We can use the coefficient ratio mentioned in part (b), this time $\frac{2}{7}$, for the change in \ce{C2H6}.
                \item   The formula C = E - I can be turned into I = E - C to get initial quantity of \ce{C2H6}.
            \end{enumerate}

            With everything filled out, we have the final answer of \boxed{0.502\,\unit{\mole}\,\ce{C2H6}}.

    \pagebreak
    \section{Topic A Problem 19}
        Ammonia reacts with oxygen according to the following chemical equation:
        \begin{equation}
            \ce{4 NH3 + 3 O2 -> 2 N2 + 6 H2O}
        \end{equation}
        Suppose you mix x moles of \ce{NH3} with y moles of \ce{O2}.\\
        a) If \ce{NH3} is the limiting reactant, how many moles of each substance will you end up with, assuming the reaction goes to completion? 
        Include an ICE table in your answer.\\
        b) If \ce{O2} is the limiting reactant, how many moles of each substance will you end up with, assuming the reaction goes to completion? 
        Include an ICE table in your answer.\\
        c) If you end up with 0.4y moles of \ce{O2}, what must the relationship be between x and y, assuming the reaction goes to completion?

        \subsection{Solution (a)}
            \begin{center}
                \begin{tabular}{| c | c |c| c |c| c |c| c |}
                    \hline
                    \multicolumn{1}{|c|}{\unit{\mole}}  &   \multicolumn{1}{c}{\ce{4 NH3}} &\multicolumn{1}{c}{+}& \multicolumn{1}{c}{\ce{3 O2}}   &\multicolumn{1}{c}{\ce{->}}&   \multicolumn{1}{c}{\ce{2 N2}}  &\multicolumn{1}{c}{+}& \multicolumn{1}{c|}{\ce{6 H2O}}
                    \\  \hline
                    I   &   x   &&  y                   &&  0               &&  0
                    \\  \hline
                    C   &   -x  &&  -$\frac{3}{4}$x     &&  $\frac{1}{2}$x  &&  $\frac{3}{2}$x
                    \\  \hline
                    E   &   0   &&  y - $\frac{3}{4}$x  &&  $\frac{1}{2}$x  &&  $\frac{3}{2}$x
                    \\  \hline
                \end{tabular}
            \end{center}

            My answers are found in the End (E) section of the ICE table.
            I did everything in my head.

        \subsection{Solution (b)}
            \begin{center}
                \begin{tabular}{| c | c |c| c |c| c |c| c |}
                    \hline
                    \multicolumn{1}{|c|}{\unit{\mole}}  &   \multicolumn{1}{c}{\ce{4 NH3}} &\multicolumn{1}{c}{+}& \multicolumn{1}{c}{\ce{3 O2}}   &\multicolumn{1}{c}{\ce{->}}&   \multicolumn{1}{c}{\ce{2 N2}}  &\multicolumn{1}{c}{+}& \multicolumn{1}{c|}{\ce{6 H2O}}
                    \\  \hline
                    I   &   x                   &&  y   &&  0               &&  0
                    \\  \hline
                    C   &   -$\frac{4}{3}$y     &&  -y  &&  $\frac{2}{3}$y  &&  2y
                    \\  \hline
                    E   &   x - $\frac{4}{3}$y  &&  0   &&  $\frac{2}{3}$y  &&  2y
                    \\  \hline
                \end{tabular}
            \end{center}

            My answers are found in the End (E) section of the ICE table.
            I did everything in my head.
        
        \subsection{Solution (c)}
            \begin{center}
                \begin{tabular}{| c | c |c| c |c| c |c| c |}
                    \hline
                    \multicolumn{1}{|c|}{\unit{\mole}}  &   \multicolumn{1}{c}{\ce{4 NH3}} &\multicolumn{1}{c}{+}& \multicolumn{1}{c}{\ce{3 O2}}   &\multicolumn{1}{c}{\ce{->}}&   \multicolumn{1}{c}{\ce{2 N2}}  &\multicolumn{1}{c}{+}& \multicolumn{1}{c|}{\ce{6 H2O}}
                    \\  \hline
                    I   &   x   &&  y     &&    &&  
                    \\  \hline
                    C   &   -x  &&  -0.6y &&    &&  
                    \\  \hline
                    E   &   0   &&  0.4y  &&    &&  
                    \\  \hline
                \end{tabular}
            \end{center}

            Out of respect for the unknowns, I have left the boxes unnecessary to calculate for completion of this problem blank, all of which lie in the products.
            \begin{enumerate}
                \item   The things I did initially fill out are the final amount of \ce{O2} and the initial amounts of \ce{O2} and \ce{NH3}.
                \item   The fact that there is \ce{O2} remaining suggests that the \ce{NH3} is the limiting reactant, so it would be 0 at the end.
                \item   E = C + I $\to$ C = E - I gives us our change values.
                        This in turn gives us an equation for their relationship, bearing in mind the ratio of their coefficients.
            \end{enumerate}
            \begin{gather}
                x * \frac{3}{4}    =   \frac{3}{5}y\\
                \boxed{x   =   \frac{4}{5}y    =   0.8y}
            \end{gather}

    \pagebreak
    \section{Topic A Problem 20}
        You have x grams of \ce{Na2Cr2O7}. 
        How many grams of \ce{CrCl3} will be formed if the \ce{Na2Cr2O7} undergoes the reaction below? 
        Express your answer in terms of x.
        \begin{equation}
            \ce{Na2Cr2O7 + 3 Zn + 14 HCl -> 2 CrCl3 + 3 ZnCl2 + 2 NaCl + 7 H2O}
        \end{equation}

        \subsection{Solution}
            First, we find the molar masses of each involved chemical that we need, which are only \ce{Na2Cr2O7} and \ce{CrCl3}.
            \begin{align}
                MM(\ce{Na2Cr2O7})   &=  2 * 22.99\,\unit{\gram/\mole} + 2 * 52.00\,\unit{\gram/\mole} + 7 * 16.00\,\unit{\gram/\mole}\\
                    &=  261.98\,\unit{\gram/\mole}\\
                MM(\ce{CrCl3})  &=  52.00\,\unit{\gram/\mole} + 3 * 35.45\,\unit{\gram/\mole}
                    =   158.35\,\unit{\gram/\mole}
            \end{align}

            From here, the magic of Stoichiometry will guide us.
            \begin{align}
                m(\ce{CrCl3})   &=  x   \times  \frac{1\,\unit{\mole}}{261.98\,\unit{\mole}}
                                        \times  \frac{2\,\ce{CrCl3}}{1\,\ce{Na2Cr2O7}}
                                        \times  \frac{158.35\,\unit{\gram}}{1\,\unit{\mole}}\\
                    &=  \boxed{1.209x\,\unit{\gram}\,\ce{CrCl3}}
            \end{align}

    \pagebreak
    \section{Topic A Problem 21}
        A metal sample weighing 1.410 g contains a mixture of copper and aluminum. 
        When excess \ce{HCl} is added to this sample, the aluminum reacts as follows:
        \begin{equation}
            \ce{2 Al + 6 HCl -> 2 AlCl3 + 3 H2}
        \end{equation}
        849 mL of \ce{H2} (density 0.08264 g/L) is produced. 
        Calculate the mass percentage of each element in the original sample. 
        Note that copper does not react with \ce{HCl}.

        \subsection{Solution}
            The way we start this is by finding the amount (mass) of \ce{Al} that reacts.
            This is done through the magic of stoichiometry.
            \begin{align}
                m(\ce{Al})  &=  0.849\,\unit{\liter}    \times  \frac{0.08264\,\unit{\gram}}{1\,\unit{\liter}}
                                                        \times  \frac{1\,\unit{\mole}}{2.016\,\unit{\gram}}
                                                        \times  \frac{2\,\ce{Al}}{3\,\ce{H2}}
                                                        \times  \frac{26.98\,\unit{\gram}}{1\,\unit{\mole}}\\
                    &=  0.626\,\unit{\gram}\,\ce{Al}
            \end{align}

            From here, we use the formula for mass percentage of $MP = \frac{m}{m_{\Sigma}}$ for each chemical's mass.
            The value of $m_{\Sigma}$ is the total mass, in this case $m_{1.410}$.
            \begin{align}
                MP(\ce{Al}) &=  \frac{0.626}{1.410} \times 100\%
                    =   \boxed{44.4\%\,\ce{Al}}\\
                MP(\ce{Cu}) &=  \left( 1 - \frac{0.626}{1.410} \right) \times 100\%
                    =   \boxed{55.6\%\,\ce{Cu}}
            \end{align}

    % \pagebreak
    % \section{Topic A Problem 22}
    %     A chemist has a mixture of \ce{AgNO3} and \ce{KNO3} that weighs a total of 4.177 g. 
    %     The chemist dissolves the mixture in water and then adds a solution of \ce{NaOH}. 
    %     The AgNO3 reacts with the \ce{NaOH} as follows:
    %     \begin{equation}
    %         \ce{2 AgNO3(aq) + 2 NaOH(aq) -> Ag2O(s) + 2 NaNO3(aq) + H2O(l)}
    %     \end{equation}
    %     The chemist finds that 1.080 grams of \ce{Ag2O} were formed. 
    %     Calculate the mass percentages of \ce{AgNO3} and \ce{KNO3} in the original mixture. 
    %     (Note that \ce{KNO3} does not react with \ce{NaOH}.)

    %     \subsection{Solution}

    % \pagebreak
    % \section{Topic A Problem 23}
    %     A 25.000 g sample of sulfur is burned. 
    %     Some of the sulfur reacts to form \ce{SO2}:
    %     \begin{equation}
    %         \ce{S + O2 -> SO2}
    %     \end{equation}
    %     The rest of the sulfur reacts to form \ce{SO3}:
    %     \begin{equation}
    %         \ce{2 S + 3 O2 -> 2 SO3}
    %     \end{equation}
    %     The total mass of products (\ce{SO2} and \ce{SO3}) is 58.723 g. 
    %     Calculate the masses of \ce{SO2} and \ce{SO3} in this mixture.

    %     \subsection{Solution}

    \pagebreak
    \section{Topic B Problem 1}
        Answer each of the following questions about making solutions.\\
        a) If you dissolve 4.18 g of solid \ce{Mg(NO3)2} in enough water to make 150 mL of solution, what will be the molarity of the resulting solution?\\
        b) If you need to make 100 mL of 1.08 M \ce{CaCl2}, what mass of solid CaCl2 will you need?\\
        c) You have 25.0 g of solid \ce{KCl}, and you use all of it to make a 0.500 M \ce{KCl} solution. 
        What volume of solution did you make?

        \subsection{Solution (a)}
            First convert grams to moles.
            \begin{align}
                MM(\ce{Mg(NO3)2})   &=  24.31 + 2 * 14.01 + 6 * 16.00
                    =   148.33\,\unit{\gram/\mole}\\
                n(\ce{Mg(NO3)2})    &=  \frac{4.18\,\unit{gram}\,\ce{Mg(NO3)2}}{148.33\,\unit{\gram/\mole}}
                    =   0.0282\,\unit{\mole}
            \end{align}

            From here, we just find the molarity. 
            \begin{equation}
                \left[ \ce{Mg(NO3)2} \right]   =   \frac{n}{V}
                    =   \frac{0.0282\,\unit{\mole}}{0.150\,\unit{\liter}}
                    =   \boxed{0.188\,\unit{\molarity}\,\ce{Mg(NO3)2}}
            \end{equation}

        \subsection{Solution (b)}
        First find the number of moles of \ce{CaCl2}.
            \begin{equation}
                n   =   V * \left[ \ce{CaCl2} \right]
                    =   0.100\,\unit{\liter} * 1.08\,\unit{\molarity}\,\ce{CaCl2}
                    =   0.108\,\unit{\mole}\,\ce{CaCl2}
            \end{equation}

            From here, use the molar mass to find the total mass.
            \begin{align}
                MM(\ce{CaCl2})  &=  40.08 + 2 * 35.45
                    =   110.98\,\unit{\gram/\mole}\\
                m   &=  n * MM
                    =   0.108\,\unit{\mole}\,\ce{CaCl2} * 110.98\,\unit{\gram/\mole}
                    =   \boxed{12.0\,\unit{\molarity}\,\ce{CaCl2}}
            \end{align}

        \subsection{Solution (c)}
            First convert grams to moles.
            \begin{align}
                MM(\ce{KCl})    &=  39.10 + 35.45
                    =   74.55\,\unit{\gram/\mole}\\
                n   &=  \frac{25.0\,\unit{\gram}\,\ce{KCl}}{74.55\,\unit{\gram/\mole}}
                    =   0.335\,\unit{\mole}\,\ce{KCl}
            \end{align}

            From here, we can find the volume.
            \begin{equation}
                V   =   \frac{n}{\left[ \ce{KCl} \right]}
                    =   \frac{0.335\,\unit{\mole}\,\ce{KCl}}{0.500\,\unit{\molarity}\,\ce{KCl}}
                    =   \boxed{0.671\,\unit{\liter}}
            \end{equation}

    \pagebreak
    \section{Topic B Problem 2}
        Answer the following questions about dilutions.\\
        a) If you add 100 mL of water to 10 mL of 0.605 M \ce{HCl}, what will be the molarity of the resulting solution?\\
        b) You have 200 mL of 1.50 M \ce{HNO3}. 
        If you wish to dilute this solution to a final concentration of 0.300 M, what volume of water should you add?\\
        c) You need to make 1.50 liters of 0.400 M \ce{NaOH} by diluting a 2.00 M \ce{NaOH} solution. 
        What volume of the 2.00 M \ce{NaOH} should you use, and what volume of water should you add to it?

        \subsection{Solution (a)}
            Use the relationship between concentration and volume.
            \begin{gather}
                C_1 V_1 =   C_2 V_2\\
                C_2 =   C_1 \frac{V_1}{V_2}
                    =   0.605 * \frac{100}{110}
                    =   \boxed{0.550\,\unit{\molarity}\,\ce{HCl}}
            \end{gather}

        \subsection{Solution (b)}
            First the relationship between concentration and volume.
            \begin{gather}
                C_1 V_1 =   C_2 V_2\\
                V_2 =   V_1 \frac{C_1}{C_2}
                    =   200 * \frac{1.50}{0.300}
                    =   200 * 5
                    =   1000\,\unit{\milli\liter}
            \end{gather}

            Now take the difference between final and initial volume to find the added volume.
            \begin{gather}
                \Delta V    =   V_2 - V_1
                    =   1000 - 200
                    =   \boxed{800\,\unit{\milli\liter}}
            \end{gather}

        \subsection{Solution (c)}
            Here we can use the same relationship between concentrations and volumes.
            We can start with making a ratio for the volume that would turn 1 liter of 2.00 M \ce{NaOH} to 0.400 M \ce{NaOH}. 
            \begin{gather}
                C_1 V_1 =   C_2 V_2\\
                V_2 =   V_1 \frac{C_1}{C_2}
                    =   1 * \frac{2.0}{0.4}
                    =   5\,\unit{\liter}
            \end{gather}

            This means that every liter of 2.00 M \ce{NaOH} would require 4 liters of water to form 0.400 M \ce{NaOH}.
            For each chemical (pure water and 2.00 M \ce{NaOH}), we can multiply the final volume by its necessary ratio ($\frac{4}{5}$ and $\frac{1}{5}$, respectively).
            \begin{align}
                V_{\rm water}       &=  1.5\,\unit{\liter} * \frac{4}{5}
                    =   1.2\,\unit{\liter}\\
                V_{\rm \ce{NaOH}}   &=  1.5\,\unit{\liter} * \frac{1}{5}
                    =   0.3\,\unit{\liter}
            \end{align}

            For the water, we need \boxed{1.2\,\unit{\liter}}.
            For the \ce{NaOH}, we need \boxed{0.3\,\unit{\liter}}.

    \pagebreak
    \section{Topic B Problem 3}
        All of the compounds below dissolve in water. 
        Which of them are strong electrolytes, which are weak electrolytes, and which are nonelectrolytes?
        \begin{multicols}{4}
            \begin{enumerate}[label=\alph*)]
                \item   \ce{NaCl}
                \item   \ce{Mg(NO3)2}
                \item   \ce{HClO2}
                \item   \ce{MgCrO4}
                \item   \ce{H3PO4}
                \item   \ce{AgF}
                \item   \ce{C2H5OH}
                \item   \ce{HC3H5O3}
                \item   \ce{CH3CN}
                \item   \ce{H2SO4}
                \item   \ce{NH4Br}
                \item   \ce{(CH3)2CO}
            \end{enumerate}
        \end{multicols}

        \subsection{Solution}
            \begin{multicols}{3}
                \begin{enumerate}[label=\alph*)]
                    \item   Strong electrolyte
                    \item   Strong electrolyte
                    \item   Weak electrolyte
                    \item   Strong electrolyte
                    \item   Weak electrolyte
                    \item   Strong electrolyte
                    \item   Non-electrolyte
                    \item   Weak electrolyte
                    \item   Non-electrolyte
                    \item   Strong electrolyte
                    \item   Strong electrolyte
                    \item   Non-electrolyte
                \end{enumerate}
            \end{multicols}

            I \textit{really} need to work on this.

    \pagebreak
    \section{Topic B Problem 4}
        What ions (if any) are present in each of the following solutions, and what is the molar concentration of each ion?
        
        \begin{multicols}{2}
            \begin{enumerate}[label=\alph*)]
                \item   0.1 M \ce{NaBr}
                \item   0.04 M \ce{KNO3}
                \item   0.2 M \ce{FeCl3}
                \item   1.5 M \ce{(NH4)2SO4}
            \end{enumerate}
        \end{multicols}

        \subsection{Solution}
            \begin{enumerate}[label=\alph*/]
                \item   0.1 M \ce{Na+}, 0.1 M \ce{Br-}
                \item   0.04 M \ce{K+}, 0.04 M \ce{NO3-}
                \item   0.2 M \ce{Fe^3+}, 0.6 M \ce{Cl-}
                \item   3.0 M \ce{NH4+}, 1.5 M \ce{SO4^2-}
            \end{enumerate}

    \pagebreak
    \section{Topic B Problem 5}
        How many moles of each ion are present in 175 mL of 0.147 M \ce{Fe2(SO4)3}?

        \subsection{Solution}
            Start with the iron (III) \ce{Fe^3+}.
            There are 2 \ce{Fe^3+} ions per \ce{Fe2(SO4)3} atom, so we can find the molarity of the iron (III) atom.
            By multiplying that by the volume, we will find the number of moles of iron ions.
            \begin{equation}
                n   =   2\,\ce{\frac{F^3+}{Fe2(SO4)3}} * 0.147\,\unit{\M}\,\ce{Fe2(SO4)3} * 0.175\,\unit{liter}
                    =   \boxed{0.0515\,\unit{\mole}\,\ce{Fe^3+}}
            \end{equation}

            Following that, there are three \ce{SO4^2-} ions per \ce{Fe2(SO4)3} atom, so we can find the molarity of the sulfate.
            Multiply the molarity by the volume to get the number of moles.
            \begin{equation}
                n   =   3\,\ce{\frac{SO4^2-}{Fe2(SO4)3}} * 0.147\,\unit{\M}\,\ce{Fe2(SO4)3} * 0.175\,\unit{liter}
                    =   \boxed{0.0772\,\unit{\mole}\,\ce{SO4^2-}}
            \end{equation}

    \pagebreak
    \section{Topic B Problem 6}
        Which of the following are acceptable ways to make one liter of 1 M \ce{NaCl}?\\
        a) Put 1 liter of water into a container, then add 1 mole of NaCl and stir until the \ce{NaCl} dissolves.\\
        b) Put 1 mole of \ce{NaCl} into a container, then add 1 liter of water and stir until the \ce{NaCl} dissolves.\\
        c) Put 1 mole of \ce{NaCl} into a container, then add water with stirring until the total volume reaches 1 liter.

        \subsection{Solution}
            The one mole of NaCl will inevitably have some volume, which will not be lost when it dissolves.
            Both (a) and (b) result in a solution whose volume is the sum of the volume of the \ce{NaCl} and the water it is in, with exactly one mole of \ce{NaCl}.
            The molarity of the result will be $\frac{1\,\unit{\mole}}{1 + x\,\unit{\liter}}\,\ce{NaCl}$, for some volume of the \ce{NaCl} $x$.
            It doesn't matter if the resultant molarity is greater or smaller than $\frac{1\,\unit{\mole}}{1\,\unit{\liter}}\,\ce{NaCl}$, but it is undeniable that they are not equal, leaving the two strategies as unusable.
            Strategy \boxed{(c)} works, though. You measure the moles from the start, and you know exactly how much water there is.

    \pagebreak
    \section{Topic B Problem 7}
        Janet dissolves 6.50 g of solid potassium phosphate in enough water to make 100.0 mL of solution. 
        Farid then adds enough water to the solution to reduce the concentration of potassium ions to 0.250 M. 
        How much water did Farid add?

        \subsection{Solution}
            What's in a name? 
            The name ``Potassium Phosphate'' suggests that the chemical has the formula \ce{K3PO4}.
            We can convert grams to moles and use that to find the concentration, then use $C_1 V_1 = C_2 V_2$ to find the final amount of water, then subtract the initial volume to find the volume added.
            \begin{gather}
                MM(\ce{K3PO4})  =   3 * 39.10 + 30.97 + 4 * 16.00
                    =   212.27\,\unit{\gram/\mole}\\
                \left[ \ce{K3PO4} \right]   =   6.50\,\unit{\gram}  \times  \frac{1\,\unit{\mole}}{212.27\,\unit{\gram}}
                                                                    \times  \frac{1}{0.100\,\unit{\liter}}  \,\ce{K3PO4}
                    =   0.306\,\unit{\mole/\liter}\\
                C_1 V_1 =   C_2 V_2\\
                V_2 =   V_1 \frac{C_1}{C_2}
                    =   0.1 \times \frac{0.306}{0.250}
                    =   0.12249\,\unit{\liter}\\
                \Delta V    =   V_2 - V_1
                    =   0.122495 - 0.100
                    =   \xcancel{\boxed{0.022\,\unit{\liter}}}
            \end{gather}

    \pagebreak
    \section{Topic B Problem 8}
        Gerardo dissolves 8.213 g of solid \ce{Mg(NO3)2} in enough water to make 200.0 mL of solution.
        Marciela then adds enough solid \ce{Al(NO3)3} to increase the concentration of nitrate ions to 0.900 M. 
        Assuming that the solution volume does not change significantly, what mass of \ce{Al(NO3)3} did Marciela add?

        \subsection{Solution}
            First calculate the molarity of \ce{NO3}.
            \begin{align}
                MM(\ce{Mg(NO3)2})    &=  24.31 + 2 * 14.01 + 6 * 16.00\\
                    &=  148.33\,\unit{\gram/\mole}\,\ce{Mg(NO3)2}\\
                MM(\ce{Al(NO3)2})    &=  26.98 + 3 * 14.01 + 9 * 16.00\\
                    &=  213.01\,\unit{\gram/\mole}\,\ce{Al(NO3)2}\\
                \left[ \ce{Mg(NO3)2} \right]    &=  8.213   \times  \frac{1\,\unit{\mole}}{148.33\,\unit{\gram}}
                                                            \times  \frac{1}{0.2000\,\unit{\liter}}\\
                    &=  0.2768\,\unit{\M}\,\ce{Mg(NO3)2}\\
                \left[ \ce{NO3} \right]_i   &=  2 * 0.2768\,\unit{\M}
                    =   0.5537\,\unit{\M}\,\ce{NO3}
            \end{align}

            This means that the added molarity is $0.3463\,\unit{\M}\,\ce{NO3}$, which makes the molarity of \ce{Al(NO3)3} to be $0.3463\,\unit{\M}\,\ce{NO3} \times \frac{1}{3} = 0.1154\,\unit{\M}\,\ce{Al(NO3)3}$.
            Multiply this by the volume to get the number of moles of \ce{Al(NO3)3} added.
            This can be used with the molar mass to find the mass of \ce{Al(NO3)3} used.
            \begin{align}
                m(\ce{Al(NO3)3})    &=  MM(\ce{Al(NO3)3}) * \left[ \ce{Al(NO3)3} \right] * V\\
                    &=  213.01\,\unit{\gram/\mole}\,\ce{Al(NO3)2} \times 0.1154\,\unit{\M}\,\ce{Al(NO3)3} \times 0.2000\,\unit{\liter}\\
                    &=  4.918\,\unit{\gram}
            \end{align}

            This means that Marciela added \boxed{4.918\,\unit{\gram}}.

    \pagebreak
    \section{Topic B Problem 9}
        Chantelle dissolves 2.35 g of \ce{NaCl}, 3.12 g of \ce{CaCl2}, and 1.88 g of \ce{FeCl3} in enough water to make 175 mL of solution. 
        What is the molarity of chloride ions in this solution?

        \subsection{Solution}
            Find the number of moles of \ce{Cl} from each chemical and sum them together.
            \begin{align}
                MM(\ce{NaCl})   &=  22.99 + 35.45   =   58.44\,\unit{\gram/\mole}
                \\
                MM(\ce{CaCl2})  &=  40.08 + 2 * 35.45   =   110.98\,\unit{\gram/\mole}
                \\
                MM(\ce{FeCl3})  &=  55.85 + 3 * 35.45   =   162.2\,\unit{\gram/\mole}
                \\
                n(\ce{Cl-}) &=  \sum \frac{c_i m_i}{MM}
                    =   \frac{2.35}{58.44} + \frac{2 * 3.12}{110.98} + \frac{3 * 1.88}{162.2}\\
                    &=  0.0402 + 0.0562 + 0.0348
                    =   0.1312\,\unit{mole}\,\ce{Cl-}
            \end{align}

            This can be divided by the volume to get the molarity.
            \begin{align}
                \left[ \ce{Cl-} \right] &=  \frac{n}{V}
                    =   \frac{0.1312\,\unit{mole}\,\ce{Cl-}}{0.175\,\unit{\liter}}
                    =   \boxed{0.750\,\unit{\molarity}\,\ce{Cl-}}
            \end{align}

    \pagebreak
    \section{Topic B Problem 10}
        Wenzhou prepares 200 mL of a solution of \ce{SnCl4} in which the concentration of chloride ions is 0.240 M.\\
        a) What is the molarity of the \ce{SnCl4} solution (i.e. what should the bottle be labeled)?\\
        b) What mass of \ce{SnCl4} did Wenzhou use?

        \subsection{Solution}

    % \pagebreak
    % \section{Topic B Problem 11}
    %     A beaker holds x liters of 0.2 M \ce{AlBr3}. 
    %     Give answers to each part below in terms of x.\\
    %     a) How many moles of aluminum ions are in this solution?\\
    %     b) How many moles of bromide ions are in this solution?\\
    %     c) How much water must you add if want to dilute the original solution to a concentration of 0.02 M?

    %     \subsection{Solution}

    % \pagebreak
    % \section{Topic B Problem 12}
    %     Using the solubility rules, determine which of the following compounds are insoluble in water. 
    %     There is a solubility rules handout available in Canvas.
    %     \begin{multicols}{3}
    %         \begin{enumerate}[label=\alph*)]
    %             \item   \ce{K2Cr2O7} 
    %             \item   \ce{Mn(NO3)2}
    %             \item   \ce{FeS}
    %             \item   \ce{ZnBr2}
    %             \item   \ce{MgSO4}
    %             \item   \ce{NaHCO3}
    %             \item   \ce{Ba3(PO4)2}
    %         \end{enumerate}
    %     \end{multicols}

    %     \subsection{Solution}

    \pagebreak
    \tableofcontents

\end{document}