\documentclass[10pt]{article}
\usepackage[export]{adjustbox}
\usepackage{amsmath}
\usepackage[makeroom]{cancel}
\usepackage{enumitem}
\usepackage{graphicx}
%Load mhchem using some package options
\usepackage[version=4]{mhchem}
\usepackage{multicol}
\usepackage{siunitx}

\title{
    Problem Set \#6
    \\  \small
    CHEM101A: General College Chemistry
    }
\author{Donald Aingworth IV}
\date{September 26, 2025}

\newcommand{\E}[1]{\times 10^{#1}}
\newcommand{\U}[1]{\underline{#1}}

\begin{document}
    \DeclareSIUnit{\atm}{atm}
    \DeclareSIUnit{\molarity}{M}
    \DeclareSIUnit{\M}{M}

    \maketitle

    % \setcounter{section}{9}

    \pagebreak
    \section{Topic D Problem 1}
        What is the sign of q for each of the following processes, assuming that the water is the system? 
        You do not need to calculate the numerical values of q.
        \begin{enumerate}[label=\alph*)]
            \item 50 g of ice melts
            \item 50 L of steam condenses
            \item 50 mL of water cools down from 20\unit{\celsius} to 10\unit{\celsius}
        \end{enumerate}

        \subsection{Solution}
            \begin{enumerate}[label=\alph*)]
                \item Positive. 
                \item Negative. 
                \item Negative.
            \end{enumerate}

    \pagebreak
    \section{Topic D Problem 2}
        A student puts a hot rock into some cold water and waits until the temperature of the rock and the water are no longer changing.
        \begin{enumerate}[label=\alph*)]
            \item The student writes the following equation: final $\rm T_{rock} =$ final $\rm T_{water}$. Is this a true statement? If not, how should it be modified to make it true?
            \item The student then writes the following equation: $\rm q_{rock} = \rm q_{water}$. Is this a true statement? If not, how should it be modified to make it true? \textit{(Assume that the rock and water are in an insulated container, so no energy can enter or leave.)}
        \end{enumerate}

        \subsection{Solution (a)}
            The statement is \boxed{true}. 
            A temperature difference would require subsequent heat exchange to reach thermal equilibrium.

        \subsection{Solution (b)}
            The statement is false. 
            A correct statement would have either value but not both as negative.
            \begin{gather}
                \rm 0 = \Delta E = q_{rock} + q_{water} \to \boxed{\rm q_{rock} = -q_{water}}
            \end{gather}

    \pagebreak
    \section{Topic D Problem 3}
        Use the following properties of bromine (\ce{Br2}) to answer parts a through g below.\\
            Specific heat capacity of \ce{Br2}(l) = 0.473 \unit{\joule/\gram\cdot\celsius}\\
            Heat of fusion of \ce{Br2} = 33.1 J/g\\
            Heat of vaporization of \ce{Br2} = 96.6 J/g\\
            Melting point of \ce{Br2} = -7.2\unit{\celsius}\\
            Boiling point of \ce{Br2} = 58.9\unit{\celsius}
        
        \begin{enumerate}[label=\alph*)]
            \item How much heat is needed to raise the temperature of 40.0 g of liquid bromine from -5.8\unit{\celsius} to 11.9\unit{\celsius}?
            \item How much heat is needed to boil 40.0 g of liquid bromine?
            \item How much heat is needed to melt 40.0 g of solid bromine?
            \item A 25.0 g sample of liquid bromine is at 10.0\unit{\celsius}. If you add 125 J of heat to the bromine, what will its final temperature be?
            \item A 25.0 g sample of liquid bromine is at its boiling point. If you add 1.500 kJ of heat to the bromine, how much of it will boil?
            \item How much heat is needed to convert 10.0 g of solid bromine at $-7.2\unit{\celsius}$ into gaseous bromine at $58.9\unit{\celsius}$?
        \end{enumerate}

        \subsection{Solution (a)}
            This relies on a single use of heat.
            \begin{align}
                Q   &=  cm\,\Delta T
                    =   0.473 * 40.0 * (11.9 - (-5.8))\\
                    &=  18.92 * 17.7
                    =   334.884\,\unit{\joule}
                    \approx \boxed{335\,\unit{\joule}}
            \end{align}

        \subsection{Solution (b)}
            Assuming we start the boiling point, it only relies on one use of heat.
            \begin{align}
                Q   &=  m\,\Delta H
                    =   40.0 * 96.6
                    =   3864\,\unit{\joule}
                    =   \boxed{3860\,\unit{\joule}}
            \end{align}

        \subsection{Solution (c)}
            Same katastasy (situation) as part (b).
            \begin{align}
                Q   &=  m\,\Delta H
                    =   40.0 * 33.1
                    =   1324\,\unit{\joule}
                    =   \boxed{1320\,\unit{\joule}}
            \end{align}

        \subsection{Solution (d)}
            We can apply the formula for change in temperature from the heat.
            \begin{gather}
                Q   =   cm\,\Delta T\\
                \Delta T    =   T_f - T_i
                    =   \frac{Q}{cm}\\
                \begin{align}
                    T_f &=  T_i + \frac{Q}{cm}
                        =   10.0\unit{\celsius} + \frac{125\,\unit{\joule}}{0.473\,\unit{\joule/\gram\cdot\celsius} \times 25.0\,\unit{\gram}}\\
                        &=  10.0\unit{\celsius} + \frac{5.00\,\unit{\joule}}{0.473\,\unit{\joule/\celsius}}
                        =   10.0\unit{\celsius} + \underline{10.5}708\unit{\celsius}\\
                        &=  \U{20.5}708\unit{\celsius}
                        =   \boxed{20.7\unit{\celsius}}
                \end{align}
            \end{gather}

        \subsection{Solution (e)}
            We have an equation for phase change to apply.
            \begin{gather}
                Q   =   m\,\Delta H\\
                \begin{align}
                    m   &=  \frac{Q}{\Delta H}
                        =   \frac{1500\,\unit{\joule}}{96.6\,\unit{\joule/\gram}}
                        =   \U{15.5}2795\,\unit{\gram}
                        =   \boxed{15.5\,\unit{\gram}}
                \end{align}
            \end{gather}

        \subsection{Solution (f)}
            There are three parts to this: melting, heating, and evaporating.
            We can divide the total heat necessary into these three parts.
            \begin{align}
                Q   &=  Q_{\rm melt} + Q_{\rm heat} + Q_{\rm evap}\\
                    &=  m\,\Delta H_{\rm melt} + cm\,\Delta T + m\,\Delta H_{\rm evap}\\
                    &=  m\,\Delta H_{\rm melt} + cm\,(58.9\unit{\celsius} - (-7.2\unit{\celsius})) + m\,\Delta H_{\rm evap}\\
                    &=  10.0\,\unit{\gram} * 33.1\,\unit{\joule/\gram}
                        +   0.473\,\unit{\joule/\gram\cdot\celsius} * 10.0\,\unit{\gram} * 66.1\unit{\celsius}
                        +   10.0\,\unit{\gram} * 96.6\,\unit{\joule/\gram}\\
                    &=  331\,\unit{\joule} + \U{312}.653\,\unit{\joule} + 966\,\unit{\joule}
                    =   \U{1609}.653\,\unit{\joule}
                    =   \boxed{1610\,\unit{\joule}}
            \end{align}

    \pagebreak
    \section{Topic D Problem 4}
        A 34.82 g piece of beryllium at 61.9\unit{\celsius} is placed into 41.33 g of water at 16.1\unit{\celsius}. 
        What will the temperature be after the water and the beryllium have reached thermal equilibrium, assuming that no heat is lost to the surroundings? 
        The specific heat capacity of beryllium is 1.82 \unit{\joule/\gram\cdot\celsius}, and that of water is 4.18 \unit{\joule/\gram\cdot\celsius}.

        \subsection{Solution}
            Let's just brute force this.
            \begin{gather}
                \begin{align}
                    0   &=  Q_{\rm Be} + Q_{\rm H_2 O}
                        =   c_{\rm Be}m_{\rm Be}\,\Delta T_{\rm Be} + c_{\rm H_2 O}m_{\rm H_2 O}\,\Delta T_{\rm H_2 O}\\
                        &=  1.82 * 34.82 * (T_f - 61.9) + 4.18 * 41.33 * (T_f - 16.1)
                \end{align}\\
                T_f (1.82 * 34.82 + 4.18 * 41.33)   =   (1.82 * 34.82 * 61.9 + 4.18 * 41.33 * 16.1)\\
                \begin{align}
                    T_f &=  \frac{1.82 * 34.82 * 61.9 + 4.18 * 41.33 * 16.1}{1.82 * 34.82 + 4.18 * 41.33}\\
                        &=  \frac{\U{392}2.752 + \U{278}1.426}{\U{63.3}724 + \U{172}.759}
                        =   \frac{\U{670}4.1779}{\U{236.1}318}\\
                        &=  \U{28.3}9168\unit{\celsius}
                        =   \boxed{28.4\unit{\celsius}}
                \end{align}
            \end{gather}

    \pagebreak
    \section{Topic D Problem 5}
        A chemist puts an unknown mass of aluminum at 68.0\unit{\celsius} and 67.3 g of chromium at 75.0\unit{\celsius} into 125.0 g of water at 8.2\unit{\celsius}. 
        When the mixture reaches thermal equilibrium, the temperature is 16.4\unit{\celsius}. 
        Using this information, and assuming that no heat is lost to the surroundings, calculate the mass of the piece of aluminum. 
        The relevant specific heat capacities are: aluminum = 0.902 \unit{\joule/\gram\cdot\celsius}, chromium = 0.448 \unit{\joule/\gram\cdot\celsius}, water = 4.18 \unit{\joule/\gram\cdot\celsius}.

        \subsection{Solution}
            We have three substances, each of which absorb or emit heat energy.
            For two of them, we can calculate the exact value of their heat traded.
            For the other, we cna find its heat traded with regard to the mass.
            \begin{align}
                Q   &=  cm\,\Delta T
                    =   cm(T_f - T_i)\\
                Q_{\rm Cr}  &=  0.448 * 67.3 * (16.4 - 75.0)
                    =   -\U{176}6.81\,\unit{\joule}\\
                Q_{\rm H_2O}    &=  4.18 * 125.0 * (16.4 - 8.2)
                    =   \U{42}84.5\,\unit{\joule}\\
                Q_{\rm Al}  &=  0.902 * m * (16.4 - 68.0)
                    =   -\U{46.5}432\,m\,\unit{\joule}
            \end{align}

            These three, added together should add up to zero, since no heat comes in from the outside.
            This can be put together as a sum.
            \begin{gather}
                0   =   Q_{\rm Cr} + Q_{\rm H_2O} + Q_{\rm Al}
                    =   -\U{176}6.81\,\unit{\joule} + \U{42}84.5\,\unit{\joule} - \U{46.5}432\,m\,\unit{\joule}\\
                \U{46.5}432\,m  =   \U{251}7.687\\
                m   =   \frac{\U{251}7.687}{\U{46.5}432}
                    =   \U{54.0}94\,\unit{\gram}
                    =   \boxed{54.1\,\unit{\gram}}
            \end{gather}

    \pagebreak
    \section{Topic D Problem 6}
        A 46.51 g piece of copper at 71.2\unit{\celsius} is put into 51.32 g of liquid ethanol (\ce{C2H5OH}) at 4.9\unit{\celsius}.
        When the mixture reaches thermal equilibrium, the temperature is 13.2\unit{\celsius}. 
        The specific heat capacity of copper is 0.385 \unit{\joule/\gram\cdot\celsius}. 
        Calculate the specific heat capacity of liquid ethanol.

        \subsection{Solution}
            This is algebra.
            \begin{gather}
                \begin{align}
                    Q   &=  cm\,\Delta T
                        =   cm(T_f - T_i)\\
                    Q_{\rm Cu}  &=  0.385 * 46.51 * (13.2 - 71.2)
                        =   -\U{103}8.5683\,\unit{\joule}\\
                    Q_{\rm ethanol} &=  c * 51.32 * (13.2 - 4.9)
                        =   c * \U{42}5.956\\
                    0   &=  Q_{\rm Cu} + Q_{\rm ethanol}
                        =   -\U{103}8.5683 + c * \U{42}5.956
                \end{align}\\
                c * \U{42}5.956 =   \U{103}8.5683\\
                c   =   \frac{\U{103}8.5683}{\U{42}5.956}
                    =   \U{2.43}82\,\unit{\frac{\joule}{\gram\cdot\celsius}}
                    =   \boxed{2.44\,\unit{\frac{\joule}{\gram\cdot\celsius}}}
            \end{gather}

    \pagebreak
    \section{Topic D Problem 7}
        The melting point of gallium (Ga) is 29.8\unit{\celsius}; this is also the temperature at which liquid gallium freezes. 
        A chemist pours 54.72 g of liquid gallium at 29.8\unit{\celsius} into 67.09 g of water at 5.2\unit{\celsius}. 
        When the mixture reaches thermal equilibrium, the gallium has solidified and the temperature is 21.4\unit{\celsius}. 
        The specific heat of gallium is 0.371 \unit{\joule/\gram\cdot\celsius} and the specific heat of water is 4.18 \unit{\joule/\gram\cdot\celsius}. 
        Use this information to calculate the heat of fusion of gallium. 
        Give your answer in J/g and in kJ/mol.

        \subsection{Solution}
            The heat absorbed by the gallium should be equal to the heat released by the water.
            The heat absorbed by the gallium can be divided into the heat used to melt and the hest used to cool down after melting.
            \begin{gather}
                \begin{align}
                    Q_{\rm H_2O}    &=  cm\,\Delta T
                        =   cm(T_f - T_i)\\
                        &=  4.18 * 67.09 * (21.4 - 5.2)
                        =   \U{454}3.066\,\unit{\joule}\\
                    Q_{\rm Ga}  &=  cm\,\Delta T + m\,\Delta H\\
                        &=  0.371 * 54.72 * (21.4 - 29.8) + 54.72 * \Delta H\\
                        &=  -\U{17}0.5294\,\unit{\joule} + 54.72 * \Delta H
                \end{align}
            \end{gather}
    
    \pagebreak
    \tableofcontents
\end{document}