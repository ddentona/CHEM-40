\documentclass[11pt]{article}
\usepackage{amsmath}
\usepackage{cancel}
\usepackage{mhchem}
\usepackage{siunitx}
\usepackage{titlesec}

\title{Gaussian Matrix Method for Hess' Law}
\author{Donald Aingworth IV}
\date{}

\begin{document}

    \maketitle

    \begin{abstract}
        In Linear Algebra, Gauss' Matrix Method refers to a manipulation of systems of equations through one or more of three operations to find the values of specific combinations of variables.
        In Chemistry, Hess' Law refers to how the magnitude of a path-independent process is equivalent to the sum of multiple operations that put together go from the starting conditions to the ending conditions.
        Pairing each column of a matrix with a molecule/ion (or the enthalpy) and each row with an equation, then setting the intersection of each row and column to the number of moles of that molecule/ion produced allows us to use the Gaussian Matrix Method to solve for things like the enthalpy of a system of chemical equations.
    \end{abstract}

    \section{Introduction}
        In Linear Algebra, Gauss' Matrix Method refers to a manipulation of systems of equations through one or more of three operations to find the values of specific combinations of variables.
        There are a total of three operations that can be used: (1) swapping two equations, (2) multiplying an equation entirely by a non-zero scalar, and (3) replacing one equation with the sum of it and another equation.
        These systems of equations can in turn be turned into a matrix, as shown below.
        \begin{equation}
            \left.
            \begin{matrix}
                ax + by + cz = l\\
                dx + ey + fz = m\\
                gx + hy + kz = n
            \end{matrix}
            \right\} \rightarrow
            \left[ \begin{array}{c c c | c}
                a & b & c & l\\
                d & e & f & m\\
                g & h & k & n
            \end{array} \right]
        \end{equation}

        In Chemistry, Hess' Law refers to how the magnitude of a path independent process is equivalent to the sum of any number of its parts that added together would go from the same start to the same end.
        This tends to refer to working with enthalpy of reactions, denoted $\rm \Delta H$. 
        In these reactions, each molecule or ion is paired with a coefficient, which denotes the (relative) number of moles of that molecule or ion for said reaction to take place.
        Below is an example of one such reaction.
        \begin{equation}
            \ce{2 C(s) + O2(g) -> 2 CO(g), \Delta H = -221 \unit{\kilo\joule}}
        \end{equation}
        
    \section{Proof of Likeness}
        If any set of equations can respect the three forms of system manipulation of matrices, it can be represented as a matrix.
        Reaction equations are themselves equations, so we can test the three operations.
        
        First is swapping.
        Suppose you have a system of two (or more) chemical reaction equations for chemicals A, B, C, D, and E.
        \begin{align}
            \ce{n_1A + n_2B} & \ce{-> n_3C} & \ce{\Delta H = h_1}\\
            \ce{n_4C} & \ce{-> n_5D + n_6E} & \ce{\Delta H = h_2}
        \end{align}
        We can swap these two equations. The result is shown directly below.
        \begin{align}
            \ce{n_4C} & \ce{-> n_5D + n_6E} & \ce{\Delta H = h_2}\\
            \ce{n_1A + n_2B} & \ce{-> n_3C} & \ce{\Delta H = h_1}
        \end{align}
        Neither equation is changed.
        The order is changed, but since the equations are independent, they are not impacted at all.
        
        Second is multiplication by a non-zero scalar.
        This is simple and is possible for positive numbers.
        Take the below chemical reaction equation.
        \begin{align}
            \ce{n_1A + n_2B} & \ce{-> n_3C} & \ce{\Delta H = h_1}
        \end{align}
        Suppose we were to multiply all parts of it by a scalar $s$.
        \begin{align}
            \ce{s n_1A + s n_2B} & \ce{-> s n_3C} & \ce{\Delta H = s h_1}
        \end{align}
        Since these coefficients can be treated as ratios per equation, this works.
        To deal with the question of negatives, we can treat these coefficients as changes in the number of molecules/ions during a reation, with the coefficients of reactants being negative due to losing them in the reaction and the coefficients of products being positive due to gaining them in the reaction.
        We can perform said operation on the same example as before, multiplying it by the nonzero negative scalar $-s$ (the negative of $s$).
        \begin{align}
            \ce{n_1A + n_2B} & \ce{-> n_3C} & \ce{\Delta H = h_1}\\
            \ce{s n_3C} & \ce{-> s n_1A + s n_2B} & \ce{\Delta H = -s h_1}
        \end{align}

        Last is the additive property.
        This is given from Hess' Law.
        Take the following two equations.
        \begin{align}
            \ce{n_1A + n_2B} & \ce{-> n_3C} & \ce{\Delta H = h_1}\\
            \ce{n_3C} & \ce{-> n_4D + n_5E} & \ce{\Delta H = h_2}
        \end{align}
        We can add them together to form a third equation.
        \begin{align}
            \ce{n_1A + n_2B + \cancel{n_3C}} & \ce{-> \cancel{n_3C} + n_5D + n_6E} & \ce{\Delta H = h_1 + h_2}\\
            \ce{n_1A + n_2B} & \ce{-> n_5D + n_6E} & \ce{\Delta H = h_1 + h_2}
        \end{align}
    
    \section{Conversion}
        We can turn these equations into matrices.
        Suppose a system of chemical equations. 
        \begin{align}
            \ce{n_1A + n_2B} & \ce{-> n_3C} & \ce{\Delta H = h_1}\\
            \ce{n_3C} & \ce{-> n_4D + n_5E} & \ce{\Delta H = h_2}
        \end{align}
        Suppose a matrix where every single column corresponds to a chemical within the equation and every equation to a row.
        The last column can correspond to $\Delta H$.
        Reactants have negative coefficients to the coefficients in the equation while products will have the their coefficients as in the equation.
        Please excuse poor spacing.
        \begin{gather*}
            \begin{array}{c c c c c c c c c c c c}
                A && B && C && D && E && \Delta H
            \end{array}\\
            \left[ 
                \begin{array}{c c c c c | c}
                    -n_1 & -n_2 & n_3 & 0 & 0   & h_1\\
                    0 & 0 & -n_3 & n_4 & n_5    & h_2
                \end{array}
            \right]
        \end{gather*}

        This results in a matrix we can use to get specific combinations that we want.
        \begin{equation}
            \left[ 
                \begin{array}{c c c c c | c}
                    -n_1 & -n_2 & n_3 & 0 & 0   & h_1\\
                    0 & 0 & -n_3 & n_4 & n_5    & h_2
                \end{array}
            \right]
        \end{equation}
\end{document}