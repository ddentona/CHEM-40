\documentclass[10pt]{article}
\usepackage[export]{adjustbox}
\usepackage{amsmath}
\usepackage[makeroom]{cancel}
\usepackage{enumitem}
\usepackage{graphicx}
%Load mhchem using some package options
\usepackage[version=4]{mhchem}
\usepackage{multicol}
\usepackage{siunitx}

\title{
    Problem Set \#6
    \\  \small
    CHEM101A: General College Chemistry
    }
\author{Donald Aingworth IV}
\date{September 26, 2025}

\newcommand{\E}[1]{\times 10^{#1}}
\newcommand{\U}[1]{\underline{#1}}

\begin{document}
    \DeclareSIUnit{\atm}{atm}
    \DeclareSIUnit{\molarity}{M}
    \DeclareSIUnit{\M}{M}

    \maketitle

    % \setcounter{section}{9}

    \pagebreak
    \section{Topic D Problem 8}
        Consider the following two facts about the reaction of HCl with NaOH
        
        Fact \#1: $\rm \Delta E$ is a negative number for this reaction, which means that the energy of the chemicals decreases.
        
        Fact \#2: When HCl and NaOH are mixed, the mixture becomes hotter, which means that the energy of the chemicals increases.
        
        Explain how both of these statements can be true at the same time.

        \subsection{Solution}

    \pagebreak
    \section{Topic D Problem 9}
        When aqueous HCl reacts with solid \ce{NaHCO3}, the mixture becomes colder.
        \begin{enumerate}
            \item   Does the kinetic energy of the chemicals increase, decrease, or remain the same during the reaction?
            \item   Does the potential energy of the chemical increase, decrease, or remain the same during the reaction?
            \item   What is the sign of $\rm \Delta E$ for this reaction?
        \end{enumerate}
        
        \subsection{Solution}

    \pagebreak
    \section{Topic D Problem 10}
        For the reaction 2 Al(s) + 6 H+(aq) -> 2 Al3+(aq) + 3 H2(g), $\Delta E$ = -1057 kJ
a) How much energy is given off when 2.822 g of Al reacts with hydrogen ions, assuming
the hydrogen ions are in excess?
b) How much energy is given off when 1.413 g of Al is added to 23.0 mL of a solution that
contains 5.89 M H+?
c) If you want to obtain 50.0 kJ of energy from this reaction, how many grams of aluminum
and how many mL of 3.00 M HCl must you use?
d) If this reaction produces 13.3 L of H2(g) at 25.0\unit{\celsius} and 1.014 atm, how much energy does
it give off?
        
        \subsection{Solution}

    \pagebreak
    \section{Topic D Problem 11}
        When 4.00 g of Br2 reacts with excess Al to form AlBr3, 8.80 kJ of energy is given off.
        Calculate $\Delta E$ for the reaction 2 Al(s) + 3 Br2(s) -> 2 AlBr3(s).
        
        \subsection{Solution}

    \pagebreak
    \section{Topic D Problem 12}
        A chemist burns 0.5177 g of liquid hexane (C6H14) in a bomb calorimeter that has a heat
capacity of 4287 J/\unit{\celsius}. The temperature of the calorimeter rises from 21.382\unit{\celsius} to 27.170ºC
during the reaction. Calculate $\Delta E$ for the following reaction:
2 C6H14(l) + 19 O2(g) -> 12 CO2(g) + 14 H2O(l)
        
        \subsection{Solution}

    \pagebreak
    \section{Topic D Problem 13}
        Explain why some reactions give off different amounts of heat when they are carried out at constant pressure versus constant volume.
        
        \subsection{Solution}

    \pagebreak
    \section{Topic D Problem 14}
        A reaction has $\Delta E$ = -50 kJ.
a) If this reaction occurs, will the reaction mixture become hotter, or colder?
b) Is this an exothermic reaction, or is it an endothermic reaction?
c) After this reaction has occurred, will the potential energy of the chemical mixture be
higher than, lower than, or the same as it was before the reaction?
d) After this reaction has occurred, will the kinetic energy of the universe be higher than,
lower than, or the same as it was before the reaction (assuming no other reaction has
occurred)?
        
        \subsection{Solution}

    \pagebreak
    \section{Topic D Problem 15}
        A reaction has $\Delta E$ = 10 kJ, and it converts a solid into a gas.
a) If this reaction occurs in a closed, rigid container, will the amount of heat that is absorbed
be larger than 10 kJ, smaller than 10 kJ, or exactly 10 kJ? Justify your answer.
b) If this reaction occurs in an open container, will the amount of heat that is absorbed be
larger than 10 kJ, smaller than 10 kJ, or exactly 10 kJ? Justify your answer.
c) Will PV work be done in either part a or part b? If so, which part(s)?
d) What is the sign of the PV work when PV work occurs?
        
        \subsection{Solution}

    \pagebreak
    \section{Topic D Problem 16}
        A reaction has $\Delta E$ = -10 kJ, and it converts a gas into a liquid.
a) If this reaction occurs in a closed, rigid container, will the amount of heat that is given off
be larger than 10 kJ, smaller than 10 kJ, or exactly 10 kJ? Justify your answer.
b) If this reaction occurs in an open container, will the amount of heat that is given off be
larger than 10 kJ, smaller than 10 kJ, or exactly 10 kJ? Justify your answer.
c) Will PV work be done in either part a or part b? If so, which part(s)?
d) What is the sign of the PV work when PV work occurs?
        
        \subsection{Solution}

    \pagebreak
    \section{Topic D Problem 17}
        A reaction has $\Delta E$ = -10 kJ, and it converts a liquid into a gas.
a) If this reaction occurs in a closed, rigid container, will the amount of heat that is given off
be larger than 10 kJ, smaller than 10 kJ, or exactly 10 kJ? Justify your answer.
b) If this reaction occurs in an open container, will the amount of heat that is given off be
larger than 10 kJ, smaller than 10 kJ, or exactly 10 kJ? Justify your answer.
c) Will PV work be done in either part a or part b? If so, which part(s)?
d) What is the sign of the PV work when PV work occurs?
        
        \subsection{Solution}

    \pagebreak
    \section{Topic D Problem 18}
        a) What is the PV work when 3.00 moles of liquid water boils at 100\unit{\celsius} at constant pressure?
Give your answer in kJ, and include the correct sign.
b) What is the PV work when 3.00 moles of steam condenses at 100\unit{\celsius} at constant pressure?
Give your answer in kJ, and include the correct sign.

        
        \subsection{Solution}

    \pagebreak
    \section{Topic D Problem 19}
        Consider the following reaction:
4 C3H9N(l) + 21 O2(g) -> 12 CO2(g) + 18 H2O(l) + 2 N2(g) $\Delta E$ = -9667 kJ
The following questions (parts a through j) refer to this reaction.
a) Calculate $\Delta H$ for this reaction at 25\unit{\celsius}.
b) How much heat will be given off if 8.250 g of C3H9N reacts with excess O2 in a closed,
rigid container at 25\unit{\celsius}?
c) How much heat will be given off if 8.250g of C3H9N reacts with excess O2 in an open
container at 25\unit{\celsius}?
d) Calculate the PV work in part b. Include the correct sign.
e) Calculate the PV work in part c. Include the correct sign.
f) If you want to obtain 200.0 kJ of heat by reacting C3H9N with O2 in an open container,
what mass of C3H9N must you use?
g) How much heat is produced when 2.199 g of liquid C3H9N reacts with 5.738 g of gaseous
O2 in an open container?
h) If you burn enough liquid C3H9N to produce 841.2 kJ of heat in an open container, what
volume of gaseous N2 will you form at 25.0\unit{\celsius} and 752 torr?
i) If this reaction is carried out in a closed, rigid container and produces 31.74 g of liquid
H2O, how much heat does it produce?
        
        \subsection{Solution}

    \pagebreak
    \section{Topic D Problem 20}
        When 2.810 g of solid Al reacts with excess gaseous F2 in an open container at 25\unit{\celsius}, 157.3
kJ of heat is produced.
a) Calculate $\Delta H$ and $\Delta E$ for the following reaction at 25\unit{\celsius}:
2 Al(s) + 3 F2(g) → 2 AlF3(s)
b) How much heat will be produced if 8.493 g of solid Al reacts with 16.610 g of gaseous F2
in a closed, rigid container at 25\unit{\celsius}? Use your answer from part a to solve this problem.
        
        \subsection{Solution}

    \pagebreak
    \section{Topic D Problem 21}
        A chemist burns 1.628 g of liquid isopropyl alcohol (C3H8O) in a bomb calorimeter that has a
heat capacity of 3927 J/\unit{\celsius}. The temperature of the calorimeter rises from 19.085\unit{\celsius} to
31.683\unit{\celsius} during the reaction. Calculate $\Delta H$ and $\Delta E$ for the following reaction at 25\unit{\celsius}:
2 C3H8O(l) + 9 O2(g) -> 6 CO2(g) + 8 H2O(l)
        
        \subsection{Solution}

    \pagebreak
    \section{Topic D Problem 22}
        For the reaction 2 C(s) + O2(g) → 2 CO(g), $\Delta E$ = -219 kJ. When will $\Delta H$ also be -219 kJ?
a) When the reaction is carried out in an open container.
b) When the reaction is carried out in a closed, rigid container.
c) $\Delta H$ will always equal $\Delta E$.
d) $\Delta H$ will never equal $\Delta E$.
        
        \subsection{Solution}

\end{document}