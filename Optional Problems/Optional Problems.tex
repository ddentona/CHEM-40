\documentclass[11pt]{report}
\usepackage{amsmath}
\usepackage{mhchem}
\usepackage{siunitx}
\usepackage{titlesec}

\titleformat{\section}{\bfseries\Large}{Problem}{0.25em}{\vspace{1ex}}

\begin{document}
Ch. 2 Problems: 5, 8, 9, 39, 41, 55, 63, 67, 71, 75, 83, 85, 91, 93, 99, 109, 111, 115, 119, 121, 123, 125\\
Ch. 5 Problems: 9, 10, 25, 27, 29, 33, 37, 39, 41, 43, 45, 47, 51, 53, 57, 61, 65, 69, 71, 75, 77, 81, 83, 87, 89, 93, 95, 97, 101, 105, 107

7th Edition
\setcounter{chapter}{1}
\pagebreak
\chapter{}

\section{121}
A car has a mileage rating of 38 mi per gallon of gasoline. How many miles can the car travel on 76.5 L of gasoline?

\section{123}
Consider these observations on two blocks of different unknown metals:

\begin{center}
    \begin{tabular*}{0.25\textwidth}{c c}
        Block name  & Volume \\
        Block A     & 125 \unit{\centi\meter^3}\\
        Block B     & 145 \unit{\centi\meter}
    \end{tabular*}
\end{center}          
        
If block A has a greater mass than block B, what can be said of the relative densities of the two metals? (Assume that both blocks are solid.)

\subsection{Solution}
Block A has a greater mass and a smaller volume. As such, since $\rho = \frac{V}{m}$, it has a greater density.

\section{125}
You measure the masses and volumes of two cylinders. The mass of cylinder 1 is 1.35 times the mass of cylinder 2. The volume of cylinder 1 is 0.792 the volume of cylinder 1. If the density of cylinder 1 is 3.85 \unit{\gram/\centi\meter^3}, what is the density of cylinder 2?

\subsection{Solution}
\begin{gather}
    m_1 = 1.35 m_2\\
    V_1 = 0.792 V_2\\
    \rho_1 = \frac{m_1}{V_1} = \frac{1.35 m_2}{0.792 V_2} = \frac{1.35}{0.792}\rho_2\\
    \rho_2 = \frac{0.792}{1.35}\rho_1 = \frac{0.792}{1.35}*3.85 \unit{\gram/\centi\meter^3} = \boxed{2.26 \unit{\gram/\centi\meter^3}}
\end{gather}

\pagebreak

\chapter{}
\section{114}
A portable electric water heater transfers 255 watts (W) of power to 5.5L of water, where 1 W = 1 J/s. How much time (in minutes) does it take for the water heater to heat the 5.5 L of water from 25\unit{\celsius} to 42\unit{\celsius}? (Assume that water has a density of 1.0 \unit{\gram/\milli\liter}.) 

\subsection{Solution}
We should use an appropriate equation. 
The appropriate equation for this is the equation $q = mC\Delta T$, which allows us to find the amount of energy necessary for the temperature change.
We know that the temperature change is $\Delta T = T_f - T_i = 42\unit{\celsius} - 25\unit{\celsius} = 17\unit{\celsius}$.
We know that the specific heat of water is $4.184\unit{\joule/\gram\celsius}$.
We know the volume of water is $5.5\unit{\liter} = 5500\unit{\milli\liter} * 1\unit{\gram/\milli\liter} = 5500\unit{\gram}$.
\begin{align*}
    q   &=  mC\Delta T\\
        &=  (5500\unit{\gram})(4.184\unit{\joule/\gram\celsius})(17\unit{\celsius})\\
        &=  (93500\unit{\gram\celsius})(4.184\unit{\joule/\gram\celsius})\\
        &=  391204\unit{\joule}
\end{align*}

Now that we have the energy used, we need to find how long the water heater takes to generate that amount of energy.
\begin{equation*}
    \frac{391204\unit{\joule}}{255\unit{\joule/\second}} = 1534.1\bar{3}\unit{\second} = 25.56\bar{8}\unit{\minute} \approx \boxed{26\unit{\minute}}
\end{equation*}


\section{115}
What temperature on the Celsius scale is equal to twice its value when expressed on the Fahrenheit scale?

\subsection{Solution}
The conversion between Fahrenheit and Celsius is $T_F = \frac{9}{5} T_C + 32$. 
\[T_F = \frac{9}{5} T_C + 32\]
For the proposed to hold, the Fahrenheit value must be equal to the Celsius value.
\[T_C = \frac{9}{5} T_C + 32\]
We can subtract $\frac{9}{5} T_C$ from each side.
\[-\frac{4}{5} T_F = 32\]
We now can divide both sides by $-\frac{4}{5}$.
\[T_F = -40\]
THs means that the temperature that is the same is \boxed{-40\unit{\degreeCelsius}}.


\section{116}
What temperature on the Celsius scale is equal to twice its value when expressed on the Fahrenheit scale?

\subsection{Solution}
The conversion between Fahrenheit and Celsius is $T_F = \frac{9}{5} T_C + 32$.
\[ T_F = \frac{9}{5} T_C + 32 \]
In this instance, $2T_F = T_C$.
\[ T_F = \frac{18}{5}T_F + 32 \]
We can subtract $\frac{18}{5}T_C$ from both sides.
\[ -\frac{13}{5}T_F = 32 \]
We can lastly multiply both sides by 5.
\[ T_F = -12.3078 \]
Multiplying this by two, we get the temperature in Celsius.
\[ T_C = -24.6154 \]
Thus the answer is \boxed{-24.6154\unit{\celsius}}.

\setcounter{chapter}{4}
\pagebreak

\chapter{}

\section{9}
What is the difference between a molecular element and an atomic element? List the elements that occur as diatomic molecules.

\subsection{Solution}
An atomic element contains a single copy of the element as a core building block/molecule.
A molecular element contains multiple of the same element in each of its molecules. 
The following are the diatomic molecules: $H_2, N_2, O_2, F_2, Cl_2, Br_2, I_2$.
The element names are hydrogen, nitrogen, oxygen, fluorine, chlorine, bromine, and iodine.
They are generalized as $\mathcal{D}_2$.
\pagebreak

\section{107}
Carbon has two naturally occurring isotopes: carbon-12 (mass = 12.00 amu) and carbon-13 (mass = 13.00 amu). Chlorine also has two naturally occurring isotopes: chlorine-35 (mass = 34.97 amu) and chlorine-37 (mass = 36.97 amu). How many \ce{CCl4} molecules of different masses can exist? Determine the mass (in amu) of each of them.

\subsection{Solution}
There are ten.
\begin{center}
    \begin{tabular}{ c | c | c | c | c | c }
        Number  &C-12   &C-13   &Cl-35  &Cl-37  &Mass (amu)\\
        \hline
        1       &1      &0      &4      &0      &151.88\\
        2       &0      &1      &4      &0      &152.88\\
        3       &1      &0      &3      &1      &153.88\\
        4       &0      &1      &3      &1      &154.88\\
        5       &1      &0      &2      &2      &155.88\\
        6       &0      &1      &2      &2      &156.88\\
        7       &1      &0      &1      &3      &157.88\\
        8       &0      &1      &1      &3      &158.88\\
        9       &1      &0      &0      &4      &159.88\\
        10      &0      &1      &0      &4      &160.88
    \end{tabular}
\end{center}

\end{document}