\documentclass[10pt]{article}
\usepackage[export]{adjustbox}
\usepackage{amsmath}
\usepackage{array}
\usepackage[makeroom]{cancel}
\usepackage{chemfig}
\usepackage{enumitem}
\usepackage{float}
\usepackage{graphicx}
%Load mhchem using some package options
\usepackage[version=4]{mhchem}
\usepackage{multicol}
\usepackage{siunitx}
\usepackage{wrapfig}

\title{
    Problem Set \#13
    \\  \small
    CHEM101A: General College Chemistry
    }
\author{Donald Aingworth}
\date{November 7, 2025}

\newcommand{\E}[1]{\times 10^{#1}}
\newcommand{\hc}{1.9864748\E{-25}\,\unit{\joule\,\meter}}
\newcommand{\U}[1]{\underline{#1}}

\begin{document}
    \DeclareSIUnit{\atm}{atm}
    \DeclareSIUnit{\molarity}{M}
    \DeclareSIUnit{\M}{M}
    \DeclareSIUnit{\torr}{torr}

    \maketitle

    \setcounter{section}{21}
    \pagebreak
    \section{Topic F Problem 22}
        Draw pictures showing how each of the following MOs is formed by combining the specified atomic orbitals. 
        Include the signs of the lobes for each atomic orbital.
        \begin{enumerate}[label=\alph*)]
            \item   a sigma bonding MO that is formed by two 2s orbitals
            \item   a sigma bonding MO that is formed by a 1s orbital and a 2p orbital
            \item   a sigma antibonding MO that is formed by two 2p orbitals
            \item   a pi antibonding MO that is formed by two 2p orbitals
        \end{enumerate}
        
        \subsection{Solution}

    \pagebreak
    \section{Topic F Problem 23}
        A d orbital and a p orbital can combine to form molecular orbitals in several different ways.
        Draw a picture of the overlap between these two atomic orbitals that would produce each of the following molecular orbitals. 
        Include the signs of the lobes for each atomic orbital.
        \begin{multicols}{2}
            \begin{enumerate}[label=\alph*)]
                \item   a sigma bonding MO
                \item   a sigma antibonding MO
                \item   a pi bonding MO
                \item   a pi antibonding MO
            \end{enumerate}
        \end{multicols}
        
        \subsection{Solution}

    \pagebreak
    \section{Topic F Problem 24}
        \begin{wrapfigure}{r}{0.25\textwidth}
            \vspace{-30pt}
            \includegraphics[width=0.25\textwidth]{img-F24.png}
            % \label{fig:wrapfig}
        \end{wrapfigure}
        The molecular orbital energy diagram for the valence orbitals of the \ce{NO-} ion is shown below. 
        Use this diagram to answer the following questions.
        \begin{enumerate}[label=\alph*)]
            \item   What is the bond order in \ce{NO-}?
            \item   Is \ce{NO-} diamagnetic, or is it paramagnetic? How can you tell?
            \item   Which has the larger bond distance, \ce{NO-} or NO? Assume that this energy diagram also applies to NO.
        \end{enumerate}
        
        \subsection{Solution}

    \pagebreak
    \section{Topic F Problem 25}
        \begin{wrapfigure}{r}{0.25\textwidth}
            \vspace{-30pt}
            \includegraphics[width=0.25\textwidth]{img-F25.png}
            % \label{fig:wrapfig}
        \end{wrapfigure}
        The molecular orbital energy diagram for the valence orbitals of HF is shown below. 
        Use this diagram to answer the following questions. 
        Note that the orbitals labeled $\rm 2s_F$ and $\rm 2p_F$ are nonbonding orbitals on the fluorine atom; electrons in these orbitals do not affect the bond order.
        \begin{enumerate}[label=\alph*)]
            \item   Based on this diagram, what is the bond order in HF?
            \item   Is HF diamagnetic, or is it paramagnetic? How can you tell?
            \item   How many nonbonding electrons are there in HF?
            \item   Draw a picture that shows how the $\sigma$ orbital is formed from atomic orbitals on hydrogen and fluorine.
            \item   If an electron were removed from the HF molecule, how would the bond energy be affected?
        \end{enumerate}
        
        \subsection{Solution}

    \pagebreak
    \setcounter{section}{0}
    \section{Topic G Problem 1}
        List the defining characteristics of a dynamic equilibrium as described in the video.
        
        \subsection{Solution}

    \pagebreak
    \section{Topic G Problem 2}
        \begin{enumerate}[label=\alph*)]
            \item   The formation and decomposition ammonium chloride is provided as an example of a reversible chemical reaction. Describe the characteristics required of an experimental set-up in order to establish a dynamic chemical equilibrium.
            \item   Write the equation representing the chemical equilibrium for the system of ammonium chloride, ammonia, and hydrogen chloride.
        \end{enumerate}
        
        \subsection{Solution}

    \pagebreak
    \section{Topic G Problem 3}
        \begin{enumerate}[label=\alph*)]
            \item   What is a heterogeneous chemical equilibrium? Search the internet to find an example of a heterogeneous chemical equilibrium system (other than the one in the video.) Write the equilibrium equation for your example.
            \item   What is a homogenous chemical equilibrium? Search the internet to find an example of a homogeneous equilibrium system. Write the equilibrium equation for your example.
        \end{enumerate}
        
        \subsection{Solution}

    \pagebreak
    \section{Topic G Problem 4}
        How can you tell from a graph of reaction rate vs. time when a system has established chemical equilibrium? (What is true for the rate of the forward reaction and the rate of the reverse reaction at equilibrium?)
        
        \subsection{Solution}

    \pagebreak
    \section{Topic G Problem 5}
        How can you tell from a graph of concentrations vs. time when a system has established chemical equilibrium? (What is true for the concentrations of the products and the concentrations of the reactants at equilibrium?)
        
        \subsection{Solution}

    \pagebreak
    \section{Topic G Problem 6}
        For the reaction below, the equilibrium constant K is greater than 1.
        \begin{center}
            fructose  glucose  K >> 1
        \end{center}
        If a solution initially contains equal concentrations of fructose and glucose, which of the following statements must be true at that initial moment? 
        Explain your answer.
        \begin{enumerate}[label=\alph*)]
            \item   The forward reaction (fructose $\to$ glucose) is faster than the reverse reaction.
            \item   The reverse reaction is faster than the forward reaction.
            \item   The forward and reverse reactions have equal rates.
        \end{enumerate}
        
        \subsection{Solution}

    \pagebreak
    \section{Topic G Problem 7}
        Write the $K_c$ expressions for each of the following reactions. 
        Note: the subscript ``c'' tells you that this is the equilibrium constant in terms of concentrations (mol/L).
        \begin{enumerate}[label=\alph*)]
            \item   \ce{2 NO(g) + O2(g) <-> 2 NO2(g)}
            \item   \ce{4 Ag(s) + O2(g) <-> 2 Ag2O(s)}
            \item   \ce{CaCO3(s) + CO2(aq) + H2O(l) <-> Ca2+(aq) + 2 HCO3-(aq)}
        \end{enumerate}
        
        \subsection{Solution}

    \pagebreak
    \section{Topic G Problem 8}
        \begin{enumerate}[label=\alph*)]
            \item   Write the $K_p$ expressions for reactions a and b in the previous problem. Note: the subscript “p” tells you that this is the equilibrium constant in terms of partial pressures(atm).
            \item   If the value of $K_c$ for reaction (a) in the previous is $2.8\E{11}$ at 200\unit{\celsius}, what is the value of $K_p$ at this temperature?
        \end{enumerate}
        
        \subsection{Solution}

    \pagebreak
    \section{Topic G Problem 9}
        For the reaction below, $K_c$ = 6.17 at a certain temperature.
        \begin{center}
            \ce{PCl3(g) + Cl2(g) <-> PCl5(g)}
        \end{center}
        Determine whether each of the following mixtures is at equilibrium. 
        Assume that each mixture is at the same temperature as that for the provided $K_c$.
        For each mixture that is not at equilibrium, tell whether the reaction will go forward or backward.
        \begin{enumerate}[label=\alph*)]
            \item   A mixture in which the concentration of \ce{PCl3} is 0.0381 M, the concentration of \ce{Cl2} is 0.0593 M, and the concentration of \ce{PCl5} is 0.0139 M.
            \item   A mixture in which the concentration of \ce{PCl3} is 0.0482 M, the concentration of \ce{Cl2} is 0.289 M, and the concentration of \ce{PCl5} is 0.0455 M.
            \item   A mixture that contains \ce{PCl3} and \ce{PCl5}, but no \ce{Cl2}.
            \item   A mixture that contains 0.21 mol of \ce{PCl3}, 0.48 mol of \ce{Cl2}, and 0.39 mol of \ce{PCl5} in an 8.00 L container.
        \end{enumerate}
        
        \subsection{Solution}

    \pagebreak
    \section{Topic G Problem 10}
        If Q, the reaction quotient, is greater than K for a reaction mixture, which of the following will happen?
        \begin{enumerate}[label=\alph*)]
            \item   The reaction will go in the direction that increases Q.
            \item   The reaction will go in the direction that decreases Q.
            \item   The reaction will go in the direction that increases K.
            \item   The reaction will go in the direction that decreases K.
        \end{enumerate}
        
        \subsection{Solution}

    \pagebreak
    \section{Topic G Problem 11}
        For the reaction \ce{N2(g) + 3 H2(g) <-> 2 NH3(g)}, $K_p$ = 0.0489 at 256\unit{\celsius}. 
        For parts a through d, assume that the reaction is at this temperature.
        \begin{enumerate}[label=\alph*)]
            \item   An equilibrium mixture contains 0.100 atm of \ce{N2} and 0.200 atm of \ce{H2}. What is the partial pressure of \ce{NH3} in this mixture?
            \item   A second equilibrium mixture contains 830 torr of \ce{H2} and 42 torr of \ce{NH3}. What is the partial pressure of \ce{N2} in this mixture?
            \item   A third equilibrium mixture contains 0.0100 mol/L of \ce{N2} and 0.0300 mol/L of \ce{NH3}. What is the concentration of \ce{H2} in this mixture?
            \item   A fourth equilibrium mixture contains equal concentrations of all three chemicals. What is the pressure of each substance in this mixture?
        \end{enumerate}
        
        \subsection{Solution}

    \pagebreak
    \tableofcontents
\end{document}