\documentclass[10pt]{article}
\usepackage[export]{adjustbox}
\usepackage{amsmath}
\usepackage{array}
\usepackage[makeroom]{cancel}
\usepackage{chemfig}
\usepackage{enumitem}
\usepackage{float}
\usepackage{graphicx}
%Load mhchem using some package options
\usepackage[version=4]{mhchem}
\usepackage{multicol}
\usepackage{siunitx}
\usepackage{wrapfig}

\title{
    Problem Set \#13
    \\  \small
    CHEM101A: General College Chemistry
    }
\author{Donald Aingworth}
\date{November 14, 2025}

\newcommand{\E}[1]{\times 10^{#1}}
\newcommand{\hc}{1.9864748\E{-25}\,\unit{\joule\,\meter}}
\newcommand{\U}[1]{\underline{#1}}

\begin{document}
    \DeclareSIUnit{\atm}{atm}
    \DeclareSIUnit{\molarity}{M}
    \DeclareSIUnit{\M}{M}
    \DeclareSIUnit{\torr}{torr}

    \maketitle

    \setcounter{section}{21}
    \pagebreak
    \section{Topic F Problem 22}
        Draw pictures showing how each of the following MOs is formed by combining the specified atomic orbitals. 
        Include the signs of the lobes for each atomic orbital.
        \begin{enumerate}[label=\alph*)]
            \item   a sigma bonding MO that is formed by two 2s orbitals
            \item   a sigma bonding MO that is formed by a 1s orbital and a 2p orbital
            \item   a sigma antibonding MO that is formed by two 2p orbitals
            \item   a pi antibonding MO that is formed by two 2p orbitals
        \end{enumerate}
        
        \subsection{Solution}
            \begin{center}
                \includegraphics[width=\textwidth]{Answers Images/F22.jpg}
            \end{center}

    \pagebreak
    \section{Topic F Problem 23}
        A d orbital and a p orbital can combine to form molecular orbitals in several different ways.
        Draw a picture of the overlap between these two atomic orbitals that would produce each of the following molecular orbitals. 
        Include the signs of the lobes for each atomic orbital.
        \begin{multicols}{2}
            \begin{enumerate}[label=\alph*)]
                \item   a sigma bonding MO
                \item   a sigma antibonding MO
                \item   a pi bonding MO
                \item   a pi antibonding MO
            \end{enumerate}
        \end{multicols}
        
        \subsection{Solution}
            \begin{center}
                \includegraphics[width=0.7\textwidth]{Answers Images/F23.jpg}
            \end{center}

    \pagebreak
    \section{Topic F Problem 24}
        \begin{wrapfigure}{r}{0.25\textwidth}
            \vspace{-30pt}
            \includegraphics[width=0.25\textwidth]{img-F24.png}\\
            \includegraphics[width=0.25\textwidth]{Answers Images/F24.jpg}
            % \label{fig:wrapfig}
        \end{wrapfigure}
        The molecular orbital energy diagram for the valence orbitals of the \ce{NO-} ion is shown below. 
        Use this diagram to answer the following questions.
        \begin{enumerate}[label=\alph*)]
            \item   What is the bond order in \ce{NO-}?
            \item   Is \ce{NO-} diamagnetic, or is it paramagnetic? How can you tell?
            \item   Which has the larger bond distance, \ce{NO-} or NO? Assume that this energy diagram also applies to NO.
        \end{enumerate}
        
        \subsection{Solution (a)}
            Add up the bonding electrons and subtract the antibonding electrons.
            Then divide the result by two.
            The bond order is the quotient.
            \begin{equation}
                \frac{8 - 4}{2} = \frac{4}{2} = \boxed{2}
            \end{equation}

        \subsection{Solution (b)}
            There are empty bond orbitals (both $\pi_{\rm 2p}^*$). 
            This means that the \ce{NO-} as a molecule is \underline{paramagnetic}.

        \subsection{Solution (c)}
            If we look at the diagram for NO, the only difference is one fewer electron in $\pi_{\rm 2p}^*$. 
            This raises the bond order to 2.5.
            That in turn corresponds to a smaller bond distance.
            This means that \boxed{\ce{NO-}} has the larger bond distance.

    \pagebreak
    \section{Topic F Problem 25}
        \begin{wrapfigure}{r}{0.25\textwidth}
            \vspace{-30pt}
            \includegraphics[width=0.25\textwidth]{img-F25.png}\\
            \includegraphics[width=0.25\textwidth]{Answers Images/F25.jpg}
            % \label{fig:wrapfig}
        \end{wrapfigure}
        The molecular orbital energy diagram for the valence orbitals of HF is shown below. 
        Use this diagram to answer the following questions. 
        Note that the orbitals labeled $\rm 2s_F$ and $\rm 2p_F$ are nonbonding orbitals on the fluorine atom; electrons in these orbitals do not affect the bond order.
        \begin{enumerate}[label=\alph*)]
            \item   Based on this diagram, what is the bond order in HF?
            \item   Is HF diamagnetic, or is it paramagnetic? How can you tell?
            \item   How many nonbonding electrons are there in HF?
            \item   Draw a picture that shows how the $\sigma$ orbital is formed from atomic orbitals on hydrogen and fluorine.
            \item   If an electron were removed from the HF molecule, how would the bond energy be affected?
        \end{enumerate}
        
        \subsection{Solution}
            \begin{enumerate}[label=\alph*/]
                \item   One ($\frac{2 - 0}{2}$). We ignore the 2s and 2p orbitals' electrons because they are nonbonding. 
                \item   Diamagnetic. There are no unpaired electrons.
                \item   Six, all those contained in the 2s and 2p orbitals.
                \item   See below. Bonds not drawn to scale.
                \item   The electron would come out of the 2p orbital from the Fluorine. As such, it would have no effect. 
            \end{enumerate}

            \begin{center}
                \includegraphics[width=0.7\textwidth]{Answers Images/F24d.jpg}
            \end{center}

    \pagebreak
    \setcounter{section}{0}
    \section{Topic G Problem 1}
        List the defining characteristics of a dynamic equilibrium as described in the video.
        
        \subsection{Solution}
            The system must be closed, so general conditions like temperature, pressure, and concentration must remain constant.
            Processes (reactions included) must be able to be performed in both directions. 
            There remains (roughly) a constant amount (or concentration) of each chemical in each state at every point.
            The rate of forward process is equal to the rate of reverse process.
            This follows from the prior conditions, but no change should be detected on a macroscopic scale.

    \pagebreak
    \section{Topic G Problem 2}
        \begin{enumerate}[label=\alph*)]
            \item   The formation and decomposition ammonium chloride is provided as an example of a reversible chemical reaction. Describe the characteristics required of an experimental set-up in order to establish a dynamic chemical equilibrium.
            \item   Write the equation representing the chemical equilibrium for the system of ammonium chloride, ammonia, and hydrogen chloride.
        \end{enumerate}
        
        \subsection{Solution (a)}
            The chloride and ammonium (both together and separately) must be placed in a closed system, where the temperature, pressure, and concentration of each chemical is constant.
            The temperature must be at a point that allows for the central reaction to be performed both forward and in reverse.
            No chemical can be added or removed from the system, but the chemicals can react with one another until they reach quantities necessary for chemical equilibrium.
            These would allow for the changes detectable on a macroscopic scale to be largely reduced to none.

        \subsection{Solution (b)}
        \begin{center}
            \ce{NH3 (g) + HCl (g) <-> NH4Cl (s)}
        \end{center}


    \pagebreak
    \section{Topic G Problem 3}
        \begin{enumerate}[label=\alph*)]
            \item   What is a heterogeneous chemical equilibrium? Search the internet to find an example of a heterogeneous chemical equilibrium system (other than the one in the video.) Write the equilibrium equation for your example.
            \item   What is a homogenous chemical equilibrium? Search the internet to find an example of a homogeneous equilibrium system. Write the equilibrium equation for your example.
        \end{enumerate}
        
        \subsection{Solution (a)}
            A heterogeneous chemical equilibrium is a chemical equilibrium where the reactants and products are not all in the same physical state. 
            One such example found on the web is calcium carbonate.
            \begin{center}
                \ce{CaCO3 (s) <-> CaO (s) + CO2 (g)}
            \end{center}
            As we can see, the Calcum Carbonate and Calcium Oxide are both solid, while the Carbon Dioxide is gaseous.

        \subsection{Solution (b)}
            A homogeneous chemical equilibrium is a chemical equilibrium where the reactants and products are all in the same physical state. 
            One such example found on the web regards ammonium and is called the Haber-Bosch process.
            \begin{center}
                \ce{N2 (g) + 3H2 (g) <-> 2NH3 (g)}
            \end{center}
            All three substances are gaseous at all times, and they still react.

    \pagebreak
    \section{Topic G Problem 4}
        How can you tell from a graph of reaction rate vs. time when a system has established chemical equilibrium? (What is true for the rate of the forward reaction and the rate of the reverse reaction at equilibrium?)
        
        \subsection{Solution}
            On a graph of reaction rate over time, chemical equilibrium will be achieved when the reaction rates of the forward and reverse reaction are equal.
            They would also in all likelihood be constant over time.

    \pagebreak
    \section{Topic G Problem 5}
        How can you tell from a graph of concentrations vs. time when a system has established chemical equilibrium? (What is true for the concentrations of the products and the concentrations of the reactants at equilibrium?)
        
        \subsection{Solution}
            When graphing concentrations over time, the concentrations of each chemical would have to be constant over time going forward, but they would not have to be identical, since there would not be an identical amount of each necessarily.

    \pagebreak
    \section{Topic G Problem 6}
        For the reaction below, the equilibrium constant K is greater than 1.
        \begin{center}
            fructose $\rightleftharpoons$ glucose  K $>>$ 1
        \end{center}
        If a solution initially contains equal concentrations of fructose and glucose, which of the following statements must be true at that initial moment? 
        Explain your answer.
        \begin{enumerate}[label=\alph*)]
            \item   The forward reaction (fructose $\to$ glucose) is faster than the reverse reaction.
            \item   The reverse reaction is faster than the forward reaction.
            \item   The forward and reverse reactions have equal rates.
        \end{enumerate}
        
        \subsection{Solution}
            (a) would be the only one true. 
            Since K $>$ 1, that means that by equilibrium, there will be more glucose than frucose.
            If we start with an equal amount of each, there would be more glucose being produced than fructose so as to approach the equilibrium.
            That means that the reaction to create glucose (the forward reaction) will occur more (or be faster) than the reverse reaction.

    \pagebreak
    \section{Topic G Problem 7}
        Write the $K_c$ expressions for each of the following reactions. 
        Note: the subscript ``c'' tells you that this is the equilibrium constant in terms of concentrations (mol/L).
        \begin{enumerate}[label=\alph*)]
            \item   \ce{2 NO(g) + O2(g) <-> 2 NO2(g)}
            \item   \ce{4 Ag(s) + O2(g) <-> 2 Ag2O(s)}
            \item   \ce{CaCO3(s) + CO2(aq) + H2O(l) <-> Ca2+(aq) + 2 HCO3-(aq)}
        \end{enumerate}
        
        \subsection{Solution (a)}
            \begin{equation}
                \rm K_c = \frac{\left[ \ce{NO2} \right]^2}{[\ce{NO}]^2 [\ce{O2}]}
            \end{equation}

        \subsection{Solution (b)}
            \[ \rm K_c = \frac{[\ce{Ag2O}]^2}{[\ce{Ag}]^4 [\ce{O2}]} = \frac{1}{[\ce{O2}]} \]

        \subsection{Solution (c)}
            \[ \rm K_c = \frac{[\ce{Ca2+}][\ce{HCO3-}]^2}{[\ce{CaCO3}] [\ce{CO2}] [\ce{H2O}]} = \frac{[\ce{Ca2+}][\ce{HCO3-}]^2}{[\ce{CO2}]} \]

    \pagebreak
    \section{Topic G Problem 8}
        \begin{enumerate}[label=\alph*)]
            \item   Write the $K_p$ expressions for reactions a and b in the previous problem. Note: the subscript “p” tells you that this is the equilibrium constant in terms of partial pressures (atm).
            \item   If the value of $K_c$ for reaction (a) in the previous is $2.8\E{11}$ at 200\unit{\celsius}, what is the value of $K_p$ at this temperature?
        \end{enumerate}
        
        \subsection{Solution (a)}
            This is similar to the last problem, but you use pressures rather than only consider gases.\\
            a/ 
            \begin{equation}
                \rm K_p = \frac{ P_{\ce{NO2}}^2}{ P_{\ce{NO}}^2 P_{\ce{O2}}}
            \end{equation}
            b/ 
            \begin{equation}
                \rm K_p = \frac{1}{P_{\ce{O2}}}
            \end{equation}

        \subsection{Solution (b)}
            This is solvable using an equation for $\rm K_p$ from $\rm K_c$. 
            \begin{equation}
                \rm K_p =   K_c (RT)^{\Delta n}
            \end{equation}

            We have values for $\rm K_c$, R (the value involving atmospheres), and T.
            Our value for $\Delta n$ (change in number of gaseous moles) will be -1 since the only gas is \ce{O2}, which we lose a mole of.
            \begin{align}
                \rm
                K_p &=  K_c (RT)^{\Delta n}
                    =   2.8\E{11}\,\unit{\frac{\liter}{\mole}} \left( 0.08206\,\unit{\frac{\atm\cdot\liter}{\mole\cdot\kelvin}} \times 473.15\unit{\kelvin} \right)^{-1}\\
                    &=  7.212\E{9}\,\unit{\frac{1}{\atm}}
                    =   \boxed{7.2\E{9}\,\unit{\frac{1}{\atm}}}
            \end{align}

    \pagebreak
    \section{Topic G Problem 9}
        For the reaction below, $K_c$ = 6.17 at a certain temperature.
        \begin{center}
            \ce{PCl3(g) + Cl2(g) <-> PCl5(g)}
        \end{center}
        Determine whether each of the following mixtures is at equilibrium. 
        Assume that each mixture is at the same temperature as that for the provided $K_c$.
        For each mixture that is not at equilibrium, tell whether the reaction will go forward or backward.
        \begin{enumerate}[label=\alph*)]
            \item   A mixture in which the concentration of \ce{PCl3} is 0.0381 M, the concentration of \ce{Cl2} is 0.0593 M, and the concentration of \ce{PCl5} is 0.0139 M.
            \item   A mixture in which the concentration of \ce{PCl3} is 0.0482 M, the concentration of \ce{Cl2} is 0.289 M, and the concentration of \ce{PCl5} is 0.0455 M.
            \item   A mixture that contains \ce{PCl3} and \ce{PCl5}, but no \ce{Cl2}.
            \item   A mixture that contains 0.21 mol of \ce{PCl3}, 0.48 mol of \ce{Cl2}, and 0.39 mol of \ce{PCl5} in an 8.00 L container.
        \end{enumerate}
        
        \subsection{Solution (a)}
            All of these will involve calculating Q, which depends on concentrations, each of which would be calculatable by dividing moles by volume when applicable.
            \begin{align}
                Q   &=  \frac{[\ce{PCl5}]}{[\ce{PCl3}][\ce{Cl2}]}
                    =   \frac{0.0139}{0.0593 \times 0.0381}
                    =   6.15
            \end{align}

            This is likely at equilibrium. 
            Measurements may be a little off.
            If this is indeed the value of Q at this point, the reaction would be going forward.

        \subsection{Solution (b)}
            \begin{equation}
                Q = \frac{0.0455}{0.0482 \times 0.289} = 3.27
            \end{equation}

            This reaction will definitely be going forwards.
        
        \subsection{Solution (c)}
            Well, here [\ce{Cl2}] will be zero, so Q will be $\infty$.
            What a way to involve calculus in Chemistry.
            In any case, $\infty$ is greater than 6.17, so the reaction will occur in reverse.

        \subsection{Solution (d)}
            \begin{align}
                Q   &=  \frac{\frac{n_{\ce{PCl5}}}{V}}{\frac{n_{\ce{PCl3}}}{V} \times \frac{n_{\ce{Cl2}}}{V}}
                    =   V \times \frac{n_{\ce{PCl5}}}{n_{\ce{PCl3}} \times n_{\ce{Cl2}}}\\
                    &=  8.00 \times \frac{0.39}{0.21 \times 0.48}
                    =   31.0
            \end{align}

            The reaction will be going in reverse.

    % \pagebreak
    \section{Topic G Problem 10}
        If Q, the reaction quotient, is greater than K for a reaction mixture, which of the following will happen?
        \begin{enumerate}[label=\alph*)]
            \item   The reaction will go in the direction that increases Q.
            \item   The reaction will go in the direction that decreases Q.
            \item   The reaction will go in the direction that increases K.
            \item   The reaction will go in the direction that decreases K.
        \end{enumerate}
        
        \subsection{Solution}
            Statement (b) is correct. 
            K does not change during a reaction (although chemical quantities and temperature changes may change it).
            Q will always try to approach K, so if Q is greater than K, Q will attempt to decrease.

    \pagebreak
    \section{Topic G Problem 11}
        For the reaction \ce{N2(g) + 3 H2(g) <-> 2 NH3(g)}, $K_p$ = 0.0489 at 256\unit{\celsius}. 
        For parts a through d, assume that the reaction is at this temperature.
        \begin{enumerate}[label=\alph*)]
            \item   An equilibrium mixture contains 0.100 atm of \ce{N2} and 0.200 atm of \ce{H2}. What is the partial pressure of \ce{NH3} in this mixture?
            \item   A second equilibrium mixture contains 830 torr of \ce{H2} and 42 torr of \ce{NH3}. What is the partial pressure of \ce{N2} in this mixture?
            \item   A third equilibrium mixture contains 0.0100 mol/L of \ce{N2} and 0.0300 mol/L of \ce{NH3}. What is the concentration of \ce{H2} in this mixture?
            \item   A fourth equilibrium mixture contains equal concentrations of all three chemicals. What is the pressure of each substance in this mixture?
        \end{enumerate}
        
        \subsection{Solution (a)}
            For all of this, we will be using an equation for $\rm K_p$. 
            \begin{equation}
                \rm K_p = \frac{P_{\ce{NH3}}^2}{P_{\ce{N2}} P_{\ce{H2}}^3}
            \end{equation}

            Solve for $P_{\ce{NH3}}$ and find the answer.
            \begin{equation}
                \rm 
                P_{\ce{NH3}} = \sqrt{K_p P_{\ce{N2}} P_{\ce{H2}}^3}
                    =   \sqrt{(0.0489) (0.100) (0.200)^3}
                    =   \boxed{0.00625\,\unit{\atm}}
            \end{equation}

        \subsection{Solution (b)}
            First convert from torr to atm.
            \begin{gather}
                P_{\ce{H2}} = 830\,\unit{\torr} \times \frac{1\,\unit{\atm}}{760\,\unit{\torr}} = 1.092\,\unit{\atm}\\
                P_{\ce{NH3}} = 42\,\unit{\torr} \times \frac{1\,\unit{\atm}}{760\,\unit{\torr}} = 0.05526\,\unit{\atm}
            \end{gather}

            Now we solve for \ce{N2}.
            \begin{equation}
                P_{\ce{N2}} =   \frac{P_{\ce{NH3}}^2}{K_p P_{\ce{H2}}^3}
                    =   \frac{0.05526^2}{0.0489 \times 1.092^3}
                    =   \boxed{0.048\,\unit{\torr}}
            \end{equation}

        \subsection{Solution (c)}
            First calculate $K_c$.
            Our value for $\Delta n$ is $-2$.
            The temperature is Kelvin is $529.15\,\unit{\kelvin}$.
            \begin{align}
                \rm
                K_c =   \frac{K_p}{(0.08206 * 529.15)^{-2}}
                    =   \frac{0.0489}{(0.08206 * 529.15)^{-2}}
                    =   \U{92.1}997
            \end{align}

            From here, find the concentration of \ce{H2}.
            \begin{equation}
                [\ce{H2}] = \sqrt[3]{\frac{[\ce{NH3}]^2}{K_c \times [\ce{N2}]}}
                    =   \sqrt[3]{\frac{0.0300^2}{\U{92.1}997 \times 0.0100}}
                    =   \boxed{0.0992\,\unit{\molarity}}
            \end{equation}

        \subsection{Solution (d)}
            Suppose that we had a concentration of $c$ of each substance.
            We can form an equation for $\rm K_c$ of each and use that to find what $K_p$ would be in this instance, then solve for $c$.
            \begin{gather}
                \rm
                K_c = \frac{c^2}{c \times c^3} = c^{-2}\\
                K_c = \frac{K_p}{(RT)^{\Delta n}} = c^{-2}\\
                c   =   \sqrt{\frac{(RT)^{\Delta n}}{K_p}}
                    =   \sqrt{\frac{(0.08206 * 529.15)^{-2}}{0.0489}}
                    =   0.\U{104}144\,\unit{\molarity}
            \end{gather}

            This can be used alongside the ideal gas law to find the pressure of all of them, which would be identical.
            Bear in mind that $\rm c = \frac{n}{V}$.
            \begin{align}
                P   &=  \frac{n}{V}RT
                    =   cRT\\
                    &=  0.\U{104}144\,\unit{\molarity} \times 0.08206 \times 529.15
                    =   \boxed{4.52\,\unit{\atm}}
            \end{align}
    \pagebreak
    \tableofcontents
\end{document}