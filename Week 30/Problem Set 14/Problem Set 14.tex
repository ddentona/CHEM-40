\documentclass[10pt]{article}
\usepackage[export]{adjustbox}
\usepackage{amsmath}
\usepackage{array}
\usepackage[makeroom]{cancel}
\usepackage{chemfig}
\usepackage{enumitem}
\usepackage{float}
\usepackage{graphicx}
%Load mhchem using some package options
\usepackage[version=4]{mhchem}
\usepackage{multicol}
\usepackage{siunitx}
\usepackage{wrapfig}

\title{
    Problem Set \#14
    \\  \small
    CHEM101A: General College Chemistry
    }
\author{Donald Aingworth}
\date{November 21, 2025}

\newcommand{\E}[1]{\times 10^{#1}}
\newcommand{\hc}{1.9864748\E{-25}\,\unit{\joule\,\meter}}
\newcommand{\U}[1]{\underline{#1}}

\begin{document}
    \DeclareSIUnit{\atm}{atm}
    \DeclareSIUnit{\molar}{M}
    \DeclareSIUnit{\M}{M}
    \DeclareSIUnit{\torr}{torr}

    \maketitle

    \setcounter{section}{11}
    \pagebreak
    \section{Topic G Problem 12}
        1.00 g of \ce{N2O4} is put into a 5.00 L container and heated to 50\unit{\celsius}. 
        At this temperature, the following reaction occurs and reaches equilibrium:
        \begin{center}
            \ce{N2O4(g) <=> 2 NO2(g)}
        \end{center}
        
        The concentration of \ce{NO2} in the equilibrium mixture is found to be equal to 6.68$\E{-4}$ M. 
        Calculate $\rm K_c$ and $\rm K_p$ for this reaction at 50\unit{\celsius}.
        
        \subsection{Solution}
            First convert grams to moles.
            \begin{gather}
                MM(\ce{N2O4}) = 92.02\,\unit{\gram/\mole}\\
                n(\ce{N2O4}) = \frac{m}{MM} = \frac{1.00\,\unit{\gram}}{92.02\,\unit{\gram/\mole}} = 0.0\U{108}672\,\unit{\mole}\\
                M(\ce{N2O4}) = \frac{n}{V} = \frac{0.0\U{217}344\,\unit{\mole}}{5.00\,\unit{\liter}} = 0.00\U{217}344\,\unit{\molar}
            \end{gather}

            Now, I'll use an ICE table.
            \begin{center}
                \begin{tabular}{|c| c c c |}
                    \hline
                    \unit{\molar}    & \ce{N2O4} & \ce{<=>}  & \ce{2 NO2}\\
                    \hline
                    I   & 0.00\U{217}344 && 0\\
                    C   & $-3.34\E{-4}$ && $6.68\E{-4}$\\
                    E   & $\U{18.3}944\E{-4}$ && $6.68\E{-4}$\\
                    \hline
                \end{tabular}
            \end{center}

            Now we calculate $\rm K_c$.
            \begin{gather}
                \rm
                K_c =   \frac{[\ce{NO2}]^2}{[\ce{N2O4}]}
                    =   \frac{(6.68\E{-4})^2}{\U{18.3}944\E{-4}}
                    =   \U{242}.5868\E{-6}\,\unit{\molar}
                    =   \boxed{243\E{-6}\,\unit{\molar}}
            \end{gather}

            Next we use that to find $\rm K_p$. 
            \begin{align}
                K_p &=  K_c (RT)^{\Delta n}
                    =   (\U{242}.5868\E{-6}\,\unit{\molar}) (0.08206 \times 323.15)^1\\
                    &=  \U{6.43}284\E{-3}\,\unit{\atm}
                    =   \boxed{6.43\E{-3}\,\unit{\atm}}
            \end{align}
    \pagebreak
    \section{Topic G Problem 13}
        When 0.100 mol of gaseous \ce{N2} and 0.100 mol of gaseous \ce{H2} are put into a 5.00 L container at 300\unit{\celsius}, the following reaction occurs and reaches equilibrium.
        \begin{center}
            \ce{N2(g) + 3 H2(g) <=> 2 NH3(g)}
        \end{center}
        
        The partial pressure of ammonia in the equilibrium mixture is 0.0506 atm. Calculate $\rm K_p$ and $\rm K_c$ for this reaction at 300\unit{\celsius}.

        \subsection{Solution}
            I'll use an ICE table.
            \begin{center}
                \begin{tabular}{|c| c c c c c |}
                    \hline
                    \unit{\molar}    & \ce{N2 (g)}   & + & \ce{3 H2 (g)} & \ce{<=>}  & \ce{2 NH3 (g)}\\
                    \hline
                    I   & 0.0200    && 0.0200   && 0\\
                    C   & $-x$      && $-3x$    && $2x$\\
                    E   & $0.0200 - x$ && $0.0200 - 3x$ && $2x$\\
                    \hline
                \end{tabular}
            \end{center}

            Now we solve for $x$, using the partial pressure of \ce{NH3}.
            \begin{gather}
                PV = nRT\\
                \begin{align}
                    [\ce{NH3}]  &=  \frac{n}{V}
                        =   \frac{P}{RT}
                        =   \frac{0.0506\,\unit{\atm}}{\left( 0.08206\,\unit{\frac{\atm\cdot\liter}{\mole\cdot\kelvin}} \right) (573\,\unit{\kelvin})}\\
                        &=  0.00\U{107}612\,\unit{\molar}
                \end{align}\\
                2x  =   [\ce{NH3}]\\
                x   =   \frac{[\ce{NH3}]}{2}
                    =   \frac{0.0\U{106}88\,\unit{\molar}}{2}
                    =   0.000\U{538}06\,\unit{\molar}
            \end{gather}

            This gives us the value of $x$, which we can use.
            \begin{center}
                \begin{tabular}{|c| c c c c c |}
                    \hline
                    \unit{\molar}    & \ce{N2 (g)}   & + & \ce{3 H2 (g)} & \ce{<=>}  & \ce{2 NH3 (g)}\\
                    \hline
                    E   & $0.0\U{194}62$ && $0.0\U{183}86$ && $0.00\U{107}612$\\
                    \hline
                \end{tabular}
            \end{center}

            This can be used to find $\rm K_c$.
            \begin{align}
                K_c &=  \frac{[\ce{NH3}]^2}{[\ce{N2}][\ce{H2}]^2}
                    =   \frac{0.00\U{107}612^2}{0.0\U{194}62 \times 0.0\U{183}86^3}
                    =   \U{9.57}353
                    =   \boxed{9.57}
            \end{align}

            Next convert it to pressure.
            \begin{align}
                K_p &=  K_c (RT)^{\Delta n}
                    =   \U{9.57}353 (\left( 0.08206\,\unit{\frac{\atm\cdot\liter}{\mole\cdot\kelvin}} \right) (573.15\,\unit{\kelvin}))^{-2}\\
                    &=  0.00\U{433}012
                    =   \boxed{0.00433}
            \end{align}

    \pagebreak
    \section{Topic G Problem 14}
        For the reaction below, $\rm K_c = 0.0168$ at 250\unit{\celsius}:
        \begin{center}
            \ce{PCl5(g) <=> PCl3(g) + Cl2(g)}
        \end{center}

        \begin{enumerate}[label=\alph*), nosep]
            \item   A flask contains 0.100 mol/L of \ce{PCl5}. 
                What will be the concentrations of all three gases when the above reaction reaches equilibrium?
            \item   A different flask contains 0.100 mol/L of \ce{PCl5}, 0.200 mol/L of \ce{PCl3}, and 0.300 mol/L of \ce{Cl2}. 
                What will be the concentrations of all three gases when the above reaction reaches equilibrium?
        \end{enumerate}

        \subsection{Solution (a)}
            Use an ICE table.
            \begin{center}
                \begin{tabular}{|c| c c c c c |}
                    \hline
                    \unit{\molar}    & \ce{PCl5 (g)}   & \ce{<=>} & \ce{PCl3 (g)} & +  & \ce{Cl2 (g)}\\
                    \hline
                    I   & 0.100 && 0    && 0\\
                    C   & $-x$  && $+x$ && $+x$\\
                    E   & $0.100 - x$ && $x$ && $x$\\
                    \hline
                \end{tabular}
            \end{center}

            Use $\rm K_c$ to find $x$.
            \begin{gather}
                K_c =   0.0168
                    =   \frac{[\ce{PCl3}][\ce{Cl2}]}{[\ce{PCl5}]}
                    =   \frac{x^2}{0.100 - x}\\
                x^2 =   0.00168 - 0.0168x\\
                0   =   x^2 + 0.0168x - 0.00168\\
                x   =   0.0\U{334}397 \text{ or } -0.0\U{502}397
            \end{gather}

            We use the postive one to complete the ICE table.
            That contains our answers.
            \begin{center}
                \begin{tabular}{|c| c c c c c |}
                    \hline
                    \unit{\molar}    & \ce{PCl5 (g)}   & \ce{<=>} & \ce{PCl3 (g)} & +  & \ce{Cl2 (g)}\\
                    \hline
                    E   & $0.0666$ && $0.0334$ && $0.0334$\\
                    \hline
                \end{tabular}
            \end{center}

        \subsection{Solution (b)}
            First find Q.
            That tells us where it will skew.
            \begin{equation}
                Q   =   \frac{[\ce{PCl3}][\ce{Cl2}]}{[\ce{PCl5}]}
                    =   \frac{0.200 * 0.300}{0.100}
                    =   0.6
            \end{equation}

            This is way bigger than $\rm K_c$.
            This means it skews way towards \ce{PCl5} (left).
            From here, use an ICE table. 
            \begin{center}
                \begin{tabular}{|c| c c c c c |}
                    \hline
                    \unit{\molar}    & \ce{PCl5 (g)}   & \ce{<=>} & \ce{PCl3 (g)} & +  & \ce{Cl2 (g)}\\
                    \hline
                    I   & 0.100 && 0.200    && 0.300\\
                    C   & $+x$  && $-x$     && $-x$\\
                    E   & $0.100 + x$ && $0.200 - x$ && $0.300 - x$\\
                    \hline
                \end{tabular}
            \end{center}

            Find $x$ using $\rm K_c$.
            \begin{gather}
                K_c = 0.0168 = \frac{(0.200 - x)(0.300 - x)}{0.100 + x}\\
                0.00168 + 0.0168x = x^2 - 0.500x + 0.0600\\
                0   =   x^2 - 0.5168x + 0.05832\\
                x   =   0.\U{350}327 \text{ or } 0.\U{166}473
            \end{gather}

            The former is too big, so we use the latter.
            \begin{center}
                \begin{tabular}{|c| c c c c c |}
                    \hline
                    \unit{\molar}    & \ce{PCl5 (g)}   & \ce{<=>} & \ce{PCl3 (g)} & +  & \ce{Cl2 (g)}\\
                    \hline
                    E   & $0.266$ && $0.0335$ && $0.134$\\
                    \hline
                \end{tabular}
            \end{center}

    \pagebreak
    \section{Topic G Problem 15}
        For the reaction below, $\rm K_p = 0.513$ at a certain temperature.
        \begin{center}
            \ce{H2(g) + I2(g) <=> 2 HI(g)}
        \end{center}
        
        \begin{enumerate}[label=\alph*), nosep]
            \item   A flask holds some gaseous \ce{HI} at this temperature and a pressure of 3.00 atm. 
                What will be the partial pressures of all three gases when the above reaction reaches equilibrium?
            \item   A second flask contains a mixture of the three gases with the following partial pressures: \ce{H2} = 0.433 atm, \ce{I2} = 0.0471 atm, \ce{HI} = 0.0310 atm. 
                What will be the partial pressures of all three gases when the above reaction reaches equilibrium?
        \end{enumerate}

        \subsection{Solution (a)}
            ICE table.
            \begin{center}
                \begin{tabular}{|c| c c c c c |}
                    \hline
                    \unit{\atm}    & \ce{H2 (g)} & + & \ce{I2 (g)} & \ce{<=>}  & \ce{2HI (g)}\\
                    \hline
                    I   & 0 && 0    && 3.00\\
                    C   & $+x$  && $+x$     && $-2x$\\
                    E   & $x$ && $x$ && $3.00 - 2x$\\
                    \hline
                \end{tabular}
            \end{center}

            Find $x$.
            \begin{gather}
                K_p = 0.513 = \frac{(3.00 - 2x)^2}{x^2}\\
                0.513x^2 = 9.00 - 12.0x + 4x^2\\
                0 = 9.00 - 12.0x + 3.487x^2\\
                x = \U{2.33}689 \text{ or } \U{1.10}447
            \end{gather}

            Use the latter.
            \begin{center}
                \begin{tabular}{|c| c c c c c |}
                    \hline
                    \unit{\atm} & \ce{H2 (g)} & + & \ce{I2 (g)} & \ce{<=>}  & \ce{2HI (g)}\\
                    \hline
                    E   & $1.10$ && $1.10$ && $0.791$\\
                    \hline
                \end{tabular}
            \end{center}
        
        \subsection{Solution (b)}
            Find Q.
            \begin{equation}
                Q = \frac{(0.0310)^2}{0.433 * 0.0471} = 0.0\U{471}21
            \end{equation}

            It skews hard to the right.
            ICE table.
            \begin{center}
                \begin{tabular}{|c| c c c c c |}
                    \hline
                    \unit{\atm}    & \ce{H2 (g)} & + & \ce{I2 (g)} & \ce{<=>}  & \ce{2HI (g)}\\
                    \hline
                    I   & 0.433 && 0.0471 && 0.0310\\
                    C   & $-x$  && $-x$     && $+2x$\\
                    E   & $0.433 - x$ && $0.0471 - x$ && $0.0310 + 2x$\\
                    \hline
                \end{tabular}
            \end{center}

            Find $x$.
            \begin{gather}
                K_p = 0.513 = \frac{(0.0310 + 2x)^2}{(0.0471 - x)(0.433 - x)}\\
                0.513x^2 - 0.2462913x + 0.0104622759 = 0.000961 + 1.24x + 4x^2\\
                0 = 3.487x^2 + 0.3702913x - 0.0095012759\\
                x = -0.\U{127}5537 \text{ or } 0.0\U{213}618
            \end{gather}

            Use the latter.
            \begin{center}
                \begin{tabular}{|c| c c c c c |}
                    \hline
                    \unit{\atm} & \ce{H2 (g)} & + & \ce{I2 (g)} & \ce{<=>}  & \ce{2HI (g)}\\
                    \hline
                    E   & $0.412$ && $0.0257$ && $0.0737$\\
                    \hline
                \end{tabular}
            \end{center}

    \pagebreak
    \section{Topic G Problem 16}
        Parts a through d of this problem relate to the reaction below:
        \begin{center}
            \ce{O2(g) + 2 F2(g) <=> 2 OF2(g)}
        \end{center}

        \begin{enumerate}[label=\alph*), nosep]
            \item   If you add some gaseous \ce{F2} to an equilibrium mixture of these three chemicals, which way will the reaction proceed?
            \item   If you add some gaseous \ce{O2} to an equilibrium mixture of these three chemicals, what will happen to the partial pressure of \ce{F2} in the mixture (i.e. will it go up, go down, or remain the same)?
            \item   If you increase the volume of the container, which way will the reaction proceed?
            \item   If you decrease the volume of the container, what will happen to the mass of \ce{OF2} in the mixture?
            \item   If you increase the temperature, which way will the reaction proceed? 
                You will need to look up the bond energy values to answer this question.
        \end{enumerate}

        \subsection{Solution}
            \begin{enumerate}[label=\alph*), nosep]
                \item   It will proceed forward (more than reverse) until it reaches equilibrium again.
                \item   The reaction will cause more \ce{OF2} to be created to balance towards chemical equilibrium. This will inevitably lead to a lowering in the moles of \ce{F2} and resultantly its partial pressure decreasing.
                \item   It will skew left (reverse reactions).
                \item   The reaction will skew right (forward reactions), so the mass of \ce{OF2} will increase.
                \item   The bond energy at 273\unit{\kelvin}\footnote{See 10.9 of the textbook} of \ce{F-F} is 155\,\unit{\kilo\joule/\mole}, \ce{O-O} is 142\,\unit{\kilo\joule/\mole}, and \ce{O-F} is 190\,\unit{\kilo\joule/\mole}. For this, one \ce{O-O} and two \ce{F-F} bonds are broken, so we would add those bond energies together and turn them negative. Meanwhile, four \ce{O-F} bonds are formed, which we can keep positive. Adding all these together, we get 308\,\unit{\kilo\joule/\mole} as the energy of the forward reaction. The increased temperature (added heat) would result in the reaction going forward more.
            \end{enumerate}

    % \pagebreak
    \section{Topic G Problem 17}
        Will the value of the equilibrium constant K change in any of the parts of problem 16? 
        If so, which parts?

        \subsection{Solution}
            Only part (e) will change the value of K. 

    \pagebreak
    \section{Topic G Problem 18}
        Consider an equilibrium mixture of ammonium chloride, ammonia, and hydrogen chloride:
        \begin{center}
            \ce{NH4Cl(s) <=> NH3(g) + HCl(g)}
        \end{center}
        \begin{enumerate}[label=\alph*), nosep]
            \item   If you add a little solid \ce{NH4Cl} to the mixture, what will happen to the mass of \ce{NH3}?
            \item   If you add a little gaseous \ce{NH3} to the mixture, what will happen to the mass of \ce{HCl}?
            \item   If you add a little gaseous \ce{HCl} to the mixture, what will happen to the mass of \ce{NH4Cl}?
        \end{enumerate}

        \subsection{Solution}
            \begin{enumerate}[label=\alph*/, nosep]
                \item   It will not change.
                \item   It will reduce.
                \item   It will increase.
            \end{enumerate}

    \pagebreak
    \section{Topic G Problem 19}
        The reaction below is allowed to reach equilibrium:
        \begin{center}
            \ce{NH3(aq) + H2O(l) <=> NH4+(aq) + OH-(aq)}
        \end{center}
        \begin{enumerate}[label=\alph*), nosep]
            \item   If you add a little 1 M \ce{HCl} to the mixture, which way will the reaction proceed? Or will it be unaffected? Explain your answer.
            \item   If you add a little 1 M \ce{MgCl2} to the mixture, which way will the reaction proceed? Or will it be unaffected? Explain your answer.
            \item   If you add a little 1 M \ce{NH4NO3} to the mixture, which way will the reaction proceed? Or will it be unaffected? Explain your answer.
        \end{enumerate}

        \subsection{Solution}
            \begin{enumerate}[label=\alph*), nosep]
                \item   The hydrogen in the \ce{HCl} will bond with the \ce{OH-}, which will reduce the amount of products. This will in turn cause the reaction to go \U{forward}.
                \item   The \ce{Mg^2+} of the \ce{MgCl2} will bond with the \ce{OH-} and form a precipitate, reducing the amount of \ce{OH-} and causing the reaction to go \U{forward}.
                \item   The \ce{NO3-} will not bond wth anything. The \ce{NH4+} will cause an imbalance, which will result in a \underline{reverse} reaction as a counterbalance.
            \end{enumerate}

    \pagebreak
    \section{Topic G Problem 20}
        For the reaction \ce{4 HBr(aq) + O2(g) <=> 2 Br2(aq) + 2 H2O(l)}, $\rm K_c = 6.7\E{10}$. 
        Use this information to calculate the equilibrium constant for each of the following reactions.
        \begin{enumerate}[label=\alph*), nosep]
            \item   \ce{2 HBr(aq) + \frac{1}{2} O2(g) <=> Br2(aq) + H2O(l)}
            \item   \ce{4 Br2(aq) + 4 H2O(l) <=> 8 HBr(aq) + 2 O2(g)}
        \end{enumerate}

        \subsection{Solution (a)}
            Take the square root of $\rm K_c$.
            \begin{equation}
                \rm \sqrt{K_c} = \sqrt{6.7\E{10}} = \boxed{2.6\E{5}}
            \end{equation}

        \subsection{Solution (b)}
            Square the reciprocal.
            \begin{equation}
                \rm 
                \left( \frac{1}{K_c} \right)^2 = \frac{1}{(6.7\E{10})^2} = \boxed{2.2\E{-22}}
            \end{equation}

    \pagebreak
    \section{Topic G Problem 21}
        Consider the following reactions, where the equilibrium constants are represented by the variables x and y:
        \begin{center}
            \ce{2 FeCl3(aq) + Zn(s) <=> 2 FeCl2(aq) + ZnCl2(aq)} K = $x$\\
            \ce{FeCl3(aq) + CrCl2(aq) <=> FeCl2(aq) + CrCl3(aq)} K = $y$
        \end{center}
        Write an expression for the equilibrium constant for the reaction below, in terms of x and y.
        \begin{center}
            \ce{2 CrCl3(aq) + Zn(s) <=> 2 CrCl2(aq) + ZnCl2(aq)}
        \end{center}

        \subsection{Solution}
            Let's make this simple.
            Start by reversing the second equation (reciprocal for K).
            \begin{align}
                &\ce{FeCl2(aq) + CrCl3(aq) <=> FeCl3(aq) + CrCl2(aq)} &K = \frac{1}{y}
            \end{align}

            Double that (square K).
            \begin{align}
                &\ce{2FeCl2(aq) + 2CrCl3(aq) <=> 2FeCl3(aq) + 2CrCl2(aq)} &K = \frac{1}{y^2}
            \end{align}

            Next add the first and new second together (multiply K-values).
            \begin{align}
                &\begin{matrix}
                    \cancel{\ce{2 FeCl3(aq)}} + \ce{Zn(s)}\\
                    + \cancel{\ce{2FeCl2(aq)}} + \ce{2CrCl3(aq)}
                \end{matrix}\ce{<=>}
                \begin{matrix}
                    \cancel{\ce{2 FeCl2(aq)}} + \ce{ZnCl2(aq)}\\
                    + \cancel{\ce{2FeCl3(aq)}} + \ce{2CrCl2(aq)}
                \end{matrix} &K = \frac{x}{y^2}\\
                &\ce{2 CrCl3(aq) + Zn(s) <=> 2 CrCl2(aq) + ZnCl2(aq)} &K = \boxed{\frac{x}{y^2}}
            \end{align}

    \pagebreak
    \section{Topic G Problem 22}
        Calculate the equilibrium constant for the following reaction:
        \begin{center}
            \ce{3 HClO3(aq) <=> 2 HClO4(aq) + HClO(aq)}
        \end{center}
        Use the following equilibrium constants.
        \begin{center}
            \ce{Cl2(aq) + H2O(l) <=> HCl(aq) + HClO(aq)} K = $6.2\E{-2}$\\
            \ce{3 HClO(aq) <=> HClO3(aq) + 2 HCl(aq)} K = $2.1\E{-6}$\\
            \ce{4 Cl2(aq) + 4 H2O(l) <=> 7 HCl(aq) + HClO4(aq)} K = $7.9\E{-8}$
        \end{center}


        \subsection{Solution}
            Start by reversing the equations necessary so chemicals in the equations are in the same side as they are in the final equation.
            Take the reciprocal of the K-values.
            Also invert the first one, since that would be the only way to counter the bottom equation.
            \begin{gather}
                \ce{HCl(aq) + HClO(aq) <=> Cl2(aq) + H2O(l)}; K = 1.6\E{1}\\
                \ce{HClO3(aq) + 2 HCl(aq) <=> 3 HClO(aq)}; K = 4.8\E{5}\\
                \ce{4 Cl2(aq) + 4 H2O(l) <=> 7 HCl(aq) + HClO4(aq)}; K = 7.9\E{-8}
            \end{gather}

            Multiply these by coefficients that are found in the final equation.
            Raise the K-values to the coefficient's power.
            \begin{gather}
                \ce{8 HCl(aq) + 8 HClO(aq) <=> 8 Cl2(aq) + 8 H2O(l)}; K = 4.6\E{9}\\
                \ce{3 HClO3(aq) + 6 HCl(aq) <=> 9 HClO(aq)}; K = 1.1\E{17}\\
                \ce{8 Cl2(aq) + 8 H2O(l) <=> 14 HCl(aq) + 2 HClO4(aq)}; K = 6.2\E{-15}
            \end{gather}

            Add the first and the last. 
            This is mostly a formality, since the next step can be combined with this one.
            Multiply K-values together.
            \begin{gather}
                \ce{8 HClO(aq) <=> 6 HCl(aq) + 2 HClO4(aq)}; K = 2.9\E{-5}\\
                \ce{3 HClO3(aq) + 6 HCl(aq) <=> 9 HClO(aq)}; K = 1.1\E{17}
            \end{gather}

            Do the same thing with the (new) first and the second.
            \begin{gather}
                \ce{3 HClO3(aq) <=> 2HClO4(aq) + HClO(aq)}; K = \boxed{3.1\E{12}}
            \end{gather}

            That will be all.
    \pagebreak
    \tableofcontents
\end{document}


% \begin{center}
%     \includegraphics[width=0.7\textwidth]{Answers Images/F23.jpg}
% \end{center}

% \begin{wrapfigure}{r}{0.25\textwidth}
%     \vspace{-30pt}
%     \includegraphics[width=0.25\textwidth]{img-F24.png}\\
%     \includegraphics[width=0.25\textwidth]{Answers Images/F24.jpg}
%     % \label{fig:wrapfig}
% \end{wrapfigure}