\documentclass[10pt]{article}
\usepackage{enumitem}
%Load mhchem using some package options
\usepackage[version=4]{mhchem}
\usepackage{multicol}
\usepackage{siunitx}

\title{
    Problem Set \#1
    \\  \small
    CHEM101A: General College Chemistry
    }
\author{Donald Aingworth IV}
\date{August 22, 2025}

\begin{document}
    \maketitle

    I will note that I saw the comment about picking the problems best for your own review.
    After looking through all the review problems, the first ten, the most Chemistry-heavy and least math-heavy \textit{were} the best problems for me for review. 
    These are the problems I determined to be best for review for me, as well as the most go-to problems.

    \pagebreak
    \section{Review Problem 1}
        Write the chemical formulas or symbols for each of the following ions.
        \begin{multicols}{3}
            \begin{enumerate}[label=\alph*)]
                \item   sodium ion 
                \item   oxide ion 
                \item   calcium ion 
                \item   iodide ion
                \item   iron(II) ion 
                \item   copper(I) ion 
                \item   hydroxide ion 
                \item   nitrate ion
                \item   sulfate ion 
                \item   phosphate ion 
                \item   carbonate ion 
                \item   ammonium ion
                \item   bicarbonate ion
            \end{enumerate}
        \end{multicols}

        \subsection{Solution}
            \begin{multicols}{4}
                \begin{enumerate}[label=\alph*)]
                    \item   \ce{Na+}
                    \item   \ce{O^{2-}}
                    \item   \ce{Ca^{2+}}
                    \item   \ce{I-}
                    \item   \ce{Fe^{2+}}
                    \item   \ce{Cu+}
                    \item   \ce{OH-}
                    \item   \ce{NO3-}
                    \item   \ce{SO4^{2-}}
                    \item   \ce{PO4^3-}
                    \item   \ce{CO3^2-}
                    \item   \ce{NH4-}
                    \item   \ce{HCO3-}
                \end{enumerate}
            \end{multicols}
    
    \pagebreak
    \section{Review Problem 2}
        Write chemical formulas for each of the following ionic compounds.
        \begin{multicols}{2}
            \begin{enumerate}[label=\alph*)]
                \item sodium sulfide 
                \item magnesium fluoride
                \item aluminum oxide
                \item iron(III) chloride 
                \item potassium sulfate
                \item aluminum nitrate
                \item ammonium phosphate
            \end{enumerate}
        \end{multicols}

        \subsection{Solution}
            \begin{multicols}{4}
                \begin{enumerate}[label=\alph*)]
                    \item   \ce{Na2S}
                    \item   \ce{MgF2}
                    \item   \ce{Al2O3}
                    \item   \ce{FeCl3}
                    \item   \ce{K2SO4}
                    \item   \ce{Al(NO3)3}
                    \item   \ce{(NH4)3PO4}
                \end{enumerate}
            \end{multicols}

    \pagebreak
    \section{Review Problem 3}
        a) What is the ion charge on each atom of \ce{V} in the compound \ce{V2S3}?

        \noindent 
        b) What is the ion charge on each \ce{P3O10} group in the compound \ce{Ca5(P3O10)2}?

        \subsection{Solution (a)}
            My general intuition is that if in a compound, one has three in quantity nad the other has two, then the former will have two in charge (regardless of positiveness/negativeness) and the other will have three in the opposite charge.
            Let's just figure out a different way to do that.

            \ce{V} is in group 5B, so it could probably have one of several charges. 
            Looking at \ce{S} (Sulfur), it is in group 6A, so it naturally has a charge of \ce{S^2-}.
            Multiply the -2 charge by the 3 sulfur ions to get a net charge of -6.
            Since \ce{V} is a cation and not an anion, the net anion charge would be +6.
            Divide that by the two \ce{V} (Vanadium) ions to get an indivdual charge of \ce{V^3+}
            In the end, it will be a net charge of \boxed{3+}.

        \subsection{Solution (b)}
            A Cadmium ion naturally has a charge of 2+.
            Multiply that by 5 cadmium ions to get a total cation charge of 10+.
            Turn that negative for anions.
            Divide that by two \ce{P3O10} ions to get a charge of \boxed{5-}.

    \pagebreak
    \section{Review Problem 4}
        Balance the following chemical equation: \ce{Cr2O3 + HBr -> CrBr3 + H2O}.

        \subsection{Solution}
            Let's look at the water (\ce{H2O}).
            The Chromium Oxide (\ce{Cr2O3}) requires there be three oxide (\ce{O^2-}) total.
            The coefficient we can put on the water can as such be 3.
            This means that there will be six Hydrogen ions (\ce{H+}) total.
            There will as such be 6 Hydrogen Bromide (\ce{HBr}).
            This results in six Bromide (\ce{Br-}).
            They can be used in the six Bromide for the Chromium Bromide (\ce{CrBr3}) to give chromium bromide a coefficient of 2.
            The resultant total Chromium ion (\ce{Cr^3+}) will have a total count of three.
            This lines up well with the single Chromium Oxide.
            \begin{center}
                \boxed{\ce{Cr2O3 + 6HBr -> 2CrBr3 + 3H2O}}
            \end{center}

            These are technically minumum coefficients, but all chemical reaction formulae are and that should not be a problem.
    
    \pagebreak
    \section{Review Problem 5}
        If you have exactly one mole of \ce{Cr(NO3)3}, how many grams of this compound do you have?

        \subsection{Solution}
            First find the molar mass of \ce{Cr(NO3)3}.
            \begin{align}
                MM(\ce{Cr(NO3)3})   &=  MM(\ce{Cr}) + 3 * MM(\ce{N}) + 9 * MM(\ce{O})\\
                    &=  52.00 \unit{\gram/\mol} + 3 * 14.01 \unit{\gram/\mol} + 9 * 16.00 \unit{\gram/\mol}\\
                    &=  52.00 \unit{\gram/\mol} + 42.03 \unit{\gram/\mol} + 144.00 \unit{\gram/\mol}\\
                    &=  238.03 \unit{\gram/\mol}
            \end{align}

            Now we multiply that by the number of moles.
            \begin{equation}
                238.03 \unit{\gram/\mol} * 1 \unit{\mol}    =   \boxed{238.03 \unit{\gram}}
            \end{equation}
    
    \pagebreak
    \section{Review Problem 6}
        Convert each of the following to moles:
        \begin{enumerate}[label=\alph*)]
            \item 6.131 g of \ce{N} 
            \item 6.131 g of \ce{N2} 
            \item 6.131 g of \ce{N2O}
        \end{enumerate}

        \subsection{Solution (a)}
            The molar mass of Nitrogen (\ce{N}) is 14.01 \unit{\gram/\mol}.
            Divide the mass by the molar mass to get the number of moles.
            \begin{equation}
                \frac{6.131 \unit{\gram}}{14.01 \unit{\gram/\mol}} = 0.437615 \unit{\mol} \approx \boxed{0.4376 \unit{\gram}}
            \end{equation}

        \subsection{Solution (b)}
            The easy answer to this would be to divide the solution from part (a) by two since there would be twice the molar mass and as such half the moles.
            To be nice, I'll do it the traditional way instead.
            The molar mass of Nitrogen gas (\ce{N2}) is 28.02 \unit{\gram/\mol}.
            \begin{equation}
                \frac{6.131 \unit{\gram}}{28.02 \unit{\gram/\mol}} = 0.218808 \unit{\mol}   \approx \boxed{0.2188 \unit{\gram}}
            \end{equation} 

        \subsection{Solution (c)}
            Add the molar mass of oxygen (MM(\ce{O}) = 16.00 \unit{\gram/\mol}) to the molar mass of Nitrogen Gas (\ce{N2} = 28.02 \unit{\gram/\mol}) to get a total of 44.02 \unit{\gram/\mol}.
            I think you know the next step.
            \begin{equation}
                \frac{6.131 \unit{\gram}}{44.02 \unit{\gram/\mol}} = 0.139277 \unit{\mol} \approx   \boxed{0.1393 \unit{\gram}}
            \end{equation}
    
    \pagebreak
    \section{Review Problem 7}
        a) How many \ce{N2O} molecules are there in 6.131 g of \ce{N2O}? \textit{(Reminder: Avogadro's constant is $6.022 \times 10^{23}$ \unit{\mol^{-1}}.)} 
        
        \noindent
        b) How many nitrogen atoms are there in 6.131 g of \ce{N2O}?

        \subsection{Solution (a)}
            Review Problem 6(c) gave us that 6.131g of \ce{N2O} contains 0.1393 \unit{\mol} of \ce{N2O}.
            Multiply this by Avogadro's number to get the total number of atoms.
            I like the way you wrote the units of Avogadro's number as \unit{\mol^{-1}}. 
            \begin{equation}
                0.1393 \unit{\mol} * 6.022 \times 10^{23} \unit{\mol^{-1}} = \boxed{8.387 \times 10^{22}}
            \end{equation}

            I'm a real purist in terms of units (probably because of a heavy Physics background), but you could say that the units there is in terms of molecules.
        
        \subsection{Solution (b)}
            Since the ratio of \ce{N}:\ce{N2O} is 2:1, we multiply the number of molecules by 2 to get the number of Nitrogen molecules.
            \begin{equation}
                8.387 \times 10^{22} * 2 = \boxed{1.677 \times 10^{23}}
            \end{equation}

            As previously stated, I'm a units purist.
    
    \pagebreak
    \section{Review Problem 8}
        In 0.08157 moles of \ce{Al(ClO4)3}, there are…
        
        a) how many moles of \ce{Al^3+} ions?
        
        b) how many moles of \ce{ClO4-} ions?
        
        c) how many moles of oxygen atoms?

        \subsection{Solution (a)}
            Multiply the number of moles by the number of \ce{Al^3+} in each atom (1).
            \begin{equation}
                0.08157 \unit{\mol} * 1 \ce{Al^3+} = \boxed{0.08157 \unit{\mol} \ce{Al^3+}}
            \end{equation}

        \subsection{Solution (b)}
            Multiply the number of moles by the number of \ce{ClO4-} in each atom (3).
            \begin{equation}
                0.08157 \unit{\mol} * 3 \ce{ClO4-} = 0.24471 \unit{\mol} \ce{ClO4-} \approx \boxed{0.2447 \unit{\mol} \ce{ClO4-}}
            \end{equation}

        \subsection{Solution (c)}
            Multiply the number of moles of \ce{Al(ClO4)3} by the number of \ce{O} atoms ($ 3 \ce{ClO4} * 4 \ce{\frac{O}{ClO4}} = 12 \ce{O} $).
            \begin{equation}
                0.08157 \unit{\mol} * 12 \ce{O} = 0.97884 \unit{\mol} \ce{O} \approx \boxed{0.9788 \unit{\mol} \ce{O}}
            \end{equation}
    
    \pagebreak
    \section{Review Problem 9}
        If you have 285 atoms of fluorine…
        
        a) How many moles of fluorine atoms do you have?
        
        b) What is the mass of the fluorine atoms, in grams?

        \subsection{Solution (a)}
            Since each mole contains $6.022 \times 10^{23}$ atoms, we divide our atoms by avogadro's number.
            \begin{align}
                n   &=  \frac{285}{6.022 \times 10^{23}} \unit{\mol}
                    =   \boxed{4.73 \times 10^{-22} \unit{\mol}}
            \end{align}

        \subsection{Solution (b)}
            Fluorine (\ce{F}) has a molar mass of $19.00 \unit{\gram/\mol}$.
            Multiply molar mass by the number of moles to get the mass.
            \begin{align}
                m   &=  n * MM(\ce{F})
                    =   4.73 \times 10^{-22} \unit{\mol} * 19.00 \unit{\gram/\mol}\\
                    &=  \boxed{8.99 \times 10^{-21} \unit{\gram}}
            \end{align}

    \pagebreak
    \section{Review Problem 10}
        \ce{KClO3} breaks down when it is heated. The chemical equation for this reaction is:
        \begin{center}
            \ce{2 KClO3 -> 2 KCl + 3 O2}
        \end{center}
        a) If 5.000 g of \ce{KClO3} breaks down, how many grams of \ce{O2} will be formed?
        
        \noindent
        b) If 2.000 g of \ce{O2} are formed, how many grams of \ce{KCl} will also be formed?

        \subsection{Solution (a)}
            I have a simpler solution, but I'll just do the classic thing.
            Convert grams to moles.
            \begin{align}
                MM(\ce{KClO3})  &=  MM(\ce{K}) + MM(\ce{Cl}) + 3 * MM(\ce{O})\\
                    &=  39.10 \unit{\gram/\mole} + 35.45 \unit{\gram/\mole} + 3 * 16.00 \unit{\gram/\mole}\\
                    &=  122.55 \unit{\gram/\mole}\\
                n(\ce{KClO3})   &=  \frac{m}{MM(\ce{KClO3})}
                    =   \frac{5.000 \unit{\gram} \ce{KClO3}}{122.55 \unit{\gram/\mole}}\\
                    &=  0.04080 \unit{\mole}
            \end{align}

            Then multiply it by $\frac{3 \ce{O2}}{2 \ce{KClO2}}$ from the chemical equation.
            \begin{align}
                n(\ce{O2})  &=  n(\ce{KClO3}) * \frac{3 \ce{O2}}{2 \ce{KClO2}}
                    =   0.04080 \unit{\mole} \ce{KClO2} * \frac{3 \ce{O2}}{2 \ce{KClO2}}\\
                    &=  0.6120 \unit{\mole} \ce{O2}
            \end{align}

            Multiply that by the molar mass of \ce{O2} ($32.00 \unit{\gram/\mol}$).
            \begin{align}
                m(\ce{O2})  &=  0.6120 \unit{\mole} \ce{O2} * 32.00 \unit{\gram/\mol}
                    =   \boxed{1.958 \unit{\gram}}
            \end{align}

        \subsection{Solution (b)}
            Use the same strategy as the part (a).
            \begin{align}
                MM(\ce{O2}) &=  32.00 \unit{\gram/\mole}\\
                n(\ce{O2})  &=  \frac{m}{MM(\ce{O2})}
                    =   \frac{2.000 \unit{\gram}}{32.00 \unit{\gram/\mole}}
                    =   0.0625 \unit{\mole}\\
                n(\ce{KCl}) &=  n(\ce{O2}) * \frac{2 \ce{KCl}}{3 \ce{O2}}
                    =   0.0625 \unit{\mole} \ce{O2} * \frac{2 \ce{KCl}}{3\ce{O2}}
                    =   0.04167 \unit{\mole} \ce{KCl}\\
                MM(KCl) &=  MM(\ce{K}) + MM(\ce{Cl})
                    =   39.10 \unit{\gram/\mole} + 35.45 \unit{\gram/\mole}\\
                    &=  74.55 \unit{\gram/\mole}\\
                m(\ce{KCl}) &=  0.04167 \unit{\mole} \ce{KCl} * 74.55 \unit{\gram/\mole}
                    =   \boxed{3.106 \unit{\gram}}
            \end{align}

    \pagebreak
    \section{Review Problem 11}
        Write each of the following numbers using scientific notation:\\
        a) 3,710,000 \\
        b) 0.0000042 \\
        c) 3
        \subsection{Solution}
            \begin{enumerate}[label=\alph*)]
                \item $3.71 \times 10^6$
                \item $4.2 \times 10^{-6}$
                \item $3.0 \times 10^0$
            \end{enumerate}
    \section{Review Problem 12}
        Write each of the following numbers in standard (decimal) form:\\
        a) $6.3 \times 10^5$ \\
        b) $4.0760 \times 10^{-2}$

        \subsection{Solution}
            \begin{enumerate}[label=\alph*)]
                \item $630000$
                \item $0.040760$
            \end{enumerate}

    \pagebreak
    \section{Review Problem 13}
        Round the number 81.03974 in each of the following ways:
        \begin{multicols}{2}
            \begin{enumerate}[label=\alph*)]
                \item   to one significant figure 
                \item   to two significant figures
                \item   to three significant figures 
                \item   to four significant figures
                \item   to five significant figures
            \end{enumerate}
        \end{multicols}

        \subsection{Solution}
            \begin{enumerate}[label=\alph*)]
                \item $80$
                \item $81$
                \item $81.0$
                \item $81.04$
                \item $81.040$
            \end{enumerate}

    \pagebreak
    \section{Topic A Problem 1}
        A sample of magnesium phosphate contains $x$ moles of \ce{Mg3(PO4)2}.
        Express each of the following quantities in terms of $x$.

        a) The number of moles of phosphate ions in this sample
        
        b) The number of moles of oxygen atoms in this sample

        c) The number of phosphorus atoms in this sample

        d) The mass of the sample, in grams

        e) The number of grams of magnesium in this sample

        \subsection{Solution (a)}
            There are two phosphate (\ce{PO4^3-}) ions in each mole of \ce{Mg3(PO4)2} (magnesium phosphate). 
            Moles have the same ratio as the individual parts themselves. 
            As such, there would have to be \boxed{2x\ moles} of phosphate ions.

        \subsection{Solution (b)}
            With four oxygen atoms per phosphate ion, there are \boxed{8x\ moles} of oxygen atoms.

        \subsection{Solution (c)}
            There is one phosphorus atom per phosphate ion, so 2x moles of phosphorus atoms.
            Multiply this by Anogadro's Number ($6.022 \times 10^{23} \unit{\mol^{-1}}$) to get the total number of atoms.
            \begin{equation}
                2x \unit{\mole} * 6.022 \times 10^{23} \unit{\mol^{-1}}
                    =   \boxed{1.204 \times 10^{24} * x}
            \end{equation}

        \subsection{Solution (d)}
            Multiply the molar mass by the number of moles.
            \begin{align}
                m   &=  x * MM(\ce{Mg3(PO4)2})\\
                    &=  x \left( 3*MM(\ce{Mg}) + 2 * MM(\ce{P}) + 8 * MM(\ce{O}) \right)\\
                    &=  x \left( 3 * 24.31 \unit{\gram/\mol} + 2 * 30.97 \unit{\gram/\mol} + 8 * 16.00 \unit{\gram/\mol} \right)\\
                    &=  x \left( 72.93 \unit{\gram/\mol} + 61.94 \unit{\gram/mol} + 128.0 \unit{\gram/\meter} \right)\\
                    &=  \boxed{262.87x\ \unit{\gram}
                    \approx 262.9x\ \unit{\gram}}
            \end{align}

        \subsection{Solution (e)}
            We know there are $3x$ moles of \ce{Mg} ions, using the strategy from part (a).
            Multiply this by the molar mass of \ce{Mg} (24.31 \unit{\gram/\mole}) to get the total magnesium mass. 
            \begin{equation}
                MM(\ce{Mg}) * n(\ce{Mg})    =   24.31 \unit{\gram/\mole} * 3x \unit{\mole}
                    =   \boxed{72.93x\ \unit{\gram}}
            \end{equation}


    \pagebreak
    \section{Topic A Problem 2}
        You have x grams of \ce{Na2Cr2O7}. 
        Express each of the following quantities in terms of x.

        a) The number of moles of \ce{Na2Cr2O7}

        b) The number of moles of \ce{O}

        c) The number of grams of \ce{O}

        d) The number of \ce{O} atoms

        \subsection{Solution (a)}
            Divide the mass by the molar mass to get the mole count.
            \begin{align}
                MM(\ce{Na2Cr2O7})   &=  2 * MM(\ce{Na}) + 2 * MM(\ce{Cr}) + 7 * MM(\ce{O})\\
                    &=  2 * 22.99 \unit{\gram/\mole} + 2 * 52.00 \unit{\gram/\mole} + 7 * 16.00 \unit{\gram/\mole}\\
                    &=  45.98 \unit{\gram/\mole} + 104.0 \unit{\gram/\mole} + 112.0 \unit{\gram/\mole}\\
                    &=  261.98 \unit{\gram/\mole} \approx 262.0 \unit{\gram/\mole}\\
                n   &=  \frac{m}{MM}
                    =   \frac{x\ \unit{\gram}}{\underbar{261.9}8 \unit{\gram/\mole}}\\
                    &=  \boxed{0.0038171x\ \unit{\mole}
                    \approx 0.00382x\ \unit{\mole}}
            \end{align}

        \subsection{Solution (b)}
            There are 7 \ce{O} for each \ce{Na2Cr2O7}.
            Multiply the number of moles of \ce{Na2Cr2O7} by that ratio to get the number of moles of \ce{O}.
            \begin{align}
                n(\ce{O})   &=  n(\ce{Na2Cr2O7}) * \frac{7 \ce{O}}{1 \ce{Na2Cr2O7}}\\
                    &=  0.0038171x\ \unit{\mole}\ \ce{Na2Cr2O7} * \frac{7 \ce{O}}{1 \ce{Na2Cr2O7}}\\
                    &=  0.0267195x\ \unit{\mole}\ \ce{O}
                    \approx \boxed{0.026720x\ \unit{\mole}\ \ce{O}}
            \end{align}
        
        \subsection{Solution (c)}
            The molar mass of Oxygen is 16.00 \unit{\gram/\mole}.
            Multiply this by the number of moles to get the grams per mole.
            \begin{align}
                m(\ce{O})   &=  n(\ce{O}) * MM(\ce{O})
                    =   0.0267195x\ \unit{\mole}\ \ce{O} * 16.00 \unit{\gram/\mole}\\
                    &=  \boxed{0.42751x\ \unit{\gram}\ \ce{O}
                    \approx 0.428x\ \unit{\gram}\ \ce{O}}
            \end{align}

        \subsection{Solution (d)}
            Multiply the number of moles by Avogadro's Number.
            \begin{align}
                n(\ce{O}) * N_A &=  0.0267195x\ \unit{\mole}\ \ce{O} * 6.022 \times 10^{23}\\
                    &=  \boxed{1.6091 \times 10^{22}\ \ce{O}
                    \approx 1.61 \times 10^{22}\ \ce{O}}
            \end{align}

    \pagebreak
    \section{Topic A Problem 3}
        You have a sample of \ce{Al2(CO3)3} that contains x aluminum atoms. 
        How many oxygen atoms does it contain?

        \subsection{Solution}
            We can set up a couple ratios within \ce{Al2(CO3)3} and use the transistive property (ish) to get the third ratio.
            \begin{gather}
                2\ \ce{Al} : 3\ \ce{CO3}\\
                1\ \ce{CO3} : 3\ \ce{O}\\
                2\ \ce{Al} : 9\ \ce{O}
            \end{gather}

            Since our sample contains x \ce{Al} atoms, we can multiply that by our ratio.
            \begin{equation}
                x\ \ce{Al} * \frac{9\ \ce{O}}{2\ \ce{Al}} = \boxed{\frac{9}{2}x\ \ce{O} = 4.5x\ \ce{O}}
            \end{equation}

    \pagebreak
    \section{Topic A Problem 4}
        For each of the following questions (parts a and b), tell which box contains more atoms. 
        In each case, you may assume that x represents the same number.

        a) Box 1: x grams of \ce{Na}    Box 2: x grams of \ce{Mg}

        b) Box 1: x grams of \ce{O2}    Box 2: x grams of \ce{O3}

        \subsection{Solution (a)}
            Since each mole contains the same number of atoms and each compound we are comparing contains 1 atom (\ce{Na} and \ce{Mg} respectively), the sample with more moles will contain more atoms.
            We can set up an equation for the number of moles from the mass.
            \begin{equation}
                n = \frac{m}{MM}
            \end{equation}

            This leaves the number of moles (and as such the number of atoms) inversely proportional to the molar mass. 
            The molar mass of sodium (\ce{Na}) is 22.99 \unit{\gram/\mole}, while that of magnesium (\ce{Mg}) is 24.31 \unit{\gram/\mole}.
            Since \ce{Na} has a lower molar mass, that means there are more moles per gram.
            For an equal mass, that means there are more moles of \ce{Na} than \ce{Mg}, and as such there are more \ce{Na} atoms, located in box \boxed{1}. 

        \subsection{Solution (b)}
            Mass is equal.
            Let's make an equation.
            Suppose that $k$ is the number of each atom in a compound.
            \begin{gather}
                n   =   k * \frac{m}{MM}\\
                n_{\ce{O2}} =   2 * \frac{m}{2 * MM(\ce{O})} = \frac{m}{MM(\ce{O})}\\
                n_{\ce{O3}} =   3 * \frac{m}{3 * MM(\ce{O})} = \frac{m}{MM(\ce{O})}
            \end{gather}

            Apply transistivity.
            \begin{gather}
                n_{\ce{O3}}   =   n_{\ce{O2}}
            \end{gather}

            This means that the number of atoms in each box is \boxed{equal}.


    \pagebreak
    \section{Topic A Problem 5}
        In problem 4 part b, which box contains more molecules?
        
        \subsection{Solution}
            This is a similar problem to A.5.b.
            The only major difference is that we remove the $k$.
            \begin{gather}
                n   =   \frac{m}{MM}\\
                n_{\ce{O2}} =   \frac{m}{2 * MM(\ce{O})} = \frac{1}{2} * \frac{m}{MM(\ce{O})}\\
                n_{\ce{O3}} =   \frac{m}{3 * MM(\ce{O})} = \frac{1}{3} * \frac{m}{MM(\ce{O})}
            \end{gather}

            Since $\frac{1}{2} > \frac{1}{3}$, the one multiplied by $\frac{1}{2}$ is grater, which is the box with \ce{O2}, which is box \boxed{1}.


    \pagebreak
    \section{Topic A Problem 6}
        0.03774 moles of a mystery element weighs 7.363 grams. What element is this? 
            
        \subsection{Solution}
            We can take a known equation and solve for the molar mass.
            \begin{gather}
                n   =   \frac{m}{MM}\\
                MM  =   \frac{m}{n}
            \end{gather}

            We can plug in numbers to find the molar mass.
            \begin{align}
                MM  &=  \frac{7.363 \unit{\gram}}{0.03774 \unit{\mole}}
                    =   195.1 \unit{\gram/\mole}
            \end{align}

            The element that meets this molar mass requirement is \boxed{Platinum\ (\ce{Pt})}. 

    \pagebreak
    \section{Topic A Problem 7}
        A compound contains 31.89\% carbon, 5.35\% hydrogen, and 62.76\% chlorine. 
        What is the empirical formula of this compound?
        
        \subsection{Solution}
            Suppose that we have 100 \unit{\gram} of the compound.
            In that scenario, we would have 31.89\unit{\gram} of carbon (\ce{C}), 5.35\unit{\gram} of hydrogen (\ce{H}), and 62.76\unit{\gram} of carbon (\ce{Cl}). 
            Each of them has a molar mass, which we can divide the repetive masses by to get the mole counts.
            \begin{align}
                n_{\ce{C}}  &=  \frac{m}{MM(\ce{C})}
                    =   \frac{31.89\unit{\gram}}{12.01 \unit{\gram/\mole}}
                    =   2.655 \unit{\mole}\\
                n_{\ce{H}}  &=  \frac{m}{MM(\ce{H})}
                    =   \frac{5.35\unit{\gram}}{1.008 \unit{\gram/\mole}}
                    =   5.31 \unit{\mole}\\
                n_{\ce{Cl}}  &=  \frac{m}{MM(\ce{Cl})}
                    =   \frac{62.76\unit{\gram}}{35.45 \unit{\gram/\mole}}
                    =   1.770 \unit{\mole}
            \end{align}

            Now, we first divide the mole count of the carbon by chlorine to get a ratio between the two.
            We can follow that up by doing the same between hydrogen and chlorine.
            \begin{align}
                \frac{n_{\ce{C}}}{n_{\ce{Cl}}}  &\approx    1.5 =   \frac{3}{2}\\
                \frac{n_{\ce{H}}}{n_{\ce{Cl}}}  &\approx    3.0 =   \frac{6}{2}
            \end{align}

            Ths gives us a fnal ratio (and as such empirical formula) of \boxed{\ce{C3H6Cl2}}.

    \pagebreak
    \section{Topic A Problem 8}
        10.000 g of boron (B) combines with hydrogen to form 11.554 g of a pure compound. 
        What is the empirical formula of this compound?
            
        \subsection{Solution}
            First convert masses to moles, calling the compound \ce{X} for the moment.
            \begin{align}
                n(\ce{B})   &=  \frac{m(\ce{B})}{MM(\ce{B})}
                    =   \frac{10.000 \unit{\gram}}{10.81 \unit{\gram/\mole}}
                    =   0.925 \unit{\mole}\\
                n(\ce{H})   &=  \frac{m(\ce{X}) - m(\ce{B})}{MM(\ce{H})}
                    =   \frac{1.554 \unit{\gram}}{1.008 \unit{\gram/\mole}}
                    =   1.542 \unit{\mole}
            \end{align}

            There's a ratio of Boron to Hydrogen here, which we can use to find the emprical formula.
            \begin{align}
                \frac{n(\ce{B})}{n(\ce{H})} &=  \frac{0.925}{1.542} \approx 0.6 = \frac{3}{5}
            \end{align}

            This gives us our fial empirical formula of \boxed{\ce{B3H5}}

    \pagebreak
    \section{Topic A Problem 9}
        The compound in problem 8 is known to have a molar mass between 60 and 80 g/mol. 
        What is the molecular formula of this compound?
            
        \subsection{Solution}
            The coefficients of the molecular formula is a certain constant times the coefficients we got in Problem 8.
            First, we should find the molar mass of the compound in problem 8.
            \begin{align}
                MM(\ce{B3H5})   &=  3 * MM(\ce{B}) + 5 * MM(\ce{H})
                    =   32.43 + 5.04
                    =   37.47 \unit{\gram/\mole}
            \end{align}

            There is a discrete multiple of this that is between 60 and 80, the multiplier of which which would be our ``coefficients' coefficient''.
            That multiplier in question would be 2.
            Multiplying our earlier values, we have 6 boron and 10 hydrogen.
            This solidifies the answer as \boxed{\ce{B6H10}}.

    \pagebreak
    \section{Topic A Problem 10}
        A group 2A element combines with iodine to form a compound that contains 64.9\% I. 
        Which group 2A element is this? 

        \subsection{Solution}
            Since group 2A elements tend to have a net charge of -2 and iodine tends to have a net charge of +1, our formula for the compund with a group 2A element \ce{J} would have a formula \ce{I2J}.
            We can assume there to be 100\unit{\gram} of the cmpound, and as such 64.9\unit{\gram} \ce{I} and 35.1\unit{\gram} \ce{J}.
            We can calculate the number of moles of Iodine.
            \begin{align}
                n(\ce{I})   &=  \frac{m}{MM(\ce{I})}
                    =   \frac{64.9\unit{\gram}}{126.9 \unit{\gram/\mole}}
                    =   0.5114 \unit{\mole}
            \end{align}

            We can then calculate the number of moles of \ce{J}.
            \begin{align}
                n(\ce{J})   &=  \frac{1}{2} * n(\ce{I})
                    =   \frac{0.5114\unit{\mole}}{2} \ce{J}
                    =   0.2557 \unit{\mole}\ \ce{J}
            \end{align}

            We can use this and earlier information to find the molar mass.
            \begin{align}
                MM(\ce{J})  &=  \frac{m(\ce{J})}{n(\ce{J})}
                    =   \frac{35.1 \unit{\gram}}{0.2557 \unit{\mole}}
                    =   137.3 \unit{\gram/\mole}
            \end{align}

            This in turn is the molar mass of \boxed{Barrium\ (\ce{Ba})}.

    \pagebreak
    \section{Topic A Problem 11}
        A chemist has just discovered a new compound, called dunlinol. 
        A 1.9747 g sample of dunlinol is subjected to combustion analysis, producing 3.8602 g of \ce{CO2} and 0.3951 g of \ce{H2O} as the only products. 
        What is the empirical formula of dunlinol?

        \subsection{Solution}
            The molar mass of \ce{CO2} is 44.01\unit{\gram/\mole}. 
            The molar mass of \ce{H2O} is 18.016\unit{\gram/\mole}.
            This can be used to find the number of moles of both Carbon and Hydrogen.
            \begin{align}
                n(\ce{C})   &=  n(\ce{CO2})
                    =   \frac{m(\ce{CO2})}{MM(\ce{CO2})}
                    =   \frac{3.8602 \unit{\gram}}{44.01 \unit{\gram/\mole}}
                    =   0.08771 \unit{\mole}\\
                n(\ce{H})   &=  2 * n(\ce{H2O})
                    =   2 * \frac{m(\ce{H2O})}{MM(\ce{H2O})}
                    =   2 * \frac{0.3951 \unit{\gram}}{18.016 \unit{\gram/\mole}}
                    =   0.04386 \unit{\mole}
            \end{align}

            This gives us an empirical ratio of Carbon to Hydrogen of 2:1.
            We can then check the molar mass of this.
            \begin{align}
                MM(\ce{C2H})    &=  2 * MM(\ce{C}) + MM(\ce{H})
                    =   2 * 12.01 + 1.008
                    =   25.018 \unit{\gram/\mole}
            \end{align}

            We can multiply the molar mass of \ce{C2H} (25.018 \unit{\gram/\mole}) by the number of moles of \ce{C2H} (which is the number of moles of hydrogen) to get the mass of that amount of this compound.
            \begin{equation}
                m   =   25.018 * 0.04386    =   1.0973 \unit{\gram}
            \end{equation}

            This is less than the initial mass given but twice this is more than that.
            We can calculate the difference, which would be mass made of Oxygen atoms, and then calculate the number of moles of Oxygen from there.
            \begin{align}
                m(\ce{O})   &=  1.9747 \unit{\gram} - 1.0973 \unit{\gram}
                    =   0.8774 \unit{\gram}\\
                n(\ce{O})   &=  \frac{m}{MM(\ce{O})}
                    =   \frac{0.8774\unit{\gram}}{16.00\unit{\gram/\mole}}
                    =   0.5484 \unit{\mole}
            \end{align}

            This gives us a ratio of Oxygen to Hydrogen of roughly 5:4.
            We can extend this to a ratio of Carbon to Hydrogen to Oxygen of 8:4:5.
            Our final formula would as such be \boxed{\ce{C8H4O5}}

\end{document}