\documentclass[10pt]{article}
\usepackage{enumitem}
%Load mhchem using some package options
\usepackage[version=4,arrows=pgf-filled,
textfontname=sffamily,
mathfontname=mathsf]{mhchem}
\usepackage{multicol}
\usepackage{siunitx}

\begin{document}
    \section{Review Problem 1}
        Write the chemical formulas or symbols for each of the following ions.
        \begin{multicols}{3}
            \begin{enumerate}[label=\alph*)]
                \item   sodium ion 
                \item   oxide ion 
                \item   calcium ion 
                \item   iodide ion
                \item   iron(II) ion 
                \item   copper(I) ion 
                \item   hydroxide ion 
                \item   nitrate ion
                \item   sulfate ion 
                \item   phosphate ion 
                \item   carbonate ion 
                \item   ammonium ion
                \item   bicarbonate ion
            \end{enumerate}
        \end{multicols}

        \subsection{Solution}
            \begin{multicols}{4}
                \begin{enumerate}[label=\alph*)]
                    \item   \ce{Na+}
                    \item   \ce{O^{2-}}
                    \item   \ce{Ca^{2+}}
                    \item   \ce{I-}
                    \item   \ce{Fe^{2+}}
                    \item   \ce{Cu+}
                    \item   \ce{OH-}
                    \item   \ce{NO3-}
                    \item   \ce{SO4^{2-}}
                    \item   \ce{PO4^3-}
                    \item   \ce{CO3^2-}
                    \item   \ce{NH4-}
                    \item   \ce{HCO3-}
                \end{enumerate}
            \end{multicols}
    
    \pagebreak
    \section{Review Problem 2}
        Write chemical formulas for each of the following ionic compounds.
        \begin{multicols}{2}
            \begin{enumerate}[label=\alph*)]
                \item sodium sulfide 
                \item magnesium fluoride
                \item aluminum oxide
                \item iron(III) chloride 
                \item potassium sulfate
                \item aluminum nitrate
                \item ammonium phosphate
            \end{enumerate}
        \end{multicols}

        \subsection{Solution}
            \begin{multicols}{4}
                \begin{enumerate}[label=\alph*)]
                    \item   \ce{Na2S}
                    \item   \ce{MgF2}
                    \item   \ce{Al2O3}
                    \item   \ce{FeCl3}
                    \item   \ce{K2SO4}
                    \item   \ce{Al(NO3)3}
                    \item   \ce{(NH4)3PO4}
                \end{enumerate}
            \end{multicols}

    \pagebreak
    \section{Review Problem 3}
        a) What is the ion charge on each atom of \ce{V} in the compound \ce{V2S3}?

        \noindent 
        b) What is the ion charge on each \ce{P3O10} group in the compound \ce{Ca5(P3O10)2}?

        \subsection{Solution (a)}
            My general intuition is that if in a compound, one has three in quantity nad the other has two, then the former will have two in charge (regardless of positiveness/negativeness) and the other will have three in the opposite charge.
            Let's just figure out a different way to do that.

            \ce{V} is in group 5B, so it could probably have one of several charges. 
            Looking at \ce{S} (Sulfur), it is in group 6A, so it naturally has a charge of \ce{S^2-}.
            Multiply the -2 charge by the 3 sulfur ions to get a net charge of -6.
            Since \ce{V} is a cation and not an anion, the net anion charge would be +6.
            Divide that by the two \ce{V} (Vanadium) ions to get an indivdual charge of \ce{V^3+}
            In the end, it will be a net charge of \boxed{3+}.

        \subsection{Solution (b)}
            A Cadmium ion naturally has a charge of 2+.
            Multiply that by 5 cadmium ions to get a total cation charge of 10+.
            Turn that negative for anions.
            Divide that by two \ce{P3O10} ions to get a charge of \boxed{5-}.

    \pagebreak
    \section{Review Problem 4}
        Balance the following chemical equation: \ce{Cr2O3 + HBr -> CrBr3 + H2O}.

        \subsection{Solution}
            Let's look at the water (\ce{H2O}).
            The Chromium Oxide (\ce{Cr2O3}) requires there be three oxide (\ce{O^2-}) total.
            The coefficient we can put on the water can as such be 3.
            This means that there will be six Hydrogen ions (\ce{H+}) total.
            There will as such be 6 Hydrogen Bromide (\ce{HBr}).
            This results in six Bromide (\ce{Br-}).
            They can be used in the six Bromide for the Chromium Bromide (\ce{CrBr3}) to give chromium bromide a coefficient of 2.
            The resultant total Chromium ion (\ce{Cr^3+}) will have a total count of three.
            This lines up well with the single Chromium Oxide.
            \begin{center}
                \boxed{\ce{Cr2O3 + 6HBr -> 2CrBr3 + 3H2O}}
            \end{center}

            These are technically minumum coefficients, but all chemical reaction formulae are and that should not be a problem.
    
    \pagebreak
    \section{Review Problem 5}
        If you have exactly one mole of \ce{Cr(NO3)3}, how many grams of this compound do you have?

        \subsection{Solution}
            First find the molar mass of \ce{Cr(NO3)3}.
            \begin{align}
                MM(\ce{Cr(NO3)3})   &=  MM(\ce{Cr}) + 3 * MM(\ce{N}) + 9 * MM(\ce{O})\\
                    &=  52.00 \unit{\gram/\mol} + 3 * 14.01 \unit{\gram/\mol} + 9 * 16.00 \unit{\gram/\mol}\\
                    &=  52.00 \unit{\gram/\mol} + 42.03 \unit{\gram/\mol} + 144.00 \unit{\gram/\mol}\\
                    &=  238.03 \unit{\gram/\mol}
            \end{align}

            Now we multiply that by the number of moles.
            \begin{equation}
                238.03 \unit{\gram/\mol} * 1 \unit{\mol}    =   \boxed{238.03 \unit{\gram}}
            \end{equation}
    
    \pagebreak
    \section{Review Problem 6}
        Convert each of the following to moles:
        \begin{enumerate}[label=\alph*)]
            \item 6.131 g of \ce{N} 
            \item 6.131 g of \ce{N2} 
            \item 6.131 g of \ce{N2O}
        \end{enumerate}

        \subsection{Solution (a)}
            The molar mass of Nitrogen (\ce{N}) is 14.01 \unit{\gram/\mol}.
            Divide the mass by the molar mass to get the number of moles.
            \begin{equation}
                \frac{6.131 \unit{\gram}}{14.01 \unit{\gram/\mol}} = 0.437615 \unit{\mol} \approx \boxed{0.4376 \unit{\gram}}
            \end{equation}

        \subsection{Solution (b)}
            The easy answer to this would be to divide the solution from part (a) by two since there would be twice the molar mass and as such half the moles.
            To be nice, I'll do it the traditional way instead.
            The molar mass of Nitrogen gas (\ce{N2}) is 28.02 \unit{\gram/\mol}.
            \begin{equation}
                \frac{6.131 \unit{\gram}}{28.02 \unit{\gram/\mol}} = 0.218808 \unit{\mol}   \approx \boxed{0.2188 \unit{\gram}}
            \end{equation} 

        \subsection{Solution (c)}
            Add the molar mass of oxygen (MM(\ce{O}) = 16.00 \unit{\gram/\mol}) to the molar mass of Nitrogen Gas (\ce{N2} = 28.02 \unit{\gram/\mol}) to get a total of 44.02 \unit{\gram/\mol}.
            I think you know the next step.
            \begin{equation}
                \frac{6.131 \unit{\gram}}{44.02 \unit{\gram/\mol}} = 0.139277 \unit{\mol} \approx   \boxed{0.1393 \unit{\gram}}
            \end{equation}
    
    \pagebreak
    \section{Review Problem 7}
        a) How many \ce{N2O} molecules are there in 6.131 g of \ce{N2O}? \textit{(Reminder: Avogadro's constant is $6.022 \times 10^{23}$ \unit{\mol^{-1}}.)} 
        
        \noindent
        b) How many nitrogen atoms are there in 6.131 g of \ce{N2O}?

        \subsection{Solution (a)}
            Review Problem 6(c) gave us that 6.131g of \ce{N2O} contains 0.1393 \unit{\mol} of \ce{N2O}.
            Multiply this by Avogadro's number to get the total number of atoms.
            I like the way you wrote the units of Avogadro's number as \unit{\mol^{-1}}. 
            \begin{equation}
                0.1393 \unit{\mol} * 6.022 \times 10^{23} \unit{\mol^{-1}} = \boxed{8.387 \times 10^{22}}
            \end{equation}

            I'm a real purist in terms of units (probably because of a heavy Physics background), but you could say that the units there is in terms of molecules.
        
        \subsection{Solution (b)}
            Since the ratio of \ce{N}:\ce{N2O} is 2:1, we multiply the number of molecules by 2 to get the number of Nitrogen molecules.
            \begin{equation}
                8.387 \times 10^{22} * 2 = \boxed{1.677 \times 10^{23}}
            \end{equation}

            As previously stated, I'm a units purist.
    
    \pagebreak
    \section{Review Problem 8}
        In 0.08157 moles of \ce{Al(ClO4)3}, there are…
        
        a) how many moles of \ce{Al^3+} ions?
        
        b) how many moles of \ce{ClO4-} ions?
        
        c) how many moles of oxygen atoms?

        \subsection{Solution (a)}
            Multiply the number of moles by the number of \ce{Al^3+} in each atom (1).
            \begin{equation}
                0.08157 \unit{\mol} * 1 \ce{Al^3+} = \boxed{0.08157 \unit{\mol} \ce{Al^3+}}
            \end{equation}

        \subsection{Solution (b)}
            Multiply the number of moles by the number of \ce{ClO4-} in each atom (3).
            \begin{equation}
                0.08157 \unit{\mol} * 3 \ce{ClO4-} = 0.24471 \unit{\mol} \ce{ClO4-} \approx \boxed{0.2447 \unit{\mol} \ce{ClO4-}}
            \end{equation}

        \subsection{Solution (c)}
            Multiply the number of moles of \ce{Al(ClO4)3} by the number of \ce{O} atoms ($ 3 \ce{ClO4} * 4 \ce{\frac{O}{ClO4}} = 12 \ce{O} $).
            \begin{equation}
                0.08157 \unit{\mol} * 12 \ce{O} = 0.97884 \unit{\mol} \ce{O} \approx \boxed{0.9788 \unit{\mol} \ce{O}}
            \end{equation}
    
    \pagebreak
    \section{Review Problem 9}
        If you have 285 atoms of fluorine…
        
        a) How many moles of fluorine atoms do you have?
        
        b) What is the mass of the fluorine atoms, in grams?

        \subsection{Solution (a)}
            Since each mole contains $6.022 \times 10^{23}$ atoms, we divide our atoms by avogadro's number.
            \begin{align}
                n   &=  \frac{285}{6.022 \times 10^{23}} \unit{\mol}
                    =   \boxed{4.73 \times 10^{-22} \unit{\mol}}
            \end{align}

        \subsection{Solution (b)}
            Fluorine (\ce{F}) has a molar mass of $19.00 \unit{\gram/\mol}$.
            Multiply molar mass by the number of moles to get the mass.
            \begin{align}
                m   &=  n * MM(\ce{F})
                    =   4.73 \times 10^{-22} \unit{\mol} * 19.00 \unit{\gram/\mol}\\
                    &=  \boxed{8.99 \unit{\gram}}
            \end{align}

    \pagebreak
    \section{Review Problem 10}
        \ce{KClO3} breaks down when it is heated. The chemical equation for this reaction is:
        \begin{center}
            \ce{2 KClO3 -> 2 KCl + 3 O2}
        \end{center}
        a) If 5.000 g of \ce{KClO3} breaks down, how many grams of \ce{O2} will be formed?
        
        \noindent
        b) If 2.000 g of \ce{O2} are formed, how many grams of \ce{KCl} will also be formed?

        \subsection{Solution (a)}
            I have a simpler solution, but I'll just do the classic thing.
            Convert grams to moles.
            \begin{align}
                MM(\ce{KClO3})  &=  MM(\ce{K}) + MM(\ce{Cl}) + 3 * MM(\ce{O})\\
                    &=  39.10 \unit{\gram/\mole} + 35.45 \unit{\gram/\mole} + 3 * 16.00 \unit{\gram/\mole}\\
                    &=  122.55 \unit{\gram/\mole}\\
                n(\ce{KClO3})   &=  \frac{m}{MM(\ce{KClO3})}
                    =   \frac{5.000 \unit{\gram} \ce{KClO3}}{122.55 \unit{\gram/\mole}}\\
                    &=  0.04080 \unit{\mole}
            \end{align}

            Then multiply it by $\frac{3 \ce{O2}}{2 \ce{KClO2}}$ from the chemical equation.
            \begin{align}
                n(\ce{O2})  &=  n(\ce{KClO3}) * \frac{3 \ce{O2}}{2 \ce{KClO2}}
                    =   0.04080 \unit{\mole} \ce{KClO2} * \frac{3 \ce{O2}}{2 \ce{KClO2}}\\
                    &=  0.6120 \unit{\mole} \ce{O2}
            \end{align}

            Multiply that by the molar mass of \ce{O2} ($32.00 \unit{\gram/\mol}$).
            \begin{align}
                m(\ce{O2})  &=  0.6120 \unit{\mole} \ce{O2} * 32.00 \unit{\gram/\mol}
                    =   \boxed{1.958 \unit{\gram}}
            \end{align}

        \subsection{Solution (b)}
            Use the same strategy as the part (a).
            \begin{align}
                MM(\ce{O2}) &=  32.00 \unit{\gram/\mole}\\
                n(\ce{O2})  &=  \frac{m}{MM(\ce{O2})}
                    =   \frac{2.000 \unit{\gram}}{32.00 \unit{\gram/\mole}}
                    =   0.0625 \unit{\mole}\\
                n(\ce{KCl}) &=  n(\ce{O2}) * \frac{2 \ce{KCl}}{3 \ce{O2}}
                    =   0.0625 \unit{\mole} \ce{O2} * \frac{2 \ce{KCl}}{3\ce{O2}}
                    =   0.04167 \unit{\mole} \ce{KCl}\\
                MM(KCl) &=  MM(\ce{K}) + MM(\ce{Cl})
                    =   39.10 \unit{\gram/\mole} + 35.45 \unit{\gram/\mole}\\
                    &=  74.55 \unit{\gram/\mole}\\
                m(\ce{KCl}) &=  0.04167 \unit{\mole} \ce{KCl} * 74.55 \unit{\gram/\mole}
                    =   \boxed{3.106 \unit{\gram}}
            \end{align}

\end{document}