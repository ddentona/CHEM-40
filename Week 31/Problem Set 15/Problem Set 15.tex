\documentclass[10pt]{article}
\usepackage[export]{adjustbox}
\usepackage{amsmath}
\usepackage{array}
\usepackage[makeroom]{cancel}
\usepackage{chemfig}
\usepackage{enumitem}
\usepackage{float}
\usepackage{graphicx}
%Load mhchem using some package options
\usepackage[version=4]{mhchem}
\usepackage{multicol}
\usepackage{siunitx}
\usepackage{wrapfig}

\title{
    Problem Set \#15
    \\  \small
    CHEM101A: General College Chemistry
    }
\author{Donald Aingworth}
\date{November 28, 2025}

\newcommand{\E}[1]{\times 10^{#1}}
\newcommand{\hc}{1.9864748\E{-25}\,\unit{\joule\,\meter}}
\newcommand{\U}[1]{\underline{#1}}

\begin{document}
    \DeclareSIUnit{\atm}{atm}
    \DeclareSIUnit{\molar}{M}
    \DeclareSIUnit{\M}{M}
    \DeclareSIUnit{\torr}{torr}

    \maketitle

    \setcounter{section}{11}
    \pagebreak
    \section{Topic G Problem 23}
        The concentration of H+ ions in a solution is 0.315 M.
        \begin{enumerate}[label=\alph*)]
            \item   Calculate the concentration of \ce{OH-} ions in this solution.
            \item   Where did these \ce{OH-} ions come from?
            \item   What is the pH of this solution?
        \end{enumerate}

        \subsection{Solution}

    \pagebreak
    \section{Topic G Problem 24}
        The pH of an HCl solution is 2.88.
a) What is the concentration of H+ ions in this solution?
b) What is the concentration of OH– ions in this solution?
c) What is the concentration of Cl– ions in this solution?

        \subsection{Solution}
        
    \pagebreak
    \section{Topic G Problem 25}
        Calculate the pH of a $7.4\E{-4}$ M solution of Ba(OH)2.

        \subsection{Solution}
        
    \pagebreak
    \section{Topic G Problem 26}
        Write the Ka expression and the corresponding chemical equation for each of the following weak acids.
a) HClO b) H2C4H4O4 c) NH4+ d) H2PO4–

        \subsection{Solution}
        
    \pagebreak
    \section{Topic G Problem 27}
        The pH of a 0.464 M solution of phosphorous acid (H3PO3) is 1.11. 
        Using this information, calculate the Ka of phosphorous acid. 
        (You may assume that only one hydrogen ion dissociates from phosphorous acid.)

        \subsection{Solution}
        
    \pagebreak
    \section{Topic G Problem 28}
        Calculate the pH of a 0.27 M solution of HCO2H (formic acid, Ka = 1.8 x 10–4)

        \subsection{Solution}
        
    \pagebreak
    \section{Topic G Problem 29}
        Determine which solution from each of the following pairs has the higher pH. 
        You may need top refer to the Ka values in Table 12.4.2 of your textbook.
a) 0.1 M HCl or 0.1 M HNO2 b) 0.1 M HF or 0.1 M HClO
c) 0.1 M HCN or 0.1 M NaCN

        \subsection{Solution}
        
    \pagebreak
    \section{Topic G Problem 30}
        Each of the following species can function as an acid. Write the formula of its conjugate base.
a) HC3H5O3 b) N2H5+ c) H2O d) HCO3– e) H2SO4


        \subsection{Solution}
        
    \pagebreak
    \section{Topic G Problem 31}
        Each of the following species can function as a base. Write the formula of its conjugate acid.
a) NH3 b) HSO3– c) H2O d) PO43–


        \subsection{Solution}
        
    \pagebreak
    \section{Topic G Problem 32}
        Identify the acid and the base in each of the following reactions.
a) HNO2(aq) + H2O(l) → H3O+(aq) + NO2–(aq)
b) H2PO4–(aq) + HSO4–(aq) → H3PO4(aq) + SO42–(aq)


        \subsection{Solution}
        
    \pagebreak
    \section{Topic G Problem 33}
        For each of the following reactions, tell whether the equilibrium will favor the reactants or the products. 
        Use the Ka values in Table 12.4.2 of your textbook.
a) \ce{HC3H5O3(aq) + CN-(aq) <=> C3H5O3-(aq) + HCN(aq)}
b) \ce{HOCl(aq) + OH-(aq) <=> OCl-(aq) + H2O(l)}


        \subsection{Solution}
        
    \pagebreak
    \tableofcontents
\end{document}


% \begin{center}
%     \includegraphics[width=0.7\textwidth]{Answers Images/F23.jpg}
% \end{center}

% \begin{wrapfigure}{r}{0.25\textwidth}
%     \vspace{-30pt}
%     \includegraphics[width=0.25\textwidth]{img-F24.png}\\
%     \includegraphics[width=0.25\textwidth]{Answers Images/F24.jpg}
%     % \label{fig:wrapfig}
% \end{wrapfigure}