\documentclass[10pt]{article}
\usepackage[export]{adjustbox}
\usepackage{amsmath}
\usepackage{array}
\usepackage[makeroom]{cancel}
\usepackage{chemfig}
\usepackage{enumitem}
\usepackage{float}
\usepackage{graphicx}
%Load mhchem using some package options
\usepackage[version=4,textfontname=sffamily,mathfontname=mathsf]{mhchem}
\usepackage{multicol}
\usepackage{siunitx}
\usepackage{wrapfig}

\title{
    Problem Set \#15
    \\  \small
    CHEM101A: General College Chemistry
    }
\author{Donald Aingworth}
\date{November 28, 2025}

\newcommand{\E}[1]{\times 10^{#1}}
\newcommand{\hc}{1.9864748\E{-25}\,\unit{\joule\,\meter}}
\newcommand{\pH}{\text{pH}}
\newcommand{\U}[1]{\underline{#1}}

\begin{document}
    \DeclareSIUnit{\atm}{atm}
    \DeclareSIUnit{\molar}{M}
    \DeclareSIUnit{\M}{M}
    \DeclareSIUnit{\torr}{torr}

    \maketitle

    \setcounter{section}{22}
    \pagebreak
    \section{Topic G Problem 23}
        The concentration of \ce{H+} ions in a solution is 0.315 M.
        \begin{enumerate}[label=\alph*)]
            \item   Calculate the concentration of \ce{OH-} ions in this solution.
            \item   Where did these \ce{OH-} ions come from?
            \item   What is the pH of this solution?
        \end{enumerate}

        \subsection{Solution (a)}
            The concentration of hydrogen ions times the number of hydroxide ions is equal to a constant $K_w = 10^{-14}$.
            We can use that to find the concentration of hydroxide ions.
            \begin{gather}
                K_w = [\ce{H+}][\ce{OH-}]\\
                [\ce{OH-}] = \frac{K_w}{[\ce{H+}]}
                    =   \frac{10^{-14}}{0.315}
                    =   3.17\E{-14}\,\unit{\molar}
                    =   \boxed{3.2\E{-14}\,\unit{\molar}}
            \end{gather}

        \subsection{Solution (b)}
            The \ce{OH-} ions come from the surrounding water.

        \subsection{Solution (c)}
            Use the logarithm.
            \begin{equation}
                \pH =   -\log_{10}([\ce{H+}])
                    =   -\log_{10}(0.315)
                    =   \boxed{0.50}
            \end{equation}

    \pagebreak
    \section{Topic G Problem 24}
        The pH of an HCl solution is 2.88.
        \begin{enumerate}[label=\alph*)]
            \item   What is the concentration of \ce{H+} ions in this solution?
            \item   What is the concentration of \ce{OH-} ions in this solution?
            \item   What is the concentration of \ce{Cl-} ions in this solution?
        \end{enumerate}

        \subsection{Solution (a)}
            Use the exponential.
            \begin{equation}
                [\ce{H+}] = 10^{-\pH}
                    =   10^{-2.88}
                    =   \boxed{0.0013\,\unit{\molar}}
            \end{equation}

        \subsection{Solution (b)}
            The \ce{OH-} and \ce{H+} concentrations are related.
            \begin{gather}
                K_w = [\ce{H+}][\ce{OH-}]\\
                [\ce{OH-}] = \frac{K_w}{[\ce{H+}]}
                    =   \frac{10^{-14}}{0.0013}
                    =   7.586\E{-12}\,\unit{\molar}
                    =   \boxed{7.6\E{-12}\,\unit{\molar}}
            \end{gather}

        \subsection{Solution (c)}
            There is much less \ce{OH-} than there is \ce{H+}.
            Even if some of the created \ce{OH-} balanced out the \ce{H+}, it would have been little.
            This means the \ce{Cl-} would be equivalent to the amount of \ce{H+}.
            \boxed{0.0013\,\unit{\molar}}
        
    \pagebreak
    \section{Topic G Problem 25}
        Calculate the pH of a $7.4\E{-4}$ M solution of \ce{Ba(OH)2}.

        \subsection{Solution}
            Start by calculating the concentration of the \ce{OH-}.
            It would be twice the concentration of the \ce{Ba(OH)2} because it is soluble in water.
            \begin{equation}
                [\ce{OH-}]  =   2 * 7.4\E{-4}\,\unit{\molar}
                    =   1.48\E{-3}\,\unit{\molar}
            \end{equation}

            This leads into a calculation of the concentration of the \ce{H+}.
            We can use that to find the pH.
            \begin{gather}
                K_w = [\ce{H+}][\ce{OH-}]\\
                [\ce{H+}] = \frac{K_w}{[\ce{OH-}]}
                    =   \frac{10^{-14}}{1.48\E{-3}\,\unit{\molar}}
                    =   6.757\E{-12}\,\unit{\molar}\\
                \pH =   -\log_{10}[\ce{H+}]
                    =   -\log_{10}(6.757\E{-12})
                    =   \boxed{11.17}
            \end{gather}
        
    \pagebreak
    \section{Topic G Problem 26}
        Write the $K_a$ expression and the corresponding chemical equation for each of the following weak acids.
        \begin{multicols}{2}
            \begin{enumerate}[label=\alph*)]
                \item   \ce{HClO}
                \item   \ce{H2C4H4O4}
                \item   \ce{NH4+}
                \item   \ce{H2PO4-}
            \end{enumerate}
        \end{multicols}

        \subsection{Solution (a)}
            \begin{gather}
                K_a = \frac{[\ce{H+}][\ce{ClO-}]}{[\ce{HClO}]}\\
                \ce{HClO(aq) <=> H+(aq) + ClO-(aq)}
            \end{gather}

        \subsection{Solution (b)}
            \begin{gather}
                K_a = \frac{[\ce{H+}][\ce{HC4H4O4-}]}{[\ce{H2C4H4O4}]}\\
                \ce{H2C4H4O4 (aq) <=> H+ (aq) + HC4H4O4- (aq)}
            \end{gather}

        \subsection{Solution (c)}
            \begin{gather}
                K_a = \frac{[\ce{H+}][\ce{NH3}]}{[\ce{NH4+}]}\\
                \ce{NH4+ (aq) <=> H+ (aq) + NH3 (aq)}
            \end{gather}

        \subsection{Solution (d)}
            \begin{gather}
                K_a = \frac{[\ce{H+}][\ce{HPO4^2-}]}{[\ce{H2PO4-}]}\\
                \ce{H2PO4- (aq) <=> H+ (aq) + HPO4^2- (aq)}
            \end{gather}
        
    \pagebreak
    \section{Topic G Problem 27}
        The pH of a 0.464 M solution of phosphorous acid (\ce{H3PO3}) is 1.11. 
        Using this information, calculate the $K_a$ of phosphorous acid. 
        (You may assume that only one hydrogen ion dissociates from phosphorous acid.)

        \subsection{Solution}
            First find the molarity of the \ce{H+} by using the pH.
            \begin{equation}
                [\ce{H+}]   =   10^{-\pH}
                    =   10^{-1.11}
                    =   0.0776247
            \end{equation}

            That would in turn be the molarity of the \ce{H2PO3-}.
            Without having to use an ICE table, we can figure out that the equilibrium molarity of the \ce{H3PO3} will be the nitial molarity minus the molarity of the \ce{H+}.
            \begin{equation}
                [\ce{H3PO3}]_f = [\ce{H3PO3}]_i - [\ce{H+}]
                    =   0.464 - 0.078
                    =   0.386
            \end{equation}

            Use these and the molarity of the \ce{H3PO3} to fnd the $K_a$.
            \begin{equation}
                K_a =   \frac{[\ce{H+}][\ce{H2PO3-}]}{[\ce{H3PO3}]}
                    =   \frac{0.0776247^2}{0.386}
                    =   \boxed{0.016}
            \end{equation}
        
    \pagebreak
    \section{Topic G Problem 28}
        Calculate the pH of a 0.27 M solution of \ce{HCO2H} (formic acid, $K_a = 1.8\E{-4}$).

        \subsection{Solution}
            Use an ICE table to find the molarity of the \ce{H+}.
            \begin{center}
                \begin{tabular}{| l | c c c c c |}
                    \hline
                    M   &   \ce{HCO2H} &\ce{<=>}& \ce{H+} &+& \ce{CO2H-}\\
                    \hline
                    I   &   0.27    &&  0   &&  0\\
                    C   &   $-x$    &&  $+x$    &&  $+x$\\
                    E   &   $0.27-x$    &&  $x$    &&  $x$\\
                    \hline
                \end{tabular}
            \end{center}

            Use this with the $K_a$.
            Solve for $x$.
            \begin{gather}
                K_a =   \frac{[\ce{H+}][\ce{CO2H-}]}{[\ce{HCO2H}]}
                    =   \frac{x^2}{0.27 - x}
                    =   1.8\E{-4}\\
                x^2 =   4.86\E{-5} - 1.8\E{-4} x\\
                0   =   x^2 + 1.8\E{-4}x - 4.86\E{-5}\\
                x   =   0.00688195 \text{ or } \cancel{-0.00706195}
            \end{gather}

            This value of $x$ is the concentration of the \ce{H+}.
            Use this to find the pH.
            \begin{equation}
                \pH =   -\log_{10} [\ce{H+}] 
                    =   -\log_{10} x
                    =   -\log_{10} 0.00688195
                    =   \boxed{2.16}
            \end{equation}
        
    \pagebreak
    \section{Topic G Problem 29}
        Determine which solution from each of the following pairs has the higher pH. 
        You may need top refer to the $K_a$ values in Table 12.4.2 of your textbook.
        \begin{enumerate}[label=\alph*)]
            \item   0.1 M \ce{HCl} or 0.1 M \ce{HNO2}
            \item   0.1 M \ce{HF} or 0.1 M \ce{HClO}
            \item   0.1 M \ce{HCN} or 0.1 M \ce{NaCN}
        \end{enumerate}

        \subsection{Solution (a)}
            The \ce{HCl} is a strong acid, so practically all the \ce{HCl} will dissociate.
            We will resultantly have a low pH.
            That means that the higher pH goes to the \boxed{\ce{HNO2}}.

        \subsection{Solution (b)}
            \ce{HF} has $K_a = 3.5\E{-4}$.
            \ce{HClO} has $K_a = 3.5\E{-8}$.
            \ce{HClO} has the lower $K_a$, which means it will have a higher pH.
            The answer is \boxed{\ce{HClO}}.

        \subsection{Solution (c)}
            \ce{HCN} has $K_a = 6.2\E{-10}$.
            The \ce{HCN} will increase the concentration of the \ce{H+}.
            \ce{NaCN} has no $K_a$ because it has no Hydrogen.
            However, the entirety of the \ce{NaCN} will dissociate because of the \ce{Na+} present.
            The \ce{CN-} will then bond with the \ce{H+} and lower the molarity of the \ce{H+}.
            The \ce{HCN} will result in a higher concentration of \ce{H+}, while the \ce{NaCN} will lower the concentration of the \ce{H+}.
            Since pH is inversely (logarithmically) proportional to concentration, this means the \boxed{\ce{NaCN}} will have the higher pH.
        
    \pagebreak
    \section{Topic G Problem 30}
        Each of the following species can function as an acid. Write the formula of its conjugate base.
        \begin{multicols}{3}
            \begin{enumerate}[label=\alph*)]
                \item   \ce{HC3H5O3}
                \item   \ce{N2H5+}
                \item   \ce{H2O}
                \item   \ce{HCO3-}
                \item   \ce{H2SO4}
            \end{enumerate}
        \end{multicols}
        

        \subsection{Solution}
            \begin{multicols}{3}
                \begin{enumerate}[label=\alph*/]
                    \item   \ce{C3H5O3-}
                    \item   \ce{N2H4}
                    \item   \ce{OH-}
                    \item   \ce{CO3^2-}
                    \item   \ce{HSO4-}
                \end{enumerate}
            \end{multicols}
        
    % \pagebreak
    \section{Topic G Problem 31}
        Each of the following species can function as a base. Write the formula of its conjugate acid.
        \begin{multicols}{3}
            \begin{enumerate}[label=\alph*)]
                \item   \ce{NH3}
                \item   \ce{HSO3-}
                \item   \ce{H2O}
                \item   \ce{PO4^3-}
            \end{enumerate}
        \end{multicols}

        \subsection{Solution}
            \begin{multicols}{3}
                \begin{enumerate}[label=\alph*)]
                    \item   \ce{NH4+}
                    \item   \ce{H2SO3}
                    \item   \ce{H3O+}
                    \item   \ce{HPO4^2-}
                \end{enumerate}
            \end{multicols}
        
    \pagebreak
    \section{Topic G Problem 32}
        Identify the acid and the base in each of the following reactions.
        \begin{enumerate}
            \item   \ce{HNO2(aq) + H2O(l) -> H3O+(aq) + NO2-(aq)}
            \item   \ce{H2PO4-(aq) + HSO4-(aq) -> H3PO4(aq) + SO4^2-(aq)}
        \end{enumerate}

        \subsection{Solution (a)}
            Acid: \ce{HNO2}\\
            Base: \ce{H2O}
        
        \subsection{Solution (b)}
            Acid: \ce{HSO4-}\\
            Base: \ce{H2PO4-}
        
    \pagebreak
    \section{Topic G Problem 33}
        For each of the following reactions, tell whether the equilibrium will favor the reactants or the products. 
        Use the $K_a$ values in Table 12.4.2 of your textbook.
        \begin{enumerate}[label=\alph*)]
            \item   \ce{HC3H5O3(aq) + CN-(aq) <=> C3H5O3-(aq) + HCN(aq)}
            \item   \ce{HOCl(aq) + OH-(aq) <=> OCl-(aq) + H2O(l)}
        \end{enumerate}

        \subsection{Solution (a)}
            $K_a$ for \ce{HC3H5O3} is $1.4\E{-4}$.
            $K_a$ for \ce{HCN} is $6.2\E{-10}$.
            The $K_a$ for \ce{HC3H5O3} is greater, so the equilibrium will favor the \underline{products}.

        \subsection{Solution (b)}
            $K_a$ for \ce{HOCl} is $3.5\E{-8}$.
            $K_a$ for \ce{H2O} is $1.0\E{-14}$.
            The $K_a$ for \ce{HOCl} is greater, so the equilibrium will favor the \underline{products}.
        
    \pagebreak
    \tableofcontents
\end{document}


% \begin{center}
%     \includegraphics[width=0.7\textwidth]{Answers Images/F23.jpg}
% \end{center}

% \begin{wrapfigure}{r}{0.25\textwidth}
%     \vspace{-30pt}
%     \includegraphics[width=0.25\textwidth]{img-F24.png}\\
%     \includegraphics[width=0.25\textwidth]{Answers Images/F24.jpg}
%     % \label{fig:wrapfig}
% \end{wrapfigure}