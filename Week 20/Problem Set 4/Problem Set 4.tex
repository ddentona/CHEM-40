\documentclass[10pt]{article}
\usepackage[export]{adjustbox}
\usepackage{amsmath}
\usepackage[makeroom]{cancel}
\usepackage{enumitem}
\usepackage{graphicx}
%Load mhchem using some package options
\usepackage[version=4]{mhchem}
\usepackage{multicol}
\usepackage{siunitx}

\title{
    Problem Set \#4
    \\  \small
    CHEM101A: General College Chemistry
    }
\author{Donald Aingworth IV}
\date{September 12, 2025}

\begin{document}
    \DeclareSIUnit{\molarity}{M}
    \DeclareSIUnit{\M}{M}

    \maketitle

    \pagebreak
    \section{Topic C Problem R-1}
        A container holds 415 mL of air at a pressure of 1.88 atm. 
        If you want to reduce the pressure to 1.55 atm without changing the temperature, to what volume must you expand the air?

        \subsection{Solution}

    \pagebreak
    \section{Topic C Problem R-2}
        A balloon is filled with 3.85 L of oxygen at 31ºC and a pressure of 734 torr. 
        The balloon is then taken to the top of a mountain, where the pressure is 591 torr. 
        The volume of the oxygen is found to be 4.13 L. What is the temperature of the oxygen? 
        Give your answer in ºC.

        \subsection{Solution}

    \pagebreak
    \section{Topic C Problem R-3}
        A container is filled with gaseous O2 at 21.4˚C (70.5˚F) and a pressure of 754 torr. 
        If the volume of the container is one gallon (3785 mL), what is the mass of the O2?

        \subsection{Solution}

    \pagebreak
    \section{Topic C Problem R-4}
        A container is filled with 25.3 g of gaseous CO2 at 31ºC. 
        If the carbon dioxide exerts a pressure of 3.88 atm, what is the volume of the container?

        \subsection{Solution}

    \pagebreak
    \section{Topic C Problem R-5}
        What is the density of gaseous carbon dioxide at 51ºC and a pressure of 855 torr?

        \subsection{Solution}

    \pagebreak
    \section{Topic C Problem 1}
        A gaseous compound has the following composition: 85.63% carbon, 14.37% hydrogen. 
        The density of this compound is 1.63 g/L at 33.2ºC and a pressure of 739 torr. 
        What is the molecular formula of the compound?

        \subsection{Solution}

    \pagebreak
    \section{Topic C Problem 2}
        Copper reacts with nitric acid as shown below:
        \begin{gather}
            \ce{Cu(s) + 4 HNO3(aq) → Cu(NO3)2(aq) + 2 NO2(g) + 2 H2O(l)}
        \end{gather}
        If 4.71 g of copper reacts with excess 3 M nitric acid, what volume of nitrogen dioxide will be formed at 31ºC and 1.022 atm?

        \subsection{Solution}

    \pagebreak
    \section{Topic C Problem 3}
        Consider the following reaction:
        \begin{gather}
            \ce{4 Cr2+(aq) + O2(g) + 4 H+(aq) → 4 Cr3+(aq) + 2 H2O(l)}
        \end{gather}
        A container that holds 562 mL of gaseous oxygen at 21ºC is prepared. Then, 21.3 mL of a
        solution that contains 0.131 M Cr2+ ions is added to the container. After the reaction, the
        pressure of the oxygen in the container is found to be 119 torr and the temperature is still 21ºC.
        What was the pressure of oxygen in the container before the Cr2+ solution was added? (You can
        assume that H+ is present in excess.)

        \subsection{Solution}

    \pagebreak
    \section{Topic C Problem 4}
        A chemist puts 200.0 mL of water into a 2.50 L container, and adds enough gaseous H2S to give a pressure of 180.2 torr at a temperature of 13.0˚C. 
        Some of the H2S then dissolves in the water, causing the pressure in the container to drop to 155.9 torr. 
        The temperature remains constant throughout this experiment. 
        What is the molar concentration of H2S in the water at the end of the experiment?

        \subsection{Solution}

    \pagebreak
    \section{Topic C Problem 5}
        Consider the apparatus pictured below, which consists of two containers separated by a valve.
        \begin{center}
            \includegraphics{picture_C-5.png}
        \end{center}
        Assuming that the two gases are the same temperature and that the temperature does not change, what will be the total final pressure in the system after the valve is opened and the gases mix completely?

        \subsection{Solution}

    \pagebreak
    \section{Topic C Problem 6}
        Complete the following ICE table. 
        All of the reactants and products for this reaction are gases, and the temperature and volume are constant during the reaction.
        C2H6 + 7 F2 → 2 CF4 + 6 HF
        Initial pressure (torr): 127.3 329.5 89.1 0
        Change (torr):
        Ending pressure (torr):

        \subsection{Solution}

    \pagebreak
    \section{Topic C Problem 7}
        At 30ºC, propylene (C3H6) reacts with chlorine as follows:
        \begin{equation}
            \ce{C3H6(g) + 9 Cl2(g) → 3 CCl4(l) + 6 HCl(g)}
        \end{equation}
        A mixture containing 15.5 torr of C3H6 and 174.5 torr of Cl2 is allowed to react at 30ºC. 
        What will be the total gas pressure in the container when the reaction is complete? 
        You may assume that the volume and temperature are constant.

        \subsection{Solution}

    \pagebreak
    \section{Topic C Problem 8}
        Consider the apparatus pictured below, which consists of two containers separated by a valve.
        \begin{center}
            \includegraphics{picture_C-8.png}
        \end{center}
        The valve is opened and the gases react as follows (note that the temperature is high enough that
        the water is produced as a gas):
        \begin{equation}
            \ce{4 C2H7N(g) + 15 O2(g) → 8 CO2(g) + 14 H2O(g) + 2 N2(g)}
        \end{equation}
        What will be the total pressure in the apparatus when the reaction has gone to completion?
        Assume that the temperature does not change.

        \subsection{Solution}

    \pagebreak
    \section{Topic C Problem 9}
        Consider the apparatus pictured below, which consists of two containers separated by a valve.
        \begin{center}
            \includegraphics{picture_C-9.png}
        \end{center}
        The valve is opened and the gases react as follows:
        \begin{equation}
            \ce{H2S(g) + 2 NH3(g) → (NH4)2S(s)}
        \end{equation}
        What will be the total pressure in the apparatus when the reaction has gone to completion?
        Assume that the temperature does not change.

        \subsection{Solution}

    \pagebreak

    \tableofcontents
\end{document}