\documentclass[10pt]{article}
\usepackage[export]{adjustbox}
\usepackage{amsmath}
\usepackage{array}
\usepackage[makeroom]{cancel}
\usepackage{chemfig}
\usepackage{enumitem}
\usepackage{float}
\usepackage{graphicx}
%Load mhchem using some package options
\usepackage[version=4,textfontname=sffamily,mathfontname=mathsf]{mhchem}
\usepackage{multicol}
\usepackage{siunitx}
\usepackage{wrapfig}

\title{
    Problem Set \#16
    \\  \small
    CHEM101A: General College Chemistry
    }
\author{Donald Aingworth}
\date{December 8, 2025}

\newcommand{\E}[1]{\times 10^{#1}}
\newcommand{\hc}{1.9864748\E{-25}\,\unit{\joule\,\meter}}
\newcommand{\pH}{\text{pH}}
\newcommand{\U}[1]{\underline{#1}}

\begin{document}
    \DeclareSIUnit{\atm}{atm}
    \DeclareSIUnit{\molar}{M}
    \DeclareSIUnit{\M}{M}
    \DeclareSIUnit{\torr}{torr}

    \maketitle

    % \setcounter{section}{22}
    \pagebreak
    \section{Topic H Problem 1}
        Classify each of the following substances as molecular, ionic, or metallic, and explain your answers. 
        (There are no network covalent substances in this list.) 
        Base your answers on the chemical formulas and the positions of the elements in the periodic table: you should not need to look up any additional information.
        \begin{multicols}{5}
            \begin{enumerate}[label=\alph*)]
                \item   \ce{Cl2S}
                \item   \ce{K2S}
                \item   \ce{H2SO4}
                \item   \ce{Fe}
                \item   \ce{He}
            \end{enumerate}
        \end{multicols}

        \subsection{Solution}
            \begin{enumerate}[label=\alph*/]
                \item   \ce{Cl} and \ce{S} are both nonmetals, so it is not metalic. Neither are natural cations, so it is not ionic. It is \underline{molecular}.
                \item   \ce{K} is a natural cation and \ce{S} is a natural anion, so it is \underline{ionic}.
                \item   No metals and the only natural cation is \ce{H}, so it is \underline{molecular}.
                \item   It's a metal, so it's \underline{metallic}.
                \item   It's not a metal and not a cation or anion, so it must be \underline{molecular}, although it si a noble gas.
            \end{enumerate}
        
        
    \pagebreak
    \section{Topic H Problem 2}
        Each of the following substances is a solid at room temperature. 
        What would you expect to be the most important type of attractive force between individual formula units in each of these substances? 
        Base your answers on the chemical formulas and the positions of the elements in the periodic table: you should not need to look up any additional information.
        \begin{multicols}{4}
            \begin{enumerate}[label=\alph*)]
                \item   \ce{CBr4}
                \item   \ce{CaBr2}
                \item   \ce{Ca}
                \item   \ce{C}
            \end{enumerate}
        \end{multicols}

        \subsection{Solution}
            \begin{enumerate}[label=\alph*/]
                \item   Molecular bonding and London Dispersion Forces.
                \item   Ionic bonding, but dipole-dipole bonds are also there.
                \item   Metallic bonding.
                \item   Covalent bonding.
            \end{enumerate}
        
        
    \pagebreak
    \section{Topic H Problem 3}
        For each of the following molecular substances, list all of the intermolecular forces that play a significant role in determining the physical properties of the substance. 
        \textit{Reminder: the three types of intermolecular forces are London dispersion forces, dipole-dipole attraction, and hydrogen bonding.}
        \begin{multicols}{5}
            \begin{enumerate}[label=\alph*)]
                \item   \ce{NH3}
                \item   \ce{N2}
                \item   \ce{NF3}
                \item   \ce{CH4}
                \item   \ce{CH2O}
            \end{enumerate}
        \end{multicols}

        \subsection{Solution}
            \begin{enumerate}[label=\alph*/]
                \item   LDFs and dipole-dipole (in the form of Hydrogen Bonds), so all three.
                \item   LDFs.
                \item   LDFs and dipole-dipole.
                \item   LDFs.
                \item   LDFs and dipole-dipole.
            \end{enumerate}
        
        
    \pagebreak
    \section{Topic H Problem 4}
        From each of the following pairs of compounds, select the compound that should have the higher boiling point, and explain your choice.
        \begin{enumerate}[label=\alph*)]
            \item   \ce{CO} and \ce{BeO}
            \item   \ce{NaF} and \ce{MgO}
        \end{enumerate}

        \subsection{Solution (a)}
            \ce{CO} has LDFs and some dipole-dipole forces.
            \ce{BeO} is held together by ionic bonds, so it would have stronger IMFs.
            For this reason, \boxed{\ce{BeO}} has the higher boiling point.

        \subsection{Solution (b)}
            Both molecules are metallic.
            \ce{NaF} is made up of two atoms that naturally have a $\pm 1$ charge.
            \ce{MgO} is made up of two atoms that naturally have a $\pm 2$ charge, so there would be stronger electrostatic forces between them.
            For this reason, \boxed{\ce{MgO}} would have the higher boiling point.
        
    \pagebreak
    \section{Topic H Problem 5}
        \ce{HCl} is a strong electrolyte (as you already know!). 
        Its melting point and boiling point are $-114\unit{\celsius}$ and $-85\unit{\celsius}$, respectively.
        \begin{enumerate}
            \item   What state (solid/liquid/gas) is \ce{HCl} in at room temperature?
            \item   A student concludes that \ce{HCl} is an ionic compound, based on the fact that it ionizes completely in aqueous solution. 
            Would you agree with this conclusion? 
            If not, what sort of compound is \ce{HCl}? 
            Be sure to discuss the melting and boiling points in your answer.
        \end{enumerate}

        \subsection{Solution (a)}
            Gas.

        \subsection{Solution (b)}
            It's molecular.
            Hydrogen (especially \ce{H+} since it's just a proton) is weird.
            Chlorine, being a nonmetal, cannot form ionic bonds with Hydrogen, so it would have to be a \underline{molecular} compount.
        
    \pagebreak
    \section{Topic H Problem 6}
        Using your knowledge of intermolecular forces, explain the trend in boiling points in the following series of compounds.
        \begin{multicols}{4}
            \resizebox{\textwidth/6}{!}{\chemname{\chemfig{H-[:0]C(-[:90]H)(-[:-90]H)-[:0]\charge{90=\:,0=\:,-90=\:}{F}}}{Fluoromethane}}\\
            boils at $-78\unit{\celsius}$

            \resizebox{\textwidth/6}{!}{\chemname{\chemfig{H-[:0]C(-[:90]H)(-[:-90]H)-[:0]\charge{90=\:,0=\:,-90=\:}{Cl}}}{Chloromethane}}\\
            boils at $-24\unit{\celsius}$

            \resizebox{\textwidth/6}{!}{\chemname{\chemfig{H-[:0]C(-[:90]H)(-[:-90]H)-[:0]\charge{90=\:,0=\:,-90=\:}{Br}}}{Bromethane}}\\
            boils at $4\unit{\celsius}$

            \resizebox{\textwidth/6}{!}{\chemname{\chemfig{H-[:0]C(-[:90]H)(-[:-90]H)-[:0]\charge{90=\:,0=\:,-90=\:}{I}}}{Idomethane}}\\
            boils at $42\unit{\celsius}$
        \end{multicols}

        \subsection{Solution}
            The only difference between these four is the group 7 element bonded with the Carbon, which is heavier the higher the boiling point is.
            Normally I would explain this with velocity from Kinetic Energy being dependent on mass, but that's not allowed here.
            Higher heaviness of atoms is resultant from higher amounts of protons, which leads to higher polarizability (manipulation of electron cloud shape).
            This in turn leads to stronger LDFs, which leads to a higher boiling point.
        
    \pagebreak
    \section{Topic H Problem 7}
        One of the compounds below boils at $70\unit{\celsius}$, while the other boils at $116\unit{\celsius}$. 
        Which is which?
        Explain your reasoning.
        \begin{multicols}{2}
            \chemname{\chemfig{\charge{-90=\:,90=\:}{O}=C*6(-C(-H)=C(-H)-C(=\charge{-90=\:,90=\:}{O})-C(-H)=C(-H)-)}}{compound 1}
            
            \chemname{\chemfig{H-C*6(=C(-H)-C(=\charge{-150=\:,30=\:}{O})-C(=\charge{-90=\:,90=\:}{O})-C(-H)=C(-H)-)}}{compound 2}
        \end{multicols}

        \subsection{Solution}
            The one with the greater intermolecular forces will have the the higher boiling point.
            Both have the same composition and a similar structure.
            One has all Carbon-Oxygen combinations surrounded by Carbon-Hydrogen combinations, while the other has the Carbon-Oxygen combinations adjacent one another.
            The adjacency in Compound 2 leads to less symmatry in the molecule, which leads to an overall polarization of the molecule in one direction that is not canceled out.
            That leads to the dipole being stronger for compound 2 and as such the intermolecular forces.
            This leads to compount 2 having the higher boiling point.
            Compund 1 boils at 70\unit{\celsius}, while compound 2 boils at 116\unit{\celsius}.
        
    \pagebreak
    \section{Topic H Problem 8}
        The following three compounds are isomers (they have the same chemical formula), but they have quite different boiling points: one boils at $82\unit{\celsius}$, one at $135\unit{\celsius}$, and one at $235\unit{\celsius}$. 
        Match each substance with the correct boiling point, and explain your reasoning.
        \begin{center}
            \chemname{\chemfig{H-\charge{-90=\:,90=\:}{O}-C(-[:90]H)(-[:-90]H)-C(-[:90]H)(-[:-90]H)-C(-[:90]H)(-[:-90]H)-C(-[:90]H)(-[:-90]H)-\charge{-90=\:,90=\:}{O}-H}}{compound 1}

            \chemname{\chemfig{H-C(-[:90]H)(-[:-90]H)-C(-[:90]H)(-[:-90]H)-\charge{-90=\:,90=\:}{O}-C(-[:90]H)(-[:-90]H)-C(-[:90]H)(-[:-90]H)-\charge{-90=\:,90=\:}{O}-H}}{compound 2}

            \chemname{\chemfig{H-C(-[:90]H)(-[:-90]H)-\charge{-90=\:,90=\:}{O}-C(-[:90]H)(-[:-90]H)-C(-[:90]H)(-[:-90]H)-\charge{-90=\:,90=\:}{O}-C(-[:90]H)(-[:-90]H)-H}}{compound 3}
        \end{center}
        
        \subsection{Solution}
            Compound 1 has 2 possible Hydrogen bonds.
            Compound 2 has 1 possible Hydrogen bonds.
            Compound 3 has 0 possible Hydrogen bonds.
            More possible Hydrogen bonds means a higher boiling point due to more intermolecular forces.
            \begin{enumerate}
                \item   82\unit{\celsius}: Compound 3
                \item   135\unit{\celsius}: Compound 2
                \item   235\unit{\celsius}: Compound 1
            \end{enumerate}
        
        
    \pagebreak
    \section{Topic H Problem 9}
        The phase diagram for ammonia is shown below (not drawn to scale).
        \begin{center}
            \includegraphics[width=\textwidth]{P9graph.png}
        \end{center}
        \begin{enumerate}[label=\alph*)]
            \item   What state is ammonia in at 25°C and 1 atm?
            \item   What state is ammonia in at -80°C and 10 atm?
            \item   What are the temperature and pressure at the triple point of ammonia?
            \item   What are the temperature and pressure at the critical point of ammonia?
            \item   If you heat ammonia from $-100\unit{\celsius}$ to $150\unit{\celsius}$ at a pressure of 5 atm, what states will you observe?
            \item   If you heat ammonia from $-100\unit{\celsius}$ to $150\unit{\celsius}$ at a pressure of 0.01 atm, what states will you observe?
            \item   A container holds ammonia at $-77.70\unit{\celsius}$ and a pressure of 0.010 atm. If the pressure is increased to 150 atm without changing the temperature, what states will you observe?
            \item   If you increase the temperature of ammonia from $125\unit{\celsius}$ to $175\unit{\celsius}$ at a pressure of 200 atm, what states will you observe?
        \end{enumerate}

        \subsection{Solution}
            \begin{enumerate}[label=\alph*/]
                \item   Gas
                \item   Solid, although it would be close to liquid
                \item   0.060 atm and $-77.75\unit{\celsius}$
                \item   111.3 atm and $132.4\unit{\celsius}$
                \item   Solid, then melting to liquid, then evaporation to gas.
                \item   Solid, then sublimation to gas.
                \item   Gas, then condensation to liquid, then fusion to solid.
                \item   Only liquid, but it would go to a point where liquid and gas are indistinguishable.
            \end{enumerate}
        
    \pagebreak
    \tableofcontents
\end{document}