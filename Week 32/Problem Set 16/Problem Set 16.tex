\documentclass[10pt]{article}
\usepackage[export]{adjustbox}
\usepackage{amsmath}
\usepackage{array}
\usepackage[makeroom]{cancel}
\usepackage{chemfig}
\usepackage{enumitem}
\usepackage{float}
\usepackage{graphicx}
%Load mhchem using some package options
\usepackage[version=4,textfontname=sffamily,mathfontname=mathsf]{mhchem}
\usepackage{multicol}
\usepackage{siunitx}
\usepackage{wrapfig}

\title{
    Problem Set \#16
    \\  \small
    CHEM101A: General College Chemistry
    }
\author{Donald Aingworth}
\date{December 8, 2025}

\newcommand{\E}[1]{\times 10^{#1}}
\newcommand{\hc}{1.9864748\E{-25}\,\unit{\joule\,\meter}}
\newcommand{\pH}{\text{pH}}
\newcommand{\U}[1]{\underline{#1}}

\begin{document}
    \DeclareSIUnit{\atm}{atm}
    \DeclareSIUnit{\molar}{M}
    \DeclareSIUnit{\M}{M}
    \DeclareSIUnit{\torr}{torr}

    \maketitle

    % \setcounter{section}{22}
    \pagebreak
    \section{Topic H Problem 1}
        Classify each of the following substances as molecular, ionic, or metallic, and explain your answers. 
        (There are no network covalent substances in this list.) 
        Base your answers on the chemical formulas and the positions of the elements in the periodic table: you should not need to look up any additional information.
        \begin{enumerate}[label=\alph*]
            \item   \ce{Cl2S}
            \item   \ce{K2S}
            \item   \ce{H2SO4}
            \item   \ce{Fe}
            \item   \ce{He}
        \end{enumerate}

        \subsection{Solution}
        
        
    \pagebreak
    \section{Topic H Problem 2}
        Each of the following substances is a solid at room temperature. 
        What would you expect to be the most important type of attractive force between individual formula units in each of these substances? 
        Base your answers on the chemical formulas and the positions of the elements in the periodic table: you should not need to look up any additional information.
        \begin{enumerate}
            \item   \ce{CBr4}
            \item   \ce{CaBr2}
            \item   \ce{Ca}
            \item   \ce{C}
        \end{enumerate}

        \subsection{Solution}
        
        
    \pagebreak
    \section{Topic H Problem 3}
        For each of the following molecular substances, list all of the intermolecular forces that play a significant role in determining the physical properties of the substance. 
        \textit{Reminder: the three types of intermolecular forces are London dispersion forces, dipole-dipole attraction, and hydrogen bonding.}
        \begin{enumerate}
            \item   \ce{NH3}
            \item   \ce{N2}
            \item   \ce{NF3}
            \item   \ce{CH4}
            \item   \ce{CH2O}
        \end{enumerate}

        \subsection{Solution}
        
        
    \pagebreak
    \section{Topic H Problem 4}
        From each of the following pairs of compounds, select the compound that should have the higher boiling point, and explain your choice.
        \begin{enumerate}
            \item   \ce{CO} and \ce{BeO}
            \item   \ce{NaF} and \ce{MgO}
        \end{enumerate}

        \subsection{Solution}
        
        
    \pagebreak
    \section{Topic H Problem 5}
        \ce{HCl} is a strong electrolyte (as you already know!). 
        Its melting point and boiling point are $-114\unit{\celsius}$ and $-85\unit{\celsius}$, respectively.
        \begin{enumerate}
            \item   What state (solid/liquid/gas) is \ce{HCl} in at room temperature?
            \item   A student concludes that HCl is an ionic compound, based on the fact that it ionizes completely in aqueous solution. 
            Would you agree with this conclusion? 
            If not, what sort of compound is HCl? 
            Be sure to discuss the melting and boiling points in your answer.
        \end{enumerate}

        \subsection{Solution}
        
        
    \pagebreak
    \section{Topic H Problem 6}
        Using your knowledge of intermolecular forces, explain the trend in boiling points in the following series of compounds.
        \begin{multicols}{4}
            \resizebox{\textwidth/6}{!}{\chemname{\chemfig{H-[:0]C(-[:90]H)(-[:-90]H)-[:0]\charge{90=\:,0=\:,-90=\:}{F}}}{Fluoromethane}}\\
            boils at $-78\unit{\celsius}$

            \resizebox{\textwidth/6}{!}{\chemname{\chemfig{H-[:0]C(-[:90]H)(-[:-90]H)-[:0]\charge{90=\:,0=\:,-90=\:}{Cl}}}{Chloromethane}}\\
            boils at $-24\unit{\celsius}$

            \resizebox{\textwidth/6}{!}{\chemname{\chemfig{H-[:0]C(-[:90]H)(-[:-90]H)-[:0]\charge{90=\:,0=\:,-90=\:}{Br}}}{Bromethane}}\\
            boils at $4\unit{\celsius}$

            \resizebox{\textwidth/6}{!}{\chemname{\chemfig{H-[:0]C(-[:90]H)(-[:-90]H)-[:0]\charge{90=\:,0=\:,-90=\:}{I}}}{Idomethane}}\\
            boils at $42\unit{\celsius}$
        \end{multicols}

        \subsection{Solution}
        
        
    \pagebreak
    \section{Topic H Problem 7}
        One of the compounds below boils at $70\unit{\celsius}$, while the other boils at $116\unit{\celsius}$. 
        Which is which?
        Explain your reasoning.
        \begin{multicols}{2}
            \chemname{\chemfig{\charge{-90=\:,90=\:}{O}=C*6(-C(-H)=C(-H)-C(=\charge{-90=\:,90=\:}{O})-C(-H)=C(-H)-)}}{compound 1}
            
            \chemname{\chemfig{H-C*6(=C(-H)-C(=\charge{-150=\:,30=\:}{O})-C(=\charge{-90=\:,90=\:}{O})-C(-H)=C(-H)-)}}{compound 2}
        \end{multicols}

        \subsection{Solution}
        
        
    \pagebreak
    \section{Topic H Problem 8}
        The following three compounds are isomers (they have the same chemical formula), but they have quite different boiling points: one boils at $82\unit{\celsius}$, one at $135\unit{\celsius}$, and one at $235\unit{\celsius}$. 
        Match each substance with the correct boiling point, and explain your reasoning.
        \begin{center}
            \chemname{\chemfig{H-\charge{-90=\:,90=\:}{O}-C(-[:90]H)(-[:-90]H)-C(-[:90]H)(-[:-90]H)-C(-[:90]H)(-[:-90]H)-C(-[:90]H)(-[:-90]H)-\charge{-90=\:,90=\:}{O}-H}}{compound 1}

            \chemname{\chemfig{H-C(-[:90]H)(-[:-90]H)-C(-[:90]H)(-[:-90]H)-\charge{-90=\:,90=\:}{O}-C(-[:90]H)(-[:-90]H)-C(-[:90]H)(-[:-90]H)-\charge{-90=\:,90=\:}{O}-H}}{compound 2}

            \chemname{\chemfig{H-C(-[:90]H)(-[:-90]H)-\charge{-90=\:,90=\:}{O}-C(-[:90]H)(-[:-90]H)-C(-[:90]H)(-[:-90]H)-\charge{-90=\:,90=\:}{O}-C(-[:90]H)(-[:-90]H)-H}}{compound 3}
        \end{center}
        
        \subsection{Solution}
        
        
    \pagebreak
    \section{Topic H Problem 9}
        The phase diagram for ammonia is shown below (not drawn to scale).
        \begin{center}
            \includegraphics[width=\textwidth]{P9graph.png}
        \end{center}
        \begin{enumerate}
            \item   What state is ammonia in at 25°C and 1 atm?
            \item   What state is ammonia in at -80°C and 10 atm?
            \item   What are the temperature and pressure at the triple point of ammonia?
            \item   What are the temperature and pressure at the critical point of ammonia?
            \item   If you heat ammonia from $-100\unit{\celsius}$ to $150\unit{\celsius}$ at a pressure of 5 atm, what states will you observe?
            \item   If you heat ammonia from $-100\unit{\celsius}$ to $150\unit{\celsius}$ at a pressure of 0.01 atm, what states will you observe?
            \item   A container holds ammonia at $-77.70\unit{\celsius}$ and a pressure of 0.010 atm. If the pressure is increased to 150 atm without changing the temperature, what states will you observe?
            \item   If you increase the temperature of ammonia from $125\unit{\celsius}$ to $175\unit{\celsius}$ at a pressure of 200 atm, what states will you observe?
        \end{enumerate}

        \subsection{Solution}
        
        
    \pagebreak
    \tableofcontents
\end{document}