\documentclass[11pt]{article}
\usepackage{amsmath}
\usepackage{mhchem}
\usepackage{siunitx}
\usepackage{titlesec}


\begin{document}

\title{The Molarity Limiter}
\author{}
\date{}
\maketitle

\section{Introduction and Abstract}
When doing stoichometric calculations, there is a point where you can determine the limiting reactant without completely calculating the mass of the products.
This document is about documenting that point and resultant value (which we will call the ``Molarity Limiter'') and its applications to CHEM 40 (Chapter 8).
Said Molarity Limiter, denoted $ML$, can be found with the following equation, for the mass $m$\footnote{$MM(J)$ denotes the molar mass of J} of a chemical $J$ with stoichiometric coefficient $c$.
\begin{equation}
    ML = m * \frac{1}{MM(J)}* \frac{1}{c}
\end{equation}

\section{Derivation}
Suppose a chemical reaction with the following general formula:
\begin{equation}
    n_1\ce{A} + n_2\ce{B} \to n_3\ce{C} + n_4\ce{D}
\end{equation}

Suppose \ce{A} has a limited total mass $m_A$ and \ce{B} has a limited total mass $m_B$. 


\end{document}